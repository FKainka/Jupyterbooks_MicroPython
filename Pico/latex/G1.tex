\documentclass[11pt]{article}

    \usepackage[breakable]{tcolorbox}
    \usepackage{parskip} % Stop auto-indenting (to mimic markdown behaviour)
    
    \usepackage{iftex}
    \ifPDFTeX
    	\usepackage[T1]{fontenc}
    	\usepackage{mathpazo}
    \else
    	\usepackage{fontspec}
    \fi

    % Basic figure setup, for now with no caption control since it's done
    % automatically by Pandoc (which extracts ![](path) syntax from Markdown).
    \usepackage{graphicx}
    % Maintain compatibility with old templates. Remove in nbconvert 6.0
    \let\Oldincludegraphics\includegraphics
    % Ensure that by default, figures have no caption (until we provide a
    % proper Figure object with a Caption API and a way to capture that
    % in the conversion process - todo).
    \usepackage{caption}
    \DeclareCaptionFormat{nocaption}{}
    \captionsetup{format=nocaption,aboveskip=0pt,belowskip=0pt}

    \usepackage{float}
    \floatplacement{figure}{H} % forces figures to be placed at the correct location
    \usepackage{xcolor} % Allow colors to be defined
    \usepackage{enumerate} % Needed for markdown enumerations to work
    \usepackage{geometry} % Used to adjust the document margins
    \usepackage{amsmath} % Equations
    \usepackage{amssymb} % Equations
    \usepackage{textcomp} % defines textquotesingle
    % Hack from http://tex.stackexchange.com/a/47451/13684:
    \AtBeginDocument{%
        \def\PYZsq{\textquotesingle}% Upright quotes in Pygmentized code
    }
    \usepackage{upquote} % Upright quotes for verbatim code
    \usepackage{eurosym} % defines \euro
    \usepackage[mathletters]{ucs} % Extended unicode (utf-8) support
    \usepackage{fancyvrb} % verbatim replacement that allows latex
    \usepackage{grffile} % extends the file name processing of package graphics 
                         % to support a larger range
    \makeatletter % fix for old versions of grffile with XeLaTeX
    \@ifpackagelater{grffile}{2019/11/01}
    {
      % Do nothing on new versions
    }
    {
      \def\Gread@@xetex#1{%
        \IfFileExists{"\Gin@base".bb}%
        {\Gread@eps{\Gin@base.bb}}%
        {\Gread@@xetex@aux#1}%
      }
    }
    \makeatother
    \usepackage[Export]{adjustbox} % Used to constrain images to a maximum size
    \adjustboxset{max size={0.9\linewidth}{0.9\paperheight}}

    % The hyperref package gives us a pdf with properly built
    % internal navigation ('pdf bookmarks' for the table of contents,
    % internal cross-reference links, web links for URLs, etc.)
    \usepackage{hyperref}
    % The default LaTeX title has an obnoxious amount of whitespace. By default,
    % titling removes some of it. It also provides customization options.
    \usepackage{titling}
    \usepackage{longtable} % longtable support required by pandoc >1.10
    \usepackage{booktabs}  % table support for pandoc > 1.12.2
    \usepackage[inline]{enumitem} % IRkernel/repr support (it uses the enumerate* environment)
    \usepackage[normalem]{ulem} % ulem is needed to support strikethroughs (\sout)
                                % normalem makes italics be italics, not underlines
    \usepackage{mathrsfs}
    

    
    % Colors for the hyperref package
    \definecolor{urlcolor}{rgb}{0,.145,.698}
    \definecolor{linkcolor}{rgb}{.71,0.21,0.01}
    \definecolor{citecolor}{rgb}{.12,.54,.11}

    % ANSI colors
    \definecolor{ansi-black}{HTML}{3E424D}
    \definecolor{ansi-black-intense}{HTML}{282C36}
    \definecolor{ansi-red}{HTML}{E75C58}
    \definecolor{ansi-red-intense}{HTML}{B22B31}
    \definecolor{ansi-green}{HTML}{00A250}
    \definecolor{ansi-green-intense}{HTML}{007427}
    \definecolor{ansi-yellow}{HTML}{DDB62B}
    \definecolor{ansi-yellow-intense}{HTML}{B27D12}
    \definecolor{ansi-blue}{HTML}{208FFB}
    \definecolor{ansi-blue-intense}{HTML}{0065CA}
    \definecolor{ansi-magenta}{HTML}{D160C4}
    \definecolor{ansi-magenta-intense}{HTML}{A03196}
    \definecolor{ansi-cyan}{HTML}{60C6C8}
    \definecolor{ansi-cyan-intense}{HTML}{258F8F}
    \definecolor{ansi-white}{HTML}{C5C1B4}
    \definecolor{ansi-white-intense}{HTML}{A1A6B2}
    \definecolor{ansi-default-inverse-fg}{HTML}{FFFFFF}
    \definecolor{ansi-default-inverse-bg}{HTML}{000000}

    % common color for the border for error outputs.
    \definecolor{outerrorbackground}{HTML}{FFDFDF}

    % commands and environments needed by pandoc snippets
    % extracted from the output of `pandoc -s`
    \providecommand{\tightlist}{%
      \setlength{\itemsep}{0pt}\setlength{\parskip}{0pt}}
    \DefineVerbatimEnvironment{Highlighting}{Verbatim}{commandchars=\\\{\}}
    % Add ',fontsize=\small' for more characters per line
    \newenvironment{Shaded}{}{}
    \newcommand{\KeywordTok}[1]{\textcolor[rgb]{0.00,0.44,0.13}{\textbf{{#1}}}}
    \newcommand{\DataTypeTok}[1]{\textcolor[rgb]{0.56,0.13,0.00}{{#1}}}
    \newcommand{\DecValTok}[1]{\textcolor[rgb]{0.25,0.63,0.44}{{#1}}}
    \newcommand{\BaseNTok}[1]{\textcolor[rgb]{0.25,0.63,0.44}{{#1}}}
    \newcommand{\FloatTok}[1]{\textcolor[rgb]{0.25,0.63,0.44}{{#1}}}
    \newcommand{\CharTok}[1]{\textcolor[rgb]{0.25,0.44,0.63}{{#1}}}
    \newcommand{\StringTok}[1]{\textcolor[rgb]{0.25,0.44,0.63}{{#1}}}
    \newcommand{\CommentTok}[1]{\textcolor[rgb]{0.38,0.63,0.69}{\textit{{#1}}}}
    \newcommand{\OtherTok}[1]{\textcolor[rgb]{0.00,0.44,0.13}{{#1}}}
    \newcommand{\AlertTok}[1]{\textcolor[rgb]{1.00,0.00,0.00}{\textbf{{#1}}}}
    \newcommand{\FunctionTok}[1]{\textcolor[rgb]{0.02,0.16,0.49}{{#1}}}
    \newcommand{\RegionMarkerTok}[1]{{#1}}
    \newcommand{\ErrorTok}[1]{\textcolor[rgb]{1.00,0.00,0.00}{\textbf{{#1}}}}
    \newcommand{\NormalTok}[1]{{#1}}
    
    % Additional commands for more recent versions of Pandoc
    \newcommand{\ConstantTok}[1]{\textcolor[rgb]{0.53,0.00,0.00}{{#1}}}
    \newcommand{\SpecialCharTok}[1]{\textcolor[rgb]{0.25,0.44,0.63}{{#1}}}
    \newcommand{\VerbatimStringTok}[1]{\textcolor[rgb]{0.25,0.44,0.63}{{#1}}}
    \newcommand{\SpecialStringTok}[1]{\textcolor[rgb]{0.73,0.40,0.53}{{#1}}}
    \newcommand{\ImportTok}[1]{{#1}}
    \newcommand{\DocumentationTok}[1]{\textcolor[rgb]{0.73,0.13,0.13}{\textit{{#1}}}}
    \newcommand{\AnnotationTok}[1]{\textcolor[rgb]{0.38,0.63,0.69}{\textbf{\textit{{#1}}}}}
    \newcommand{\CommentVarTok}[1]{\textcolor[rgb]{0.38,0.63,0.69}{\textbf{\textit{{#1}}}}}
    \newcommand{\VariableTok}[1]{\textcolor[rgb]{0.10,0.09,0.49}{{#1}}}
    \newcommand{\ControlFlowTok}[1]{\textcolor[rgb]{0.00,0.44,0.13}{\textbf{{#1}}}}
    \newcommand{\OperatorTok}[1]{\textcolor[rgb]{0.40,0.40,0.40}{{#1}}}
    \newcommand{\BuiltInTok}[1]{{#1}}
    \newcommand{\ExtensionTok}[1]{{#1}}
    \newcommand{\PreprocessorTok}[1]{\textcolor[rgb]{0.74,0.48,0.00}{{#1}}}
    \newcommand{\AttributeTok}[1]{\textcolor[rgb]{0.49,0.56,0.16}{{#1}}}
    \newcommand{\InformationTok}[1]{\textcolor[rgb]{0.38,0.63,0.69}{\textbf{\textit{{#1}}}}}
    \newcommand{\WarningTok}[1]{\textcolor[rgb]{0.38,0.63,0.69}{\textbf{\textit{{#1}}}}}
    
    
    % Define a nice break command that doesn't care if a line doesn't already
    % exist.
    \def\br{\hspace*{\fill} \\* }
    % Math Jax compatibility definitions
    \def\gt{>}
    \def\lt{<}
    \let\Oldtex\TeX
    \let\Oldlatex\LaTeX
    \renewcommand{\TeX}{\textrm{\Oldtex}}
    \renewcommand{\LaTeX}{\textrm{\Oldlatex}}
    % Document parameters
    % Document title
    \title{1\_GrundlagenMicroPython}
    
    
    
    
    
% Pygments definitions
\makeatletter
\def\PY@reset{\let\PY@it=\relax \let\PY@bf=\relax%
    \let\PY@ul=\relax \let\PY@tc=\relax%
    \let\PY@bc=\relax \let\PY@ff=\relax}
\def\PY@tok#1{\csname PY@tok@#1\endcsname}
\def\PY@toks#1+{\ifx\relax#1\empty\else%
    \PY@tok{#1}\expandafter\PY@toks\fi}
\def\PY@do#1{\PY@bc{\PY@tc{\PY@ul{%
    \PY@it{\PY@bf{\PY@ff{#1}}}}}}}
\def\PY#1#2{\PY@reset\PY@toks#1+\relax+\PY@do{#2}}

\@namedef{PY@tok@w}{\def\PY@tc##1{\textcolor[rgb]{0.73,0.73,0.73}{##1}}}
\@namedef{PY@tok@c}{\let\PY@it=\textit\def\PY@tc##1{\textcolor[rgb]{0.25,0.50,0.50}{##1}}}
\@namedef{PY@tok@cp}{\def\PY@tc##1{\textcolor[rgb]{0.74,0.48,0.00}{##1}}}
\@namedef{PY@tok@k}{\let\PY@bf=\textbf\def\PY@tc##1{\textcolor[rgb]{0.00,0.50,0.00}{##1}}}
\@namedef{PY@tok@kp}{\def\PY@tc##1{\textcolor[rgb]{0.00,0.50,0.00}{##1}}}
\@namedef{PY@tok@kt}{\def\PY@tc##1{\textcolor[rgb]{0.69,0.00,0.25}{##1}}}
\@namedef{PY@tok@o}{\def\PY@tc##1{\textcolor[rgb]{0.40,0.40,0.40}{##1}}}
\@namedef{PY@tok@ow}{\let\PY@bf=\textbf\def\PY@tc##1{\textcolor[rgb]{0.67,0.13,1.00}{##1}}}
\@namedef{PY@tok@nb}{\def\PY@tc##1{\textcolor[rgb]{0.00,0.50,0.00}{##1}}}
\@namedef{PY@tok@nf}{\def\PY@tc##1{\textcolor[rgb]{0.00,0.00,1.00}{##1}}}
\@namedef{PY@tok@nc}{\let\PY@bf=\textbf\def\PY@tc##1{\textcolor[rgb]{0.00,0.00,1.00}{##1}}}
\@namedef{PY@tok@nn}{\let\PY@bf=\textbf\def\PY@tc##1{\textcolor[rgb]{0.00,0.00,1.00}{##1}}}
\@namedef{PY@tok@ne}{\let\PY@bf=\textbf\def\PY@tc##1{\textcolor[rgb]{0.82,0.25,0.23}{##1}}}
\@namedef{PY@tok@nv}{\def\PY@tc##1{\textcolor[rgb]{0.10,0.09,0.49}{##1}}}
\@namedef{PY@tok@no}{\def\PY@tc##1{\textcolor[rgb]{0.53,0.00,0.00}{##1}}}
\@namedef{PY@tok@nl}{\def\PY@tc##1{\textcolor[rgb]{0.63,0.63,0.00}{##1}}}
\@namedef{PY@tok@ni}{\let\PY@bf=\textbf\def\PY@tc##1{\textcolor[rgb]{0.60,0.60,0.60}{##1}}}
\@namedef{PY@tok@na}{\def\PY@tc##1{\textcolor[rgb]{0.49,0.56,0.16}{##1}}}
\@namedef{PY@tok@nt}{\let\PY@bf=\textbf\def\PY@tc##1{\textcolor[rgb]{0.00,0.50,0.00}{##1}}}
\@namedef{PY@tok@nd}{\def\PY@tc##1{\textcolor[rgb]{0.67,0.13,1.00}{##1}}}
\@namedef{PY@tok@s}{\def\PY@tc##1{\textcolor[rgb]{0.73,0.13,0.13}{##1}}}
\@namedef{PY@tok@sd}{\let\PY@it=\textit\def\PY@tc##1{\textcolor[rgb]{0.73,0.13,0.13}{##1}}}
\@namedef{PY@tok@si}{\let\PY@bf=\textbf\def\PY@tc##1{\textcolor[rgb]{0.73,0.40,0.53}{##1}}}
\@namedef{PY@tok@se}{\let\PY@bf=\textbf\def\PY@tc##1{\textcolor[rgb]{0.73,0.40,0.13}{##1}}}
\@namedef{PY@tok@sr}{\def\PY@tc##1{\textcolor[rgb]{0.73,0.40,0.53}{##1}}}
\@namedef{PY@tok@ss}{\def\PY@tc##1{\textcolor[rgb]{0.10,0.09,0.49}{##1}}}
\@namedef{PY@tok@sx}{\def\PY@tc##1{\textcolor[rgb]{0.00,0.50,0.00}{##1}}}
\@namedef{PY@tok@m}{\def\PY@tc##1{\textcolor[rgb]{0.40,0.40,0.40}{##1}}}
\@namedef{PY@tok@gh}{\let\PY@bf=\textbf\def\PY@tc##1{\textcolor[rgb]{0.00,0.00,0.50}{##1}}}
\@namedef{PY@tok@gu}{\let\PY@bf=\textbf\def\PY@tc##1{\textcolor[rgb]{0.50,0.00,0.50}{##1}}}
\@namedef{PY@tok@gd}{\def\PY@tc##1{\textcolor[rgb]{0.63,0.00,0.00}{##1}}}
\@namedef{PY@tok@gi}{\def\PY@tc##1{\textcolor[rgb]{0.00,0.63,0.00}{##1}}}
\@namedef{PY@tok@gr}{\def\PY@tc##1{\textcolor[rgb]{1.00,0.00,0.00}{##1}}}
\@namedef{PY@tok@ge}{\let\PY@it=\textit}
\@namedef{PY@tok@gs}{\let\PY@bf=\textbf}
\@namedef{PY@tok@gp}{\let\PY@bf=\textbf\def\PY@tc##1{\textcolor[rgb]{0.00,0.00,0.50}{##1}}}
\@namedef{PY@tok@go}{\def\PY@tc##1{\textcolor[rgb]{0.53,0.53,0.53}{##1}}}
\@namedef{PY@tok@gt}{\def\PY@tc##1{\textcolor[rgb]{0.00,0.27,0.87}{##1}}}
\@namedef{PY@tok@err}{\def\PY@bc##1{{\setlength{\fboxsep}{\string -\fboxrule}\fcolorbox[rgb]{1.00,0.00,0.00}{1,1,1}{\strut ##1}}}}
\@namedef{PY@tok@kc}{\let\PY@bf=\textbf\def\PY@tc##1{\textcolor[rgb]{0.00,0.50,0.00}{##1}}}
\@namedef{PY@tok@kd}{\let\PY@bf=\textbf\def\PY@tc##1{\textcolor[rgb]{0.00,0.50,0.00}{##1}}}
\@namedef{PY@tok@kn}{\let\PY@bf=\textbf\def\PY@tc##1{\textcolor[rgb]{0.00,0.50,0.00}{##1}}}
\@namedef{PY@tok@kr}{\let\PY@bf=\textbf\def\PY@tc##1{\textcolor[rgb]{0.00,0.50,0.00}{##1}}}
\@namedef{PY@tok@bp}{\def\PY@tc##1{\textcolor[rgb]{0.00,0.50,0.00}{##1}}}
\@namedef{PY@tok@fm}{\def\PY@tc##1{\textcolor[rgb]{0.00,0.00,1.00}{##1}}}
\@namedef{PY@tok@vc}{\def\PY@tc##1{\textcolor[rgb]{0.10,0.09,0.49}{##1}}}
\@namedef{PY@tok@vg}{\def\PY@tc##1{\textcolor[rgb]{0.10,0.09,0.49}{##1}}}
\@namedef{PY@tok@vi}{\def\PY@tc##1{\textcolor[rgb]{0.10,0.09,0.49}{##1}}}
\@namedef{PY@tok@vm}{\def\PY@tc##1{\textcolor[rgb]{0.10,0.09,0.49}{##1}}}
\@namedef{PY@tok@sa}{\def\PY@tc##1{\textcolor[rgb]{0.73,0.13,0.13}{##1}}}
\@namedef{PY@tok@sb}{\def\PY@tc##1{\textcolor[rgb]{0.73,0.13,0.13}{##1}}}
\@namedef{PY@tok@sc}{\def\PY@tc##1{\textcolor[rgb]{0.73,0.13,0.13}{##1}}}
\@namedef{PY@tok@dl}{\def\PY@tc##1{\textcolor[rgb]{0.73,0.13,0.13}{##1}}}
\@namedef{PY@tok@s2}{\def\PY@tc##1{\textcolor[rgb]{0.73,0.13,0.13}{##1}}}
\@namedef{PY@tok@sh}{\def\PY@tc##1{\textcolor[rgb]{0.73,0.13,0.13}{##1}}}
\@namedef{PY@tok@s1}{\def\PY@tc##1{\textcolor[rgb]{0.73,0.13,0.13}{##1}}}
\@namedef{PY@tok@mb}{\def\PY@tc##1{\textcolor[rgb]{0.40,0.40,0.40}{##1}}}
\@namedef{PY@tok@mf}{\def\PY@tc##1{\textcolor[rgb]{0.40,0.40,0.40}{##1}}}
\@namedef{PY@tok@mh}{\def\PY@tc##1{\textcolor[rgb]{0.40,0.40,0.40}{##1}}}
\@namedef{PY@tok@mi}{\def\PY@tc##1{\textcolor[rgb]{0.40,0.40,0.40}{##1}}}
\@namedef{PY@tok@il}{\def\PY@tc##1{\textcolor[rgb]{0.40,0.40,0.40}{##1}}}
\@namedef{PY@tok@mo}{\def\PY@tc##1{\textcolor[rgb]{0.40,0.40,0.40}{##1}}}
\@namedef{PY@tok@ch}{\let\PY@it=\textit\def\PY@tc##1{\textcolor[rgb]{0.25,0.50,0.50}{##1}}}
\@namedef{PY@tok@cm}{\let\PY@it=\textit\def\PY@tc##1{\textcolor[rgb]{0.25,0.50,0.50}{##1}}}
\@namedef{PY@tok@cpf}{\let\PY@it=\textit\def\PY@tc##1{\textcolor[rgb]{0.25,0.50,0.50}{##1}}}
\@namedef{PY@tok@c1}{\let\PY@it=\textit\def\PY@tc##1{\textcolor[rgb]{0.25,0.50,0.50}{##1}}}
\@namedef{PY@tok@cs}{\let\PY@it=\textit\def\PY@tc##1{\textcolor[rgb]{0.25,0.50,0.50}{##1}}}

\def\PYZbs{\char`\\}
\def\PYZus{\char`\_}
\def\PYZob{\char`\{}
\def\PYZcb{\char`\}}
\def\PYZca{\char`\^}
\def\PYZam{\char`\&}
\def\PYZlt{\char`\<}
\def\PYZgt{\char`\>}
\def\PYZsh{\char`\#}
\def\PYZpc{\char`\%}
\def\PYZdl{\char`\$}
\def\PYZhy{\char`\-}
\def\PYZsq{\char`\'}
\def\PYZdq{\char`\"}
\def\PYZti{\char`\~}
% for compatibility with earlier versions
\def\PYZat{@}
\def\PYZlb{[}
\def\PYZrb{]}
\makeatother


    % For linebreaks inside Verbatim environment from package fancyvrb. 
    \makeatletter
        \newbox\Wrappedcontinuationbox 
        \newbox\Wrappedvisiblespacebox 
        \newcommand*\Wrappedvisiblespace {\textcolor{red}{\textvisiblespace}} 
        \newcommand*\Wrappedcontinuationsymbol {\textcolor{red}{\llap{\tiny$\m@th\hookrightarrow$}}} 
        \newcommand*\Wrappedcontinuationindent {3ex } 
        \newcommand*\Wrappedafterbreak {\kern\Wrappedcontinuationindent\copy\Wrappedcontinuationbox} 
        % Take advantage of the already applied Pygments mark-up to insert 
        % potential linebreaks for TeX processing. 
        %        {, <, #, %, $, ' and ": go to next line. 
        %        _, }, ^, &, >, - and ~: stay at end of broken line. 
        % Use of \textquotesingle for straight quote. 
        \newcommand*\Wrappedbreaksatspecials {% 
            \def\PYGZus{\discretionary{\char`\_}{\Wrappedafterbreak}{\char`\_}}% 
            \def\PYGZob{\discretionary{}{\Wrappedafterbreak\char`\{}{\char`\{}}% 
            \def\PYGZcb{\discretionary{\char`\}}{\Wrappedafterbreak}{\char`\}}}% 
            \def\PYGZca{\discretionary{\char`\^}{\Wrappedafterbreak}{\char`\^}}% 
            \def\PYGZam{\discretionary{\char`\&}{\Wrappedafterbreak}{\char`\&}}% 
            \def\PYGZlt{\discretionary{}{\Wrappedafterbreak\char`\<}{\char`\<}}% 
            \def\PYGZgt{\discretionary{\char`\>}{\Wrappedafterbreak}{\char`\>}}% 
            \def\PYGZsh{\discretionary{}{\Wrappedafterbreak\char`\#}{\char`\#}}% 
            \def\PYGZpc{\discretionary{}{\Wrappedafterbreak\char`\%}{\char`\%}}% 
            \def\PYGZdl{\discretionary{}{\Wrappedafterbreak\char`\$}{\char`\$}}% 
            \def\PYGZhy{\discretionary{\char`\-}{\Wrappedafterbreak}{\char`\-}}% 
            \def\PYGZsq{\discretionary{}{\Wrappedafterbreak\textquotesingle}{\textquotesingle}}% 
            \def\PYGZdq{\discretionary{}{\Wrappedafterbreak\char`\"}{\char`\"}}% 
            \def\PYGZti{\discretionary{\char`\~}{\Wrappedafterbreak}{\char`\~}}% 
        } 
        % Some characters . , ; ? ! / are not pygmentized. 
        % This macro makes them "active" and they will insert potential linebreaks 
        \newcommand*\Wrappedbreaksatpunct {% 
            \lccode`\~`\.\lowercase{\def~}{\discretionary{\hbox{\char`\.}}{\Wrappedafterbreak}{\hbox{\char`\.}}}% 
            \lccode`\~`\,\lowercase{\def~}{\discretionary{\hbox{\char`\,}}{\Wrappedafterbreak}{\hbox{\char`\,}}}% 
            \lccode`\~`\;\lowercase{\def~}{\discretionary{\hbox{\char`\;}}{\Wrappedafterbreak}{\hbox{\char`\;}}}% 
            \lccode`\~`\:\lowercase{\def~}{\discretionary{\hbox{\char`\:}}{\Wrappedafterbreak}{\hbox{\char`\:}}}% 
            \lccode`\~`\?\lowercase{\def~}{\discretionary{\hbox{\char`\?}}{\Wrappedafterbreak}{\hbox{\char`\?}}}% 
            \lccode`\~`\!\lowercase{\def~}{\discretionary{\hbox{\char`\!}}{\Wrappedafterbreak}{\hbox{\char`\!}}}% 
            \lccode`\~`\/\lowercase{\def~}{\discretionary{\hbox{\char`\/}}{\Wrappedafterbreak}{\hbox{\char`\/}}}% 
            \catcode`\.\active
            \catcode`\,\active 
            \catcode`\;\active
            \catcode`\:\active
            \catcode`\?\active
            \catcode`\!\active
            \catcode`\/\active 
            \lccode`\~`\~ 	
        }
    \makeatother

    \let\OriginalVerbatim=\Verbatim
    \makeatletter
    \renewcommand{\Verbatim}[1][1]{%
        %\parskip\z@skip
        \sbox\Wrappedcontinuationbox {\Wrappedcontinuationsymbol}%
        \sbox\Wrappedvisiblespacebox {\FV@SetupFont\Wrappedvisiblespace}%
        \def\FancyVerbFormatLine ##1{\hsize\linewidth
            \vtop{\raggedright\hyphenpenalty\z@\exhyphenpenalty\z@
                \doublehyphendemerits\z@\finalhyphendemerits\z@
                \strut ##1\strut}%
        }%
        % If the linebreak is at a space, the latter will be displayed as visible
        % space at end of first line, and a continuation symbol starts next line.
        % Stretch/shrink are however usually zero for typewriter font.
        \def\FV@Space {%
            \nobreak\hskip\z@ plus\fontdimen3\font minus\fontdimen4\font
            \discretionary{\copy\Wrappedvisiblespacebox}{\Wrappedafterbreak}
            {\kern\fontdimen2\font}%
        }%
        
        % Allow breaks at special characters using \PYG... macros.
        \Wrappedbreaksatspecials
        % Breaks at punctuation characters . , ; ? ! and / need catcode=\active 	
        \OriginalVerbatim[#1,codes*=\Wrappedbreaksatpunct]%
    }
    \makeatother

    % Exact colors from NB
    \definecolor{incolor}{HTML}{303F9F}
    \definecolor{outcolor}{HTML}{D84315}
    \definecolor{cellborder}{HTML}{CFCFCF}
    \definecolor{cellbackground}{HTML}{F7F7F7}
    
    % prompt
    \makeatletter
    \newcommand{\boxspacing}{\kern\kvtcb@left@rule\kern\kvtcb@boxsep}
    \makeatother
    \newcommand{\prompt}[4]{
        {\ttfamily\llap{{\color{#2}[#3]:\hspace{3pt}#4}}\vspace{-\baselineskip}}
    }
    

    
    % Prevent overflowing lines due to hard-to-break entities
    \sloppy 
    % Setup hyperref package
    \hypersetup{
      breaklinks=true,  % so long urls are correctly broken across lines
      colorlinks=true,
      urlcolor=urlcolor,
      linkcolor=linkcolor,
      citecolor=citecolor,
      }
    % Slightly bigger margins than the latex defaults
    
    \geometry{verbose,tmargin=1in,bmargin=1in,lmargin=1in,rmargin=1in}
    
    

\begin{document}
    
    \maketitle
    
    

    
    \hypertarget{python-micropython-einstieg}{%
\section{Python / MicroPython
Einstieg}\label{python-micropython-einstieg}}

\begin{itemize}
\tightlist
\item
  \emph{\textbf{Board: RPi Pico}}
\item
  \emph{\textbf{Firmware: Micropython 1.14 (2021-02-05)}}
\item
  \emph{\textbf{Kernel: MicroPyhton - USB}}
\end{itemize}

Dieses Notebook soll als Einstieg in Pyhton bzw. konkret in MicroPyhton
dienen. Es werden vor allem Grundlegende Befehle und Keywords behandelt,
sowie die builtin-Funktionen und Klassen.

\hypertarget{muxf6gliche-grundlagen-pyhtonmicropyhton-themen}{%
\subsubsection{Mögliche Grundlagen Pyhton/MicroPyhton
Themen}\label{muxf6gliche-grundlagen-pyhtonmicropyhton-themen}}

Hier: - exec, eval, keywords allgemein, pass - Generelle Befehle -
buildins - klassen : hasattr. type, isinstance

\hypertarget{todo}{%
\subsubsection{ToDo:}\label{todo}}

\begin{verbatim}
- Klassen
- Keywords ergänzen
\end{verbatim}

    \hypertarget{connect}{%
\section{Connect}\label{connect}}

    \begin{tcolorbox}[breakable, size=fbox, boxrule=1pt, pad at break*=1mm,colback=cellbackground, colframe=cellborder]
\prompt{In}{incolor}{2}{\boxspacing}
\begin{Verbatim}[commandchars=\\\{\}]
\PY{o}{\PYZpc{}}\PY{k}{serialconnect} \PYZhy{}\PYZhy{}port=COM4
\end{Verbatim}
\end{tcolorbox}

    \begin{Verbatim}[commandchars=\\\{\}]
\textcolor{ansi-blue}{Connecting to --port=COM4 --baud=115200 }
\textcolor{ansi-blue}{Ready.
}
    \end{Verbatim}

    \hypertarget{erste-schritte}{%
\section{Erste Schritte}\label{erste-schritte}}

    \begin{tcolorbox}[breakable, size=fbox, boxrule=1pt, pad at break*=1mm,colback=cellbackground, colframe=cellborder]
\prompt{In}{incolor}{19}{\boxspacing}
\begin{Verbatim}[commandchars=\\\{\}]
\PY{n+nb}{print}\PY{p}{(}\PY{l+s+s2}{\PYZdq{}}\PY{l+s+s2}{Hello World!}\PY{l+s+s2}{\PYZdq{}}\PY{p}{)}
\PY{n+nb}{print}\PY{p}{(}\PY{l+s+s2}{\PYZdq{}}\PY{l+s+se}{\PYZbs{}n}\PY{l+s+s2}{\PYZdq{}}\PY{p}{)}   \PY{c+c1}{\PYZsh{} Sonderzeichen für neue Zeile}
\PY{n+nb}{print}\PY{p}{(}\PY{l+s+s2}{\PYZdq{}}\PY{l+s+s2}{Neue Zeile}\PY{l+s+s2}{\PYZdq{}}\PY{p}{)}
\end{Verbatim}
\end{tcolorbox}

    \begin{Verbatim}[commandchars=\\\{\}]
Hello World!


Neue Zeile
    \end{Verbatim}

    \begin{tcolorbox}[breakable, size=fbox, boxrule=1pt, pad at break*=1mm,colback=cellbackground, colframe=cellborder]
\prompt{In}{incolor}{10}{\boxspacing}
\begin{Verbatim}[commandchars=\\\{\}]
\PY{c+c1}{\PYZsh{} Kommentare mit }

\PY{n+nb}{print}\PY{p}{(}\PY{l+s+s2}{\PYZdq{}}\PY{l+s+s2}{Hello World}\PY{l+s+s2}{\PYZdq{}}\PY{p}{)} \PY{c+c1}{\PYZsh{} Kommentar zu dieser Zeile}

\PY{c+c1}{\PYZsh{} Mehrzeilige Kommentare existieren nicht in Python. Ein workaround ist die Verwendung eines mehrzeiligen Strings, da Python (auch MPython) }
\PY{c+c1}{\PYZsh{} String die keiner Variable zugeordnet wurden ignoriert. }
\PY{l+s+sd}{\PYZdq{}\PYZdq{}\PYZdq{}}
\PY{l+s+sd}{Kommentare über mehrere Zeilen }
\PY{l+s+sd}{Kommentarzeile 1}
\PY{l+s+sd}{Kommentarzeile 2}
\PY{l+s+sd}{....}
\PY{l+s+sd}{\PYZdq{}\PYZdq{}\PYZdq{}}
\end{Verbatim}
\end{tcolorbox}

    \begin{Verbatim}[commandchars=\\\{\}]
Hello World
    \end{Verbatim}

    \begin{tcolorbox}[breakable, size=fbox, boxrule=1pt, pad at break*=1mm,colback=cellbackground, colframe=cellborder]
\prompt{In}{incolor}{52}{\boxspacing}
\begin{Verbatim}[commandchars=\\\{\}]
\PY{c+c1}{\PYZsh{} Einrücken von Code\PYZhy{}Blöcken}
\PY{k}{if} \PY{l+m+mi}{5} \PY{o}{\PYZgt{}} \PY{l+m+mi}{2}\PY{p}{:}
 \PY{n+nb}{print}\PY{p}{(}\PY{l+s+s2}{\PYZdq{}}\PY{l+s+s2}{Five is greater than two!}\PY{l+s+s2}{\PYZdq{}}\PY{p}{)}  \PY{c+c1}{\PYZsh{} Möglich aber kein guter Stil!}
\PY{k}{if} \PY{l+m+mi}{5} \PY{o}{\PYZgt{}} \PY{l+m+mi}{2}\PY{p}{:}
        \PY{n+nb}{print}\PY{p}{(}\PY{l+s+s2}{\PYZdq{}}\PY{l+s+s2}{Five is greater than two!}\PY{l+s+s2}{\PYZdq{}}\PY{p}{)} 
\end{Verbatim}
\end{tcolorbox}

    \begin{Verbatim}[commandchars=\\\{\}]
Five is greater than two!
Five is greater than two!
    \end{Verbatim}

    \begin{tcolorbox}[breakable, size=fbox, boxrule=1pt, pad at break*=1mm,colback=cellbackground, colframe=cellborder]
\prompt{In}{incolor}{5}{\boxspacing}
\begin{Verbatim}[commandchars=\\\{\}]
\PY{c+c1}{\PYZsh{} Verschachteln von Befehlen}
\PY{n+nb}{print}\PY{p}{(}\PY{p}{(}\PY{l+m+mi}{3} \PY{o}{+} \PY{l+m+mi}{5}\PY{p}{)} \PY{o}{*} \PY{l+m+mi}{3}\PY{p}{)}

\PY{c+c1}{\PYZsh{} Mehrere Befehle in eine Zeile}
\PY{n+nb}{print}\PY{p}{(}\PY{l+s+s2}{\PYZdq{}}\PY{l+s+s2}{Zeile 1}\PY{l+s+s2}{\PYZdq{}}\PY{p}{)}\PY{p}{;}\PY{n+nb}{print}\PY{p}{(}\PY{l+s+s2}{\PYZdq{}}\PY{l+s+s2}{Zeile 2}\PY{l+s+s2}{\PYZdq{}}\PY{p}{)}\PY{p}{;}\PY{n+nb}{print}\PY{p}{(}\PY{l+s+s2}{\PYZdq{}}\PY{l+s+s2}{Zeile 3}\PY{l+s+s2}{\PYZdq{}}\PY{p}{)}
\end{Verbatim}
\end{tcolorbox}

    \begin{Verbatim}[commandchars=\\\{\}]
24
Zeile 1
Zeile 2
Zeile 3
    \end{Verbatim}

    \begin{tcolorbox}[breakable, size=fbox, boxrule=1pt, pad at break*=1mm,colback=cellbackground, colframe=cellborder]
\prompt{In}{incolor}{116}{\boxspacing}
\begin{Verbatim}[commandchars=\\\{\}]
\PY{c+c1}{\PYZsh{} Funktionen definieren}

\PY{c+c1}{\PYZsh{}myFunction() \PYZsh{} !! Hier kann die Funktion noch nicht aufgerufen werden}

\PY{k}{def} \PY{n+nf}{myFunction}\PY{p}{(}\PY{p}{)}\PY{p}{:}
    \PY{n+nb}{print}\PY{p}{(}\PY{l+s+s2}{\PYZdq{}}\PY{l+s+s2}{Dies ist meine Funktion}\PY{l+s+s2}{\PYZdq{}}\PY{p}{)}

\PY{n}{myFunction}\PY{p}{(}\PY{p}{)}
\end{Verbatim}
\end{tcolorbox}

    \begin{Verbatim}[commandchars=\\\{\}]
Dies ist meine Funktion
    \end{Verbatim}

    \begin{tcolorbox}[breakable, size=fbox, boxrule=1pt, pad at break*=1mm,colback=cellbackground, colframe=cellborder]
\prompt{In}{incolor}{123}{\boxspacing}
\begin{Verbatim}[commandchars=\\\{\}]
\PY{c+c1}{\PYZsh{} Hilfe\PYZhy{}Funktion}
\PY{n}{help}\PY{p}{(}\PY{p}{)}

\PY{n}{help}\PY{p}{(}\PY{l+s+s2}{\PYZdq{}}\PY{l+s+s2}{modules}\PY{l+s+s2}{\PYZdq{}}\PY{p}{)}

\PY{c+c1}{\PYZsh{} Einbinden eines Modules}
\PY{k+kn}{import} \PY{n+nn}{builtins}
\PY{c+c1}{\PYZsh{} Hilfe zum Modul}
\PY{n}{help}\PY{p}{(}\PY{n}{builtins}\PY{p}{)}
\end{Verbatim}
\end{tcolorbox}

    \begin{Verbatim}[commandchars=\\\{\}]
Welcome to MicroPython!

For online help please visit https://micropython.org/help/.

For access to the hardware use the 'machine' module.  RP2 specific commands
are in the 'rp2' module.

Quick overview of some objects:
  machine.Pin(pin) -- get a pin, eg machine.Pin(0)
  machine.Pin(pin, m, [p]) -- get a pin and configure it for IO mode m, pull
mode p
    methods: init(..), value([v]), high(), low(), irq(handler)
  machine.ADC(pin) -- make an analog object from a pin
    methods: read\_u16()
  machine.PWM(pin) -- make a PWM object from a pin
    methods: deinit(), freq([f]), duty\_u16([d]), duty\_ns([d])
  machine.I2C(id) -- create an I2C object (id=0,1)
    methods: readfrom(addr, buf, stop=True), writeto(addr, buf, stop=True)
             readfrom\_mem(addr, memaddr, arg), writeto\_mem(addr, memaddr, arg)
  machine.SPI(id, baudrate=1000000) -- create an SPI object (id=0,1)
    methods: read(nbytes, write=0x00), write(buf), write\_readinto(wr\_buf,
rd\_buf)
  machine.Timer(freq, callback) -- create a software timer object
    eg: machine.Timer(freq=1, callback=lambda t:print(t))

Pins are numbered 0-29, and 26-29 have ADC capabilities
Pin IO modes are: Pin.IN, Pin.OUT, Pin.ALT
Pin pull modes are: Pin.PULL\_UP, Pin.PULL\_DOWN

Useful control commands:
  CTRL-C -- interrupt a running program
  CTRL-D -- on a blank line, do a soft reset of the board
  CTRL-E -- on a blank line, enter paste mode

For further help on a specific object, type help(obj)
For a list of available modules, type help('modules')
\_\_main\_\_          gc                uasyncio/funcs    uos
\_boot             machine           uasyncio/lock     urandom
\_onewire          math              uasyncio/stream   ure
\_rp2              micropython       ubinascii         uselect
\_thread           onewire           ucollections      ustruct
\_uasyncio         rp2               uctypes           usys
builtins          uasyncio/\_\_init\_\_ uhashlib          utime
ds18x20           uasyncio/core     uio               uzlib
framebuf          uasyncio/event    ujson
Plus any modules on the filesystem
object <module 'builtins'> is of type module
  \_\_name\_\_ -- builtins
  \_\_build\_class\_\_ -- <function>
  \_\_import\_\_ -- <function>
  \_\_repl\_print\_\_ -- <function>
  bool -- <class 'bool'>
  bytes -- <class 'bytes'>
  bytearray -- <class 'bytearray'>
  complex -- <class 'complex'>
  dict -- <class 'dict'>
  enumerate -- <class 'enumerate'>
  filter -- <class 'filter'>
  float -- <class 'float'>
  int -- <class 'int'>
  list -- <class 'list'>
  map -- <class 'map'>
  memoryview -- <class 'memoryview'>
  object -- <class 'object'>
  property -- <class 'property'>
  range -- <class 'range'>
  reversed -- <class 'reversed'>
  set -- <class 'set'>
  slice -- <class 'slice'>
  str -- <class 'str'>
  super -- <class 'super'>
  tuple -- <class 'tuple'>
  type -- <class 'type'>
  zip -- <class 'zip'>
  classmethod -- <class 'classmethod'>
  staticmethod -- <class 'staticmethod'>
  Ellipsis -- Ellipsis
  abs -- <function>
  all -- <function>
  any -- <function>
  bin -- <function>
  callable -- <function>
  chr -- <function>
  delattr -- <function>
  dir -- <function>
  divmod -- <function>
  eval -- <function>
  exec -- <function>
  getattr -- <function>
  setattr -- <function>
  globals -- <function>
  hasattr -- <function>
  hash -- <function>
  help -- <function>
  hex -- <function>
  id -- <function>
  input -- <function>
  isinstance -- <function>
  issubclass -- <function>
  iter -- <function>
  len -- <function>
  locals -- <function>
  max -- <function>
  min -- <function>
  next -- <function>
  oct -- <function>
  ord -- <function>
  pow -- <function>
  print -- <function>
  repr -- <function>
  round -- <function>
  sorted -- <function>
  sum -- <function>
  BaseException -- <class 'BaseException'>
  ArithmeticError -- <class 'ArithmeticError'>
  AssertionError -- <class 'AssertionError'>
  AttributeError -- <class 'AttributeError'>
  EOFError -- <class 'EOFError'>
  Exception -- <class 'Exception'>
  GeneratorExit -- <class 'GeneratorExit'>
  ImportError -- <class 'ImportError'>
  IndentationError -- <class 'IndentationError'>
  IndexError -- <class 'IndexError'>
  KeyboardInterrupt -- <class 'KeyboardInterrupt'>
  KeyError -- <class 'KeyError'>
  LookupError -- <class 'LookupError'>
  MemoryError -- <class 'MemoryError'>
  NameError -- <class 'NameError'>
  NotImplementedError -- <class 'NotImplementedError'>
  OSError -- <class 'OSError'>
  OverflowError -- <class 'OverflowError'>
  RuntimeError -- <class 'RuntimeError'>
  StopAsyncIteration -- <class 'StopAsyncIteration'>
  StopIteration -- <class 'StopIteration'>
  SyntaxError -- <class 'SyntaxError'>
  SystemExit -- <class 'SystemExit'>
  TypeError -- <class 'TypeError'>
  UnicodeError -- <class 'UnicodeError'>
  ValueError -- <class 'ValueError'>
  ViperTypeError -- <class 'ViperTypeError'>
  ZeroDivisionError -- <class 'ZeroDivisionError'>
  open -- <function>
    \end{Verbatim}

    \hypertarget{datentypen}{%
\section{Datentypen}\label{datentypen}}

Datentypen: * \textbf{Numerisch:} int, float, complex, bool, bytes,
bytearray, memoryview * \textbf{Text:} str

\begin{itemize}
\tightlist
\item
  \textbf{Listen:} list, tuple, set, dict, frozenset, range, enumerate,
\item
  \textbf{Sonstiges:} function, class
\end{itemize}

    \hypertarget{variablen}{%
\subsection{Variablen}\label{variablen}}

    \begin{tcolorbox}[breakable, size=fbox, boxrule=1pt, pad at break*=1mm,colback=cellbackground, colframe=cellborder]
\prompt{In}{incolor}{31}{\boxspacing}
\begin{Verbatim}[commandchars=\\\{\}]
\PY{c+c1}{\PYZsh{} Anders als bei z.B. C müssen Variablen nicht deklariert werden}
\PY{n}{x} \PY{o}{=} \PY{l+m+mi}{3}
\PY{n}{y} \PY{o}{=} \PY{l+s+s2}{\PYZdq{}}\PY{l+s+s2}{MicroPython}\PY{l+s+s2}{\PYZdq{}}

\PY{n+nb}{print}\PY{p}{(}\PY{n}{x}\PY{p}{)}
\PY{n+nb}{print}\PY{p}{(}\PY{n}{y}\PY{p}{)}
\end{Verbatim}
\end{tcolorbox}

    \begin{Verbatim}[commandchars=\\\{\}]
3
MicroPython
    \end{Verbatim}

    \begin{tcolorbox}[breakable, size=fbox, boxrule=1pt, pad at break*=1mm,colback=cellbackground, colframe=cellborder]
\prompt{In}{incolor}{30}{\boxspacing}
\begin{Verbatim}[commandchars=\\\{\}]
\PY{c+c1}{\PYZsh{} Variablen im Symbol Table (https://en.wikipedia.org/wiki/Symbol\PYZus{}table)}
\PY{n+nb}{print}\PY{p}{(}\PY{n+nb}{dir}\PY{p}{(}\PY{p}{)}\PY{p}{)}      \PY{c+c1}{\PYZsh{} Liste von Strings}
\PY{n+nb}{print}\PY{p}{(}\PY{n+nb}{locals}\PY{p}{(}\PY{p}{)}\PY{p}{)}   \PY{c+c1}{\PYZsh{} Dictionary (kann auch verändert werden)}
\PY{n+nb}{print}\PY{p}{(}\PY{n+nb}{globals}\PY{p}{(}\PY{p}{)}\PY{p}{)}  \PY{c+c1}{\PYZsh{} Dictionary (kann auch verändert werden)}
\end{Verbatim}
\end{tcolorbox}

    \begin{Verbatim}[commandchars=\\\{\}]
['machine', 'x', 'y', 'builtins', '\_\_name\_\_', 'rp2']
\{'machine': <module 'umachine'>, 'x': 3, 'y': 'MicroPython', 'builtins': <module
'builtins'>, '\_\_name\_\_': '\_\_main\_\_', 'rp2': <module 'rp2'>\}
\{'machine': <module 'umachine'>, 'x': 3, 'y': 'MicroPython', 'builtins': <module
'builtins'>, '\_\_name\_\_': '\_\_main\_\_', 'rp2': <module 'rp2'>\}
    \end{Verbatim}

    \begin{tcolorbox}[breakable, size=fbox, boxrule=1pt, pad at break*=1mm,colback=cellbackground, colframe=cellborder]
\prompt{In}{incolor}{34}{\boxspacing}
\begin{Verbatim}[commandchars=\\\{\}]
\PY{c+c1}{\PYZsh{} Überschreiben einer Variable}
\PY{n}{y} \PY{o}{=} \PY{n}{x}
\PY{n+nb}{print}\PY{p}{(}\PY{n}{y}\PY{p}{)}
\PY{n+nb}{print}\PY{p}{(}\PY{n}{x}\PY{p}{)}
\end{Verbatim}
\end{tcolorbox}

    \begin{Verbatim}[commandchars=\\\{\}]
3
3
    \end{Verbatim}

    \begin{tcolorbox}[breakable, size=fbox, boxrule=1pt, pad at break*=1mm,colback=cellbackground, colframe=cellborder]
\prompt{In}{incolor}{38}{\boxspacing}
\begin{Verbatim}[commandchars=\\\{\}]
\PY{c+c1}{\PYZsh{} Löschen von Variablen}
\PY{c+c1}{\PYZsh{} Das löschen von Variablen ist nicht umbedingt nötig \PYZhy{}\PYZgt{} Garbage Collector (gc)}
\PY{n}{x} \PY{o}{=} \PY{l+m+mi}{3}
\PY{n}{y} \PY{o}{=} \PY{n}{x}

\PY{k}{del} \PY{n}{x}
\PY{c+c1}{\PYZsh{} print(x) \PYZsh{} Error!}

\PY{n+nb}{print}\PY{p}{(}\PY{n}{y}\PY{p}{)}
\PY{n+nb}{print}\PY{p}{(}\PY{n+nb}{dir}\PY{p}{(}\PY{p}{)}\PY{p}{)}
\end{Verbatim}
\end{tcolorbox}

    \begin{Verbatim}[commandchars=\\\{\}]
3
['machine', 'y', 'builtins', '\_\_name\_\_', 'rp2']
    \end{Verbatim}

    \begin{tcolorbox}[breakable, size=fbox, boxrule=1pt, pad at break*=1mm,colback=cellbackground, colframe=cellborder]
\prompt{In}{incolor}{42}{\boxspacing}
\begin{Verbatim}[commandchars=\\\{\}]
\PY{c+c1}{\PYZsh{} Variablentypen}
\PY{n}{x} \PY{o}{=} \PY{l+m+mi}{3}
\PY{n}{y} \PY{o}{=} \PY{l+s+s2}{\PYZdq{}}\PY{l+s+s2}{MicroPython}\PY{l+s+s2}{\PYZdq{}}
\PY{n+nb}{print}\PY{p}{(}\PY{n}{x}\PY{p}{)}
\PY{n+nb}{print}\PY{p}{(}\PY{n+nb}{type}\PY{p}{(}\PY{n}{x}\PY{p}{)}\PY{p}{)}
\PY{n+nb}{print}\PY{p}{(}\PY{n}{y}\PY{p}{)}
\PY{n+nb}{print}\PY{p}{(}\PY{n+nb}{type}\PY{p}{(}\PY{n}{y}\PY{p}{)}\PY{p}{)}
\end{Verbatim}
\end{tcolorbox}

    \begin{Verbatim}[commandchars=\\\{\}]
3
<class 'int'>
MicroPython
<class 'str'>
    \end{Verbatim}

    \begin{tcolorbox}[breakable, size=fbox, boxrule=1pt, pad at break*=1mm,colback=cellbackground, colframe=cellborder]
\prompt{In}{incolor}{47}{\boxspacing}
\begin{Verbatim}[commandchars=\\\{\}]
\PY{c+c1}{\PYZsh{} Casting von Variablen }

\PY{c+c1}{\PYZsh{} Variablen kann ein bestimmter Typ bewust zugewiesen werden}
\PY{n}{x} \PY{o}{=} \PY{n+nb}{str}\PY{p}{(}\PY{l+m+mi}{3}\PY{p}{)}    
\PY{n}{y} \PY{o}{=} \PY{n+nb}{int}\PY{p}{(}\PY{l+m+mi}{3}\PY{p}{)}    
\PY{n}{z} \PY{o}{=} \PY{n+nb}{float}\PY{p}{(}\PY{l+m+mi}{3}\PY{p}{)}  

\PY{n+nb}{print}\PY{p}{(}\PY{n}{x} \PY{o}{+} \PY{l+s+s2}{\PYZdq{}}\PY{l+s+s2}{: }\PY{l+s+s2}{\PYZdq{}} \PY{o}{+} \PY{n+nb}{str}\PY{p}{(}\PY{n+nb}{type}\PY{p}{(}\PY{n}{x}\PY{p}{)}\PY{p}{)}\PY{p}{)}       \PY{c+c1}{\PYZsh{}Zusammensetzen eines Strings für eine klarere Ausgabe}
\PY{n+nb}{print}\PY{p}{(}\PY{n+nb}{str}\PY{p}{(}\PY{n}{y}\PY{p}{)} \PY{o}{+} \PY{l+s+s2}{\PYZdq{}}\PY{l+s+s2}{: }\PY{l+s+s2}{\PYZdq{}} \PY{o}{+} \PY{n+nb}{str}\PY{p}{(}\PY{n+nb}{type}\PY{p}{(}\PY{n}{y}\PY{p}{)}\PY{p}{)}\PY{p}{)}  \PY{c+c1}{\PYZsh{}Print kann nur Strings ausgeben, also müssen Variablen eines anderen Types als String gecastet werden}
\PY{n+nb}{print}\PY{p}{(}\PY{n+nb}{str}\PY{p}{(}\PY{n}{z}\PY{p}{)} \PY{o}{+} \PY{l+s+s1}{\PYZsq{}}\PY{l+s+s1}{: }\PY{l+s+s1}{\PYZsq{}} \PY{o}{+} \PY{n+nb}{str}\PY{p}{(}\PY{n+nb}{type}\PY{p}{(}\PY{n}{z}\PY{p}{)}\PY{p}{)}\PY{p}{)}  \PY{c+c1}{\PYZsh{}In Python kann ein String durch ein einfaches oder doppeltes Anführungszeichen gekennzeichnet werden}
\end{Verbatim}
\end{tcolorbox}

    \begin{Verbatim}[commandchars=\\\{\}]
3: <class 'str'>
3: <class 'int'>
3.0: <class 'float'>
    \end{Verbatim}

    \begin{tcolorbox}[breakable, size=fbox, boxrule=1pt, pad at break*=1mm,colback=cellbackground, colframe=cellborder]
\prompt{In}{incolor}{49}{\boxspacing}
\begin{Verbatim}[commandchars=\\\{\}]
\PY{c+c1}{\PYZsh{} Kurzschreibweisen}

\PY{n}{x}\PY{p}{,} \PY{n}{y}\PY{p}{,} \PY{n}{z} \PY{o}{=} \PY{l+s+s2}{\PYZdq{}}\PY{l+s+s2}{Orange}\PY{l+s+s2}{\PYZdq{}}\PY{p}{,} \PY{l+s+s2}{\PYZdq{}}\PY{l+s+s2}{Banane}\PY{l+s+s2}{\PYZdq{}}\PY{p}{,} \PY{l+s+s2}{\PYZdq{}}\PY{l+s+s2}{Kirsche}\PY{l+s+s2}{\PYZdq{}}
\PY{n+nb}{print}\PY{p}{(}\PY{n}{x}\PY{p}{)}
\PY{n+nb}{print}\PY{p}{(}\PY{n}{y}\PY{p}{)}
\PY{n+nb}{print}\PY{p}{(}\PY{n}{z}\PY{p}{)}

\PY{n}{x} \PY{o}{=} \PY{n}{y} \PY{o}{=} \PY{n}{z} \PY{o}{=} \PY{l+s+s2}{\PYZdq{}}\PY{l+s+s2}{Orange}\PY{l+s+s2}{\PYZdq{}}
\PY{n+nb}{print}\PY{p}{(}\PY{n}{x}\PY{p}{)}
\PY{n+nb}{print}\PY{p}{(}\PY{n}{y}\PY{p}{)}
\PY{n+nb}{print}\PY{p}{(}\PY{n}{z}\PY{p}{)}

\PY{k}{del} \PY{n}{x}\PY{p}{,} \PY{n}{y}\PY{p}{,} \PY{n}{z}
\end{Verbatim}
\end{tcolorbox}

    \begin{Verbatim}[commandchars=\\\{\}]
Orange
Banane
Kirsche
Orange
Orange
Orange
    \end{Verbatim}

    \hypertarget{sichtbarkeit}{%
\subsubsection{Sichtbarkeit}\label{sichtbarkeit}}

    \begin{tcolorbox}[breakable, size=fbox, boxrule=1pt, pad at break*=1mm,colback=cellbackground, colframe=cellborder]
\prompt{In}{incolor}{87}{\boxspacing}
\begin{Verbatim}[commandchars=\\\{\}]
\PY{c+c1}{\PYZsh{} Variablen Sichtbarkeit (Scope)}

\PY{n}{x} \PY{o}{=} \PY{l+m+mi}{3}

\PY{k}{def} \PY{n+nf}{myFunc}\PY{p}{(}\PY{p}{)}\PY{p}{:}
    \PY{n}{x} \PY{o}{=} \PY{l+m+mi}{4}
    \PY{n+nb}{print}\PY{p}{(}\PY{l+s+s2}{\PYZdq{}}\PY{l+s+s2}{Globaler Wert: }\PY{l+s+s2}{\PYZdq{}} \PY{o}{+} \PY{n+nb}{str}\PY{p}{(}\PY{n}{x}\PY{p}{)}\PY{p}{)} 
    \PY{n}{y} \PY{o}{=} \PY{l+m+mi}{5}
    \PY{n+nb}{print}\PY{p}{(}\PY{l+s+s2}{\PYZdq{}}\PY{l+s+s2}{Localer Wert: }\PY{l+s+s2}{\PYZdq{}} \PY{o}{+} \PY{n+nb}{str}\PY{p}{(}\PY{n}{y}\PY{p}{)}\PY{p}{)}
    \PY{n+nb}{print}\PY{p}{(}\PY{n+nb}{locals}\PY{p}{(}\PY{p}{)}\PY{p}{)} \PY{c+c1}{\PYZsh{} s.u.}
    
\PY{n}{myFunc}\PY{p}{(}\PY{p}{)}
\PY{n+nb}{print}\PY{p}{(}\PY{n}{x}\PY{p}{)}
\PY{c+c1}{\PYZsh{} print(y) \PYZsh{}! Fehler, y ist nicht bekannt}

\PY{k}{del} \PY{n}{x}

\PY{c+c1}{\PYZsh{} locals:}
\PY{c+c1}{\PYZsh{} https://github.com/micropython/micropython/wiki/Differences}
\PY{c+c1}{\PYZsh{} MicroPython optimizes local variable handling and does not record or provide any introspection info for them, e.g. locals() doesn\PYZsq{}t have entries for locals.}
\end{Verbatim}
\end{tcolorbox}

    \begin{Verbatim}[commandchars=\\\{\}]
Globaler Wert: 4
Localer Wert: 5
\{'x': 3, 'myFunc': <function myFunc at 0x2000c720>, 'myFunction': <function
myFunction at 0x2000a720>, 'machine': <module 'umachine'>, '\_\_name\_\_':
'\_\_main\_\_', 'builtins': <module 'builtins'>, 'rp2': <module 'rp2'>\}
3
    \end{Verbatim}

    \begin{tcolorbox}[breakable, size=fbox, boxrule=1pt, pad at break*=1mm,colback=cellbackground, colframe=cellborder]
\prompt{In}{incolor}{97}{\boxspacing}
\begin{Verbatim}[commandchars=\\\{\}]
\PY{c+c1}{\PYZsh{} Variablen Sichtbarkeit (Scope)}
\PY{n}{x} \PY{o}{=} \PY{l+m+mi}{3}

\PY{k}{def} \PY{n+nf}{myFunc}\PY{p}{(}\PY{p}{)}\PY{p}{:}
    \PY{k}{global} \PY{n}{y}    \PY{c+c1}{\PYZsh{} y existiert nun global}
    \PY{n}{y} \PY{o}{=} \PY{l+m+mi}{5}  
    \PY{k}{global} \PY{n}{x}    \PY{c+c1}{\PYZsh{} die globale Variable x wird verändert}
    \PY{n}{x} \PY{o}{=} \PY{l+m+mi}{4}  
    
\PY{n}{myFunc}\PY{p}{(}\PY{p}{)}
\PY{n+nb}{print}\PY{p}{(}\PY{l+s+s2}{\PYZdq{}}\PY{l+s+s2}{Globaler Wert: }\PY{l+s+s2}{\PYZdq{}} \PY{o}{+} \PY{n+nb}{str}\PY{p}{(}\PY{n}{y}\PY{p}{)}\PY{p}{)}
\PY{n+nb}{print}\PY{p}{(}\PY{l+s+s2}{\PYZdq{}}\PY{l+s+s2}{Globaler Wert: }\PY{l+s+s2}{\PYZdq{}} \PY{o}{+} \PY{n+nb}{str}\PY{p}{(}\PY{n}{x}\PY{p}{)}\PY{p}{)}

\PY{k}{del} \PY{n}{x}\PY{p}{,}\PY{n}{y}
\end{Verbatim}
\end{tcolorbox}

    \begin{Verbatim}[commandchars=\\\{\}]
Globaler Wert: 5
Globaler Wert: 4
    \end{Verbatim}

    \begin{tcolorbox}[breakable, size=fbox, boxrule=1pt, pad at break*=1mm,colback=cellbackground, colframe=cellborder]
\prompt{In}{incolor}{6}{\boxspacing}
\begin{Verbatim}[commandchars=\\\{\}]
\PY{c+c1}{\PYZsh{} Funktionen als Variablen}
\PY{k}{def} \PY{n+nf}{myFunc}\PY{p}{(}\PY{p}{)}\PY{p}{:}
    \PY{n+nb}{print}\PY{p}{(}\PY{l+s+s2}{\PYZdq{}}\PY{l+s+s2}{Meine Funktion}\PY{l+s+s2}{\PYZdq{}}\PY{p}{)}

\PY{n}{otherName} \PY{o}{=} \PY{n}{myFunc}
\PY{n+nb}{print}\PY{p}{(}\PY{n+nb}{type}\PY{p}{(}\PY{n}{myFunc}\PY{p}{)}\PY{p}{)}
\PY{n}{otherName}\PY{p}{(}\PY{p}{)}
\PY{n+nb}{print}\PY{p}{(}\PY{n}{callable}\PY{p}{(}\PY{n}{otherName}\PY{p}{)}\PY{p}{)}
\end{Verbatim}
\end{tcolorbox}

    \begin{Verbatim}[commandchars=\\\{\}]
<class 'function'>
Meine Funktion
True
    \end{Verbatim}

    \hypertarget{numerische-typen}{%
\subsection{Numerische Typen}\label{numerische-typen}}

    \begin{tcolorbox}[breakable, size=fbox, boxrule=1pt, pad at break*=1mm,colback=cellbackground, colframe=cellborder]
\prompt{In}{incolor}{150}{\boxspacing}
\begin{Verbatim}[commandchars=\\\{\}]
\PY{c+c1}{\PYZsh{} Boolsche Variable}
\PY{n}{x} \PY{o}{=} \PY{k+kc}{True}
\PY{n+nb}{print}\PY{p}{(}\PY{n+nb}{str}\PY{p}{(}\PY{n}{x}\PY{p}{)} \PY{o}{+} \PY{l+s+s1}{\PYZsq{}}\PY{l+s+se}{\PYZbs{}t}\PY{l+s+se}{\PYZbs{}t}\PY{l+s+s1}{\PYZsq{}} \PY{o}{+} \PY{n+nb}{str}\PY{p}{(}\PY{n+nb}{type}\PY{p}{(}\PY{n}{x}\PY{p}{)}\PY{p}{)} \PY{o}{+} \PY{l+s+s1}{\PYZsq{}}\PY{l+s+se}{\PYZbs{}t}\PY{l+s+se}{\PYZbs{}t}\PY{l+s+s1}{\PYZsq{}} \PY{o}{+} \PY{n+nb}{str}\PY{p}{(}\PY{n}{x} \PY{o}{==} \PY{n}{x}\PY{p}{)}\PY{p}{)}

\PY{c+c1}{\PYZsh{} Komplexe Variable}
\PY{n}{x} \PY{o}{=} \PY{l+m+mi}{2}\PY{n}{j}
\PY{n+nb}{print}\PY{p}{(}\PY{n+nb}{str}\PY{p}{(}\PY{n}{x}\PY{p}{)} \PY{o}{+} \PY{l+s+s1}{\PYZsq{}}\PY{l+s+se}{\PYZbs{}t}\PY{l+s+se}{\PYZbs{}t}\PY{l+s+s1}{\PYZsq{}} \PY{o}{+} \PY{n+nb}{str}\PY{p}{(}\PY{n+nb}{type}\PY{p}{(}\PY{n}{x}\PY{p}{)}\PY{p}{)} \PY{o}{+} \PY{l+s+s1}{\PYZsq{}}\PY{l+s+se}{\PYZbs{}t}\PY{l+s+s1}{\PYZsq{}} \PY{o}{+} \PY{n+nb}{str}\PY{p}{(}\PY{n}{x}\PY{o}{*}\PY{n}{x}\PY{p}{)}\PY{p}{)}

\PY{c+c1}{\PYZsh{} Bytes}
\PY{n}{x} \PY{o}{=} \PY{n+nb}{bytes}\PY{p}{(}\PY{l+m+mi}{1}\PY{p}{)} \PY{c+c1}{\PYZsh{} Anzahl an Bytes}
\PY{n+nb}{print}\PY{p}{(}\PY{n+nb}{str}\PY{p}{(}\PY{n}{x}\PY{p}{)} \PY{o}{+} \PY{l+s+s1}{\PYZsq{}}\PY{l+s+se}{\PYZbs{}t}\PY{l+s+se}{\PYZbs{}t}\PY{l+s+s1}{\PYZsq{}} \PY{o}{+} \PY{n+nb}{str}\PY{p}{(}\PY{n+nb}{type}\PY{p}{(}\PY{n}{x}\PY{p}{)}\PY{p}{)} \PY{o}{+} \PY{l+s+s1}{\PYZsq{}}\PY{l+s+se}{\PYZbs{}t}\PY{l+s+se}{\PYZbs{}t}\PY{l+s+s1}{\PYZsq{}} \PY{o}{+} \PY{n+nb}{str}\PY{p}{(}\PY{n}{x}\PY{p}{[}\PY{l+m+mi}{0}\PY{p}{]}\PY{p}{)}\PY{p}{)}
\PY{n}{x} \PY{o}{=} \PY{l+s+sa}{b}\PY{l+s+s2}{\PYZdq{}}\PY{l+s+s2}{HALLO}\PY{l+s+s2}{\PYZdq{}}  \PY{c+c1}{\PYZsh{} String zu Byte }
\PY{n+nb}{print}\PY{p}{(}\PY{n+nb}{str}\PY{p}{(}\PY{n}{x}\PY{p}{)} \PY{o}{+} \PY{l+s+s1}{\PYZsq{}}\PY{l+s+se}{\PYZbs{}t}\PY{l+s+s1}{\PYZsq{}} \PY{o}{+} \PY{n+nb}{str}\PY{p}{(}\PY{n+nb}{type}\PY{p}{(}\PY{n}{x}\PY{p}{)}\PY{p}{)} \PY{o}{+} \PY{l+s+s1}{\PYZsq{}}\PY{l+s+se}{\PYZbs{}t}\PY{l+s+se}{\PYZbs{}t}\PY{l+s+s1}{x[1]: }\PY{l+s+s1}{\PYZsq{}} \PY{o}{+} \PY{n+nb}{str}\PY{p}{(}\PY{n}{x}\PY{p}{[}\PY{l+m+mi}{1}\PY{p}{]}\PY{p}{)}\PY{p}{)} \PY{c+c1}{\PYZsh{} (ASCII\PYZhy{}Tabelle https://www.pctipp.ch/praxis/software/ascii\PYZhy{}tabelle\PYZhy{}sonderzeichen\PYZhy{}1859563.html )}
\PY{n+nb}{print}\PY{p}{(}\PY{l+s+s1}{\PYZsq{}}\PY{l+s+se}{\PYZbs{}t}\PY{l+s+se}{\PYZbs{}t}\PY{l+s+se}{\PYZbs{}t}\PY{l+s+se}{\PYZbs{}t}\PY{l+s+se}{\PYZbs{}t}\PY{l+s+s1}{x[1] as hex: }\PY{l+s+si}{\PYZob{}:02x\PYZcb{}}\PY{l+s+s1}{ }\PY{l+s+se}{\PYZbs{}t}\PY{l+s+se}{\PYZbs{}t}\PY{l+s+s1}{ x[1] as bin: }\PY{l+s+s1}{\PYZdq{}}\PY{l+s+si}{\PYZob{}:08b\PYZcb{}}\PY{l+s+s1}{\PYZsq{}}\PY{o}{.}\PY{n}{format}\PY{p}{(}\PY{n}{x}\PY{p}{[}\PY{l+m+mi}{1}\PY{p}{]}\PY{p}{,}\PY{n}{x}\PY{p}{[}\PY{l+m+mi}{1}\PY{p}{]}\PY{p}{)}\PY{p}{)}

\PY{c+c1}{\PYZsh{} Memoryview}
\PY{n}{y} \PY{o}{=} \PY{l+s+sa}{b}\PY{l+s+s2}{\PYZdq{}}\PY{l+s+s2}{HALLO}\PY{l+s+s2}{\PYZdq{}}
\PY{n}{x} \PY{o}{=} \PY{n+nb}{memoryview}\PY{p}{(}\PY{n}{y}\PY{p}{)}
\PY{n+nb}{print}\PY{p}{(}\PY{n+nb}{str}\PY{p}{(}\PY{n}{x}\PY{p}{)} \PY{o}{+} \PY{l+s+s1}{\PYZsq{}}\PY{l+s+se}{\PYZbs{}t}\PY{l+s+s1}{\PYZsq{}} \PY{o}{+} \PY{n+nb}{str}\PY{p}{(}\PY{n+nb}{type}\PY{p}{(}\PY{n}{x}\PY{p}{)}\PY{p}{)} \PY{o}{+} \PY{l+s+s1}{\PYZsq{}}\PY{l+s+se}{\PYZbs{}t}\PY{l+s+s1}{\PYZsq{}} \PY{o}{+} \PY{n+nb}{str}\PY{p}{(}\PY{n}{x}\PY{p}{[}\PY{l+m+mi}{1}\PY{p}{]}\PY{p}{)}\PY{p}{)}

\PY{c+c1}{\PYZsh{} Bytearray}
\PY{n}{x} \PY{o}{=} \PY{n+nb}{bytearray}\PY{p}{(}\PY{l+m+mi}{5}\PY{p}{)}
\PY{n+nb}{print}\PY{p}{(}\PY{n+nb}{str}\PY{p}{(}\PY{n}{x}\PY{p}{)} \PY{o}{+} \PY{l+s+s1}{\PYZsq{}}\PY{l+s+se}{\PYZbs{}t}\PY{l+s+s1}{\PYZsq{}} \PY{o}{+} \PY{n+nb}{str}\PY{p}{(}\PY{n+nb}{type}\PY{p}{(}\PY{n}{x}\PY{p}{)}\PY{p}{)}\PY{p}{)}
\end{Verbatim}
\end{tcolorbox}

    \hypertarget{mathematische-funktionen}{%
\subsubsection{Mathematische
Funktionen}\label{mathematische-funktionen}}

    \begin{tcolorbox}[breakable, size=fbox, boxrule=1pt, pad at break*=1mm,colback=cellbackground, colframe=cellborder]
\prompt{In}{incolor}{40}{\boxspacing}
\begin{Verbatim}[commandchars=\\\{\}]
\PY{c+c1}{\PYZsh{} Mathematische Funktionen}
\PY{n}{x} \PY{o}{=} \PY{l+m+mi}{3}\PY{p}{;} \PY{n}{y} \PY{o}{=} \PY{o}{\PYZhy{}}\PY{l+m+mi}{5}\PY{p}{;} \PY{n}{z} \PY{o}{=} \PY{l+m+mf}{2.4}\PY{p}{;}

\PY{n+nb}{print}\PY{p}{(}\PY{n+nb}{abs}\PY{p}{(}\PY{n}{x}\PY{o}{*}\PY{n}{y}\PY{p}{)}\PY{p}{)}  \PY{c+c1}{\PYZsh{} Absolutwert (Betrag)}
\PY{n+nb}{print}\PY{p}{(}\PY{n+nb}{min}\PY{p}{(}\PY{n}{x}\PY{p}{,}\PY{n}{y}\PY{p}{)}\PY{p}{)}  \PY{c+c1}{\PYZsh{} Minimum}
\PY{n+nb}{print}\PY{p}{(}\PY{n+nb}{max}\PY{p}{(}\PY{n}{x}\PY{p}{,}\PY{n}{y}\PY{p}{)}\PY{p}{)}  \PY{c+c1}{\PYZsh{} Maximum}

\PY{c+c1}{\PYZsh{} Runden von Werten}
\PY{n+nb}{print}\PY{p}{(}\PY{n}{z}\PY{o}{*}\PY{n}{x}\PY{p}{)}     
\PY{n+nb}{print}\PY{p}{(}\PY{n+nb}{round}\PY{p}{(}\PY{n}{z}\PY{o}{*}\PY{n}{x}\PY{p}{)}\PY{p}{)} 
\PY{n+nb}{print}\PY{p}{(}\PY{n+nb}{round}\PY{p}{(}\PY{l+m+mf}{2.7456}\PY{p}{,}\PY{l+m+mi}{2}\PY{p}{)}\PY{p}{)}

\PY{c+c1}{\PYZsh{} Potenzen}
\PY{n+nb}{print}\PY{p}{(}\PY{n+nb}{pow}\PY{p}{(}\PY{n}{y}\PY{p}{,}\PY{n}{x}\PY{p}{)}\PY{p}{)}
\PY{n+nb}{print}\PY{p}{(}\PY{n}{y}\PY{o}{*}\PY{o}{*}\PY{n}{x}\PY{p}{)}
\PY{c+c1}{\PYZsh{} Schnellzuweisungen  (gilt auch für alle anderen Rechenoperationen)}
\PY{n}{y} \PY{o}{*}\PY{o}{*}\PY{o}{=} \PY{n}{x}
\PY{n+nb}{print}\PY{p}{(}\PY{n}{y}\PY{p}{)}
\end{Verbatim}
\end{tcolorbox}

    \begin{Verbatim}[commandchars=\\\{\}]
15
-5
3
7.2
7
2.75
-125
-125
-125
    \end{Verbatim}

    \begin{tcolorbox}[breakable, size=fbox, boxrule=1pt, pad at break*=1mm,colback=cellbackground, colframe=cellborder]
\prompt{In}{incolor}{61}{\boxspacing}
\begin{Verbatim}[commandchars=\\\{\}]
\PY{c+c1}{\PYZsh{} Binäre Operationen}
\PY{n}{a} \PY{o}{=} \PY{l+m+mb}{0b00110011}
\PY{n}{b} \PY{o}{=} \PY{l+m+mb}{0b11000011}

\PY{n+nb}{print}\PY{p}{(}\PY{n+nb}{bin}\PY{p}{(}\PY{l+m+mi}{195}\PY{p}{)}\PY{p}{)}\PY{p}{;}  \PY{c+c1}{\PYZsh{} bin()\PYZhy{}erzeugt einen String der Zahl binär darstellt}
\PY{n+nb}{print}\PY{p}{(}\PY{n+nb}{bin}\PY{p}{(}\PY{n}{b}\PY{p}{)}\PY{p}{)}
\PY{n+nb}{print}\PY{p}{(}\PY{n+nb}{bin}\PY{p}{(}\PY{o}{\PYZhy{}}\PY{l+m+mi}{196}\PY{p}{)}\PY{p}{)}
\PY{n+nb}{print}\PY{p}{(}\PY{n+nb}{bin}\PY{p}{(}\PY{n}{a}\PY{p}{)}\PY{p}{)}

\PY{c+c1}{\PYZsh{} Und, oder, exclusiv oder}
\PY{n+nb}{print}\PY{p}{(}\PY{n+nb}{bin}\PY{p}{(}\PY{n}{a}\PY{o}{\PYZam{}}\PY{n}{b}\PY{p}{)}\PY{p}{)}   
\PY{n+nb}{print}\PY{p}{(}\PY{n+nb}{bin}\PY{p}{(}\PY{n}{a}\PY{o}{|}\PY{n}{b}\PY{p}{)}\PY{p}{)}
\PY{n+nb}{print}\PY{p}{(}\PY{n+nb}{bin}\PY{p}{(}\PY{n}{a}\PY{o}{\PYZca{}}\PY{n}{b}\PY{p}{)}\PY{p}{)}

\PY{n+nb}{print}\PY{p}{(}\PY{n+nb}{bin}\PY{p}{(}\PY{o}{\PYZti{}}\PY{n}{b}\PY{p}{)}\PY{p}{)}    \PY{c+c1}{\PYZsh{} ! Achtung, weil zahl mit vorzeichen interpretiert wird!}
\PY{n+nb}{print}\PY{p}{(}\PY{n+nb}{bin}\PY{p}{(}\PY{l+m+mi}{255}\PY{o}{\PYZhy{}}\PY{n}{b}\PY{p}{)}\PY{p}{)}

\PY{n+nb}{print}\PY{p}{(}\PY{n+nb}{bin}\PY{p}{(}\PY{n}{b}\PY{o}{\PYZgt{}\PYZgt{}}\PY{l+m+mi}{2}\PY{p}{)}\PY{p}{)}  \PY{c+c1}{\PYZsh{} Shift um zwei Stellen nach rechts}
\PY{n+nb}{print}\PY{p}{(}\PY{n+nb}{bin}\PY{p}{(}\PY{n}{a}\PY{o}{\PYZlt{}\PYZlt{}}\PY{l+m+mi}{2}\PY{p}{)}\PY{p}{)}  \PY{c+c1}{\PYZsh{} Shift um zwei Stellen nach links}
\end{Verbatim}
\end{tcolorbox}

    \begin{Verbatim}[commandchars=\\\{\}]
0b11000011
0b11000011
-0b11000100
0b110011
0b11
0b11110011
0b11110000
-0b11000100
0b111100
0b110000
0b11001100
    \end{Verbatim}

    \hypertarget{strings}{%
\subsection{Strings}\label{strings}}

    \begin{tcolorbox}[breakable, size=fbox, boxrule=1pt, pad at break*=1mm,colback=cellbackground, colframe=cellborder]
\prompt{In}{incolor}{52}{\boxspacing}
\begin{Verbatim}[commandchars=\\\{\}]
\PY{c+c1}{\PYZsh{} Strings einer Variable zuweisen}

\PY{n}{a} \PY{o}{=} \PY{l+s+s2}{\PYZdq{}}\PY{l+s+s2}{Text A}\PY{l+s+s2}{\PYZdq{}}
\PY{n}{b} \PY{o}{=} \PY{l+s+s1}{\PYZsq{}}\PY{l+s+s1}{Text B}\PY{l+s+s1}{\PYZsq{}}    \PY{c+c1}{\PYZsh{} Mit Sonderzeilchen \PYZdq{}Neue Zeile\PYZdq{}}
\PY{n}{c} \PY{o}{=} \PY{l+s+s2}{\PYZdq{}\PYZdq{}\PYZdq{}}\PY{l+s+s2}{Mehrzeiliger Text }
\PY{l+s+s2}{Zeile 1 }
\PY{l+s+s2}{Zeile 2}\PY{l+s+s2}{\PYZdq{}\PYZdq{}\PYZdq{}} \PY{c+c1}{\PYZsh{} Mit Sonderzeilchen \PYZdq{}Tabulator\PYZdq{}}
\PY{n}{d} \PY{o}{=} \PY{l+s+s1}{\PYZsq{}\PYZsq{}\PYZsq{}}\PY{l+s+s1}{  }
\PY{l+s+se}{\PYZbs{}t}\PY{l+s+s1}{Auch ein Mehrzeiliger Text}
\PY{l+s+s1}{Zeile 1}
\PY{l+s+s1}{Zeile 2}

\PY{l+s+s1}{\PYZsq{}\PYZsq{}\PYZsq{}}

\PY{n+nb}{print}\PY{p}{(}\PY{n}{a}\PY{p}{)}\PY{p}{;} \PY{n+nb}{print}\PY{p}{(}\PY{n}{b}\PY{p}{)}\PY{p}{;} \PY{n+nb}{print}\PY{p}{(}\PY{n}{c}\PY{p}{)}\PY{p}{;} \PY{n+nb}{print}\PY{p}{(}\PY{n}{d}\PY{p}{)}
\end{Verbatim}
\end{tcolorbox}

    \begin{Verbatim}[commandchars=\\\{\}]
Text A
Text B
Mehrzeiliger Text
Zeile 1
Zeile 2

        Auch ein Mehrzeiliger Text
Zeile 1
Zeile 2


    \end{Verbatim}

    \begin{tcolorbox}[breakable, size=fbox, boxrule=1pt, pad at break*=1mm,colback=cellbackground, colframe=cellborder]
\prompt{In}{incolor}{24}{\boxspacing}
\begin{Verbatim}[commandchars=\\\{\}]
\PY{c+c1}{\PYZsh{} Strings sind nicht anderes als eine Liste (vgl. Array) von Zeichen}

\PY{n+nb}{print}\PY{p}{(}\PY{n}{a}\PY{p}{[}\PY{l+m+mi}{5}\PY{p}{]}\PY{p}{)}
\PY{c+c1}{\PYZsh{}print(a[6]) \PYZsh{}! Fehler: IndexError: string index out of range}

\PY{n+nb}{print}\PY{p}{(}\PY{n+nb}{len}\PY{p}{(}\PY{n}{a}\PY{p}{)}\PY{p}{)}  \PY{c+c1}{\PYZsh{}Länge des Strings, index beginnt mit 0}
\end{Verbatim}
\end{tcolorbox}

    \begin{Verbatim}[commandchars=\\\{\}]
A
6
    \end{Verbatim}

    \begin{tcolorbox}[breakable, size=fbox, boxrule=1pt, pad at break*=1mm,colback=cellbackground, colframe=cellborder]
\prompt{In}{incolor}{25}{\boxspacing}
\begin{Verbatim}[commandchars=\\\{\}]
\PY{c+c1}{\PYZsh{} Prüfen ob ein Teilstring enthalten ist}

\PY{n+nb}{print}\PY{p}{(}\PY{l+s+s2}{\PYZdq{}}\PY{l+s+s2}{Text}\PY{l+s+s2}{\PYZdq{}} \PY{o+ow}{in} \PY{n}{c}\PY{p}{)}
\PY{n+nb}{print}\PY{p}{(}\PY{l+s+s2}{\PYZdq{}}\PY{l+s+s2}{Text}\PY{l+s+s2}{\PYZdq{}} \PY{o+ow}{not} \PY{o+ow}{in} \PY{n}{c}\PY{p}{)}
\end{Verbatim}
\end{tcolorbox}

    \begin{Verbatim}[commandchars=\\\{\}]
True
False
    \end{Verbatim}

    \begin{tcolorbox}[breakable, size=fbox, boxrule=1pt, pad at break*=1mm,colback=cellbackground, colframe=cellborder]
\prompt{In}{incolor}{27}{\boxspacing}
\begin{Verbatim}[commandchars=\\\{\}]
\PY{c+c1}{\PYZsh{} Teilstrings (vgl. Listen)}

\PY{n+nb}{print}\PY{p}{(}\PY{n}{a}\PY{p}{[}\PY{l+m+mi}{0}\PY{p}{:}\PY{l+m+mi}{4}\PY{p}{]}\PY{p}{)}   \PY{c+c1}{\PYZsh{} Teilstring von Index 0 bis 4}
\PY{n+nb}{print}\PY{p}{(}\PY{n}{a}\PY{p}{[}\PY{p}{:}\PY{l+m+mi}{4}\PY{p}{]}\PY{p}{)}    \PY{c+c1}{\PYZsh{} Teilstring von Index 0 (automatisch weil nicht angegeben) bis 4}
\PY{n+nb}{print}\PY{p}{(}\PY{n}{a}\PY{p}{[}\PY{l+m+mi}{5}\PY{p}{:}\PY{p}{]}\PY{p}{)}    \PY{c+c1}{\PYZsh{} Teilstring von Index 5 bis Ende (automatisch weil nicht angegeben)}
\PY{n+nb}{print}\PY{p}{(}\PY{n}{a}\PY{p}{[}\PY{o}{\PYZhy{}}\PY{l+m+mi}{1}\PY{p}{:}\PY{p}{]}\PY{p}{)}   \PY{c+c1}{\PYZsh{} Teilstring mit negativen Indizes (Von \PYZhy{}1 bis Ende des Strings)}
\end{Verbatim}
\end{tcolorbox}

    \begin{Verbatim}[commandchars=\\\{\}]
Text
Text
A
A
    \end{Verbatim}

    \begin{tcolorbox}[breakable, size=fbox, boxrule=1pt, pad at break*=1mm,colback=cellbackground, colframe=cellborder]
\prompt{In}{incolor}{15}{\boxspacing}
\begin{Verbatim}[commandchars=\\\{\}]
\PY{c+c1}{\PYZsh{} String Funktionen}

\PY{n}{help}\PY{p}{(}\PY{n+nb}{str}\PY{p}{)}
\end{Verbatim}
\end{tcolorbox}

    \begin{Verbatim}[commandchars=\\\{\}]
object <class 'str'> is of type type
  encode -- <function>
  find -- <function>
  rfind -- <function>
  index -- <function>
  rindex -- <function>
  join -- <function>
  split -- <function>
  rsplit -- <function>
  startswith -- <function>
  endswith -- <function>
  strip -- <function>
  lstrip -- <function>
  rstrip -- <function>
  format -- <function>
  replace -- <function>
  count -- <function>
  lower -- <function>
  upper -- <function>
  isspace -- <function>
  isalpha -- <function>
  isdigit -- <function>
  isupper -- <function>
  islower -- <function>
    \end{Verbatim}

    \begin{tcolorbox}[breakable, size=fbox, boxrule=1pt, pad at break*=1mm,colback=cellbackground, colframe=cellborder]
\prompt{In}{incolor}{78}{\boxspacing}
\begin{Verbatim}[commandchars=\\\{\}]
\PY{c+c1}{\PYZsh{} String Funktionen \PYZsh{}1 }

\PY{n+nb}{print}\PY{p}{(}\PY{n}{a}\PY{o}{.}\PY{n}{upper}\PY{p}{(}\PY{p}{)}\PY{p}{)}
\PY{n+nb}{print}\PY{p}{(}\PY{n}{b}\PY{o}{.}\PY{n}{lower}\PY{p}{(}\PY{p}{)}\PY{p}{)}

\PY{n+nb}{print}\PY{p}{(}\PY{n}{a}\PY{o}{.}\PY{n}{lower}\PY{p}{(}\PY{p}{)}\PY{o}{.}\PY{n}{islower}\PY{p}{(}\PY{p}{)}\PY{p}{)}
\PY{n+nb}{print}\PY{p}{(}\PY{n}{b}\PY{o}{.}\PY{n}{upper}\PY{p}{(}\PY{p}{)}\PY{o}{.}\PY{n}{isupper}\PY{p}{(}\PY{p}{)}\PY{p}{)}

\PY{n+nb}{print}\PY{p}{(}\PY{n}{a}\PY{o}{.}\PY{n}{startswith}\PY{p}{(}\PY{l+s+s2}{\PYZdq{}}\PY{l+s+s2}{Text}\PY{l+s+s2}{\PYZdq{}}\PY{p}{)}\PY{p}{)}
\PY{n+nb}{print}\PY{p}{(}\PY{n}{b}\PY{o}{.}\PY{n}{endswith}\PY{p}{(}\PY{l+s+s2}{\PYZdq{}}\PY{l+s+s2}{B}\PY{l+s+s2}{\PYZdq{}}\PY{p}{)}\PY{p}{)}

\PY{n}{e} \PY{o}{=} \PY{n}{c}\PY{p}{[}\PY{p}{:}\PY{l+m+mi}{12}\PY{p}{]}\PY{p}{;} \PY{n}{f} \PY{o}{=} \PY{n}{d}\PY{p}{[}\PY{p}{:}\PY{l+m+mi}{2}\PY{p}{]}\PY{p}{;} \PY{n}{g}\PY{o}{=} \PY{n}{c}\PY{p}{[}\PY{o}{\PYZhy{}}\PY{l+m+mi}{1}\PY{p}{:}\PY{p}{]}
\PY{n+nb}{print}\PY{p}{(}\PY{n}{e}\PY{o}{+}\PY{n}{f}\PY{o}{+}\PY{n}{g}\PY{p}{)}      \PY{c+c1}{\PYZsh{} Einfache Zusammensetzung (Konkatenation, verkettung) von Strings}

\PY{n+nb}{print}\PY{p}{(}\PY{n}{e}\PY{o}{.}\PY{n}{isalpha}\PY{p}{(}\PY{p}{)}\PY{p}{)}   \PY{c+c1}{\PYZsh{} Besteht nur aus Zeichen des Alphabets}
\PY{n+nb}{print}\PY{p}{(}\PY{n}{f}\PY{o}{.}\PY{n}{isspace}\PY{p}{(}\PY{p}{)}\PY{p}{)}   \PY{c+c1}{\PYZsh{} ist ein oder mehr Leerzeichen}
\PY{n+nb}{print}\PY{p}{(}\PY{n}{g}\PY{o}{.}\PY{n}{isdigit}\PY{p}{(}\PY{p}{)}\PY{p}{)}   \PY{c+c1}{\PYZsh{} ist eine Zahl}
\end{Verbatim}
\end{tcolorbox}

    \begin{Verbatim}[commandchars=\\\{\}]
TEXT A
text b
True
True
8
True
True
Mehrzeiliger  2
True
True
True
    \end{Verbatim}

    \begin{tcolorbox}[breakable, size=fbox, boxrule=1pt, pad at break*=1mm,colback=cellbackground, colframe=cellborder]
\prompt{In}{incolor}{85}{\boxspacing}
\begin{Verbatim}[commandchars=\\\{\}]
\PY{c+c1}{\PYZsh{} String Funktionen \PYZsh{}2}
\PY{n+nb}{print}\PY{p}{(}\PY{n}{c}\PY{p}{)}
\PY{n+nb}{print}\PY{p}{(}\PY{n}{c}\PY{o}{.}\PY{n}{count}\PY{p}{(}\PY{l+s+s2}{\PYZdq{}}\PY{l+s+s2}{e}\PY{l+s+s2}{\PYZdq{}}\PY{p}{)}\PY{p}{)}      \PY{c+c1}{\PYZsh{} Zähle wieviele von Teilstrings existieren}

\PY{c+c1}{\PYZsh{} string.find(value, start, end)}
\PY{n+nb}{print}\PY{p}{(}\PY{n}{c}\PY{o}{.}\PY{n}{find}\PY{p}{(}\PY{l+s+s2}{\PYZdq{}}\PY{l+s+s2}{Zeile}\PY{l+s+s2}{\PYZdq{}}\PY{p}{)}\PY{p}{)}   \PY{c+c1}{\PYZsh{} Gib die erste Position von Teilstring aus}
\PY{n+nb}{print}\PY{p}{(}\PY{n}{c}\PY{o}{.}\PY{n}{rfind}\PY{p}{(}\PY{l+s+s2}{\PYZdq{}}\PY{l+s+s2}{Zeile}\PY{l+s+s2}{\PYZdq{}}\PY{p}{)}\PY{p}{)}  \PY{c+c1}{\PYZsh{} Gib die erste Position von Teilstring aus (von beginnend mit dem hinteren Teil des Strings)}
\PY{n+nb}{print}\PY{p}{(}\PY{n}{c}\PY{o}{.}\PY{n}{find}\PY{p}{(}\PY{l+s+s2}{\PYZdq{}}\PY{l+s+s2}{Gibts nicht}\PY{l+s+s2}{\PYZdq{}}\PY{p}{)}\PY{p}{)}
\PY{c+c1}{\PYZsh{} string.index(value, start, end)}
\PY{n+nb}{print}\PY{p}{(}\PY{n}{c}\PY{o}{.}\PY{n}{index}\PY{p}{(}\PY{l+s+s2}{\PYZdq{}}\PY{l+s+s2}{Zeile}\PY{l+s+s2}{\PYZdq{}}\PY{p}{)}\PY{p}{)}   \PY{c+c1}{\PYZsh{} Gib die erste Position von Teilstring aus}
\PY{n+nb}{print}\PY{p}{(}\PY{n}{c}\PY{o}{.}\PY{n}{rindex}\PY{p}{(}\PY{l+s+s2}{\PYZdq{}}\PY{l+s+s2}{Zeile}\PY{l+s+s2}{\PYZdq{}}\PY{p}{)}\PY{p}{)}   \PY{c+c1}{\PYZsh{} Gib die erste Position von Teilstring aus}
\PY{n+nb}{print}\PY{p}{(}\PY{n}{c}\PY{o}{.}\PY{n}{index}\PY{p}{(}\PY{l+s+s2}{\PYZdq{}}\PY{l+s+s2}{Gibts nicht}\PY{l+s+s2}{\PYZdq{}}\PY{p}{)}\PY{p}{)}  \PY{c+c1}{\PYZsh{} Hier ist der einzige Unterschied: ValueError statt \PYZhy{}1}
\end{Verbatim}
\end{tcolorbox}

    \begin{Verbatim}[commandchars=\\\{\}]
Mehrzeiliger Text
Zeile 1
Zeile 2
8
19
28
-1
19
28
    \end{Verbatim}

    \begin{Verbatim}[commandchars=\\\{\}]
Traceback (most recent call last):
  File "<stdin>", line 12, in <module>
ValueError: substring not found
    \end{Verbatim}

    \begin{tcolorbox}[breakable, size=fbox, boxrule=1pt, pad at break*=1mm,colback=cellbackground, colframe=cellborder]
\prompt{In}{incolor}{73}{\boxspacing}
\begin{Verbatim}[commandchars=\\\{\}]
\PY{c+c1}{\PYZsh{} String Funktionen \PYZsh{}3}
\PY{n+nb}{print}\PY{p}{(}\PY{n}{d}\PY{p}{)}

\PY{c+c1}{\PYZsh{} string.split(separator, maxsplit)}
\PY{n+nb}{print}\PY{p}{(}\PY{n}{d}\PY{o}{.}\PY{n}{strip}\PY{p}{(}\PY{p}{)}\PY{p}{)}        \PY{c+c1}{\PYZsh{} Leerzeichen am Anfang und am Ende des Strings werden entfernt}
\PY{n+nb}{print}\PY{p}{(}\PY{n}{d}\PY{o}{.}\PY{n}{lstrip}\PY{p}{(}\PY{p}{)}\PY{p}{)}       \PY{c+c1}{\PYZsh{} Leerzeichen am Anfang des Strings werden entfernt}
\PY{n+nb}{print}\PY{p}{(}\PY{n}{d}\PY{o}{.}\PY{n}{rstrip}\PY{p}{(}\PY{p}{)}\PY{p}{)}       \PY{c+c1}{\PYZsh{} Leerzeichen am Ende des Strings werden entfernt}
\end{Verbatim}
\end{tcolorbox}

    \begin{Verbatim}[commandchars=\\\{\}]

        Auch ein Mehrzeiliger Text
Zeile 1
Zeile 2


Auch ein Mehrzeiliger Text
Zeile 1
Zeile 2
Auch ein Mehrzeiliger Text
Zeile 1
Zeile 2



        Auch ein Mehrzeiliger Text
Zeile 1
Zeile 2
    \end{Verbatim}

    \begin{tcolorbox}[breakable, size=fbox, boxrule=1pt, pad at break*=1mm,colback=cellbackground, colframe=cellborder]
\prompt{In}{incolor}{103}{\boxspacing}
\begin{Verbatim}[commandchars=\\\{\}]
\PY{c+c1}{\PYZsh{} String Funktionen \PYZsh{}4}

\PY{c+c1}{\PYZsh{} string.replace(oldvalue, newvalue, count)}
\PY{n+nb}{print}\PY{p}{(}\PY{n}{a}\PY{o}{.}\PY{n}{replace}\PY{p}{(}\PY{l+s+s2}{\PYZdq{}}\PY{l+s+s2}{Text}\PY{l+s+s2}{\PYZdq{}}\PY{p}{,}\PY{l+s+s2}{\PYZdq{}}\PY{l+s+s2}{String}\PY{l+s+s2}{\PYZdq{}}\PY{p}{)}\PY{p}{)}   

\PY{c+c1}{\PYZsh{} string.split(separator, maxsplit)}
\PY{n}{list\PYZus{}of\PYZus{}strings} \PY{o}{=} \PY{n}{c}\PY{o}{.}\PY{n}{split}\PY{p}{(}\PY{l+s+s2}{\PYZdq{}}\PY{l+s+se}{\PYZbs{}n}\PY{l+s+s2}{\PYZdq{}}\PY{p}{)}     \PY{c+c1}{\PYZsh{} String wird aufgeteilt, jeweils am Trennungsstrring (hier neue Zeile)}
\PY{n}{list\PYZus{}of\PYZus{}strings\PYZus{}right} \PY{o}{=} \PY{n}{c}\PY{o}{.}\PY{n}{rsplit}\PY{p}{(}\PY{l+s+s2}{\PYZdq{}}\PY{l+s+se}{\PYZbs{}n}\PY{l+s+s2}{\PYZdq{}}\PY{p}{,}\PY{l+m+mi}{1}\PY{p}{)}     \PY{c+c1}{\PYZsh{} String wird aufgeteilt, jeweils am Trennungsstrring (hier neue Zeile), maximal sofoft wie angegeben, rechts startend}
\PY{n+nb}{print}\PY{p}{(}\PY{n}{list\PYZus{}of\PYZus{}strings}\PY{p}{)}
\PY{n+nb}{print}\PY{p}{(}\PY{n}{list\PYZus{}of\PYZus{}strings\PYZus{}right}\PY{p}{)}

\PY{n+nb}{print}\PY{p}{(}\PY{l+s+s2}{\PYZdq{}}\PY{l+s+se}{\PYZbs{}n}\PY{l+s+s2}{\PYZdq{}}\PY{o}{.}\PY{n}{join}\PY{p}{(}\PY{n}{list\PYZus{}of\PYZus{}strings}\PY{p}{)}\PY{p}{)}   \PY{c+c1}{\PYZsh{} inverses von split, liste wird aneinandergereit mit jeweiligem Trennzeichen}

\PY{c+c1}{\PYZsh{} ! Hier: Convertierung zwischen Strings und Byte\PYZhy{}Array}
\PY{n+nb}{print}\PY{p}{(}\PY{n}{a}\PY{o}{.}\PY{n}{encode}\PY{p}{(}\PY{p}{)}\PY{p}{)}
\PY{n+nb}{print}\PY{p}{(}\PY{n}{a}\PY{o}{.}\PY{n}{encode}\PY{p}{(}\PY{p}{)}\PY{o}{.}\PY{n}{decode}\PY{p}{(}\PY{p}{)}\PY{p}{)} 
\end{Verbatim}
\end{tcolorbox}

    \begin{Verbatim}[commandchars=\\\{\}]
String A
['Mehrzeiliger Text ', 'Zeile 1 ', 'Zeile 2']
['Mehrzeiliger Text \textbackslash{}nZeile 1 ', 'Zeile 2']
Mehrzeiliger Text
Zeile 1
Zeile 2
b'Text A'
Text A
    \end{Verbatim}

    \begin{tcolorbox}[breakable, size=fbox, boxrule=1pt, pad at break*=1mm,colback=cellbackground, colframe=cellborder]
\prompt{In}{incolor}{158}{\boxspacing}
\begin{Verbatim}[commandchars=\\\{\}]
\PY{c+c1}{\PYZsh{} String Funktionen \PYZsh{}5}

\PY{c+c1}{\PYZsh{} string.format(value1, value2...)}
\PY{n+nb}{print}\PY{p}{(}\PY{l+s+s2}{\PYZdq{}}\PY{l+s+s2}{Hallo mein Name ist }\PY{l+s+si}{\PYZob{}\PYZcb{}}\PY{l+s+s2}{\PYZdq{}}\PY{o}{.}\PY{n}{format}\PY{p}{(}\PY{l+s+s2}{\PYZdq{}}\PY{l+s+s2}{Donald}\PY{l+s+s2}{\PYZdq{}}\PY{p}{)}\PY{p}{)}   \PY{c+c1}{\PYZsh{} Platzhalter werden durch angegebene Werte ersetzt}
\PY{n+nb}{print}\PY{p}{(}\PY{l+s+s2}{\PYZdq{}}\PY{l+s+s2}{Hallo mein Name ist }\PY{l+s+si}{\PYZob{}\PYZcb{}}\PY{l+s+s2}{ }\PY{l+s+si}{\PYZob{}\PYZcb{}}\PY{l+s+s2}{\PYZdq{}}\PY{o}{.}\PY{n}{format}\PY{p}{(}\PY{l+s+s2}{\PYZdq{}}\PY{l+s+s2}{Donald}\PY{l+s+s2}{\PYZdq{}}\PY{p}{,}\PY{l+s+s2}{\PYZdq{}}\PY{l+s+s2}{Duck}\PY{l+s+s2}{\PYZdq{}}\PY{p}{)}\PY{p}{)}   \PY{c+c1}{\PYZsh{} Indentifikation durch Position}
\PY{n+nb}{print}\PY{p}{(}\PY{l+s+s2}{\PYZdq{}}\PY{l+s+s2}{Hallo mein Name ist }\PY{l+s+si}{\PYZob{}1\PYZcb{}}\PY{l+s+s2}{ }\PY{l+s+si}{\PYZob{}0\PYZcb{}}\PY{l+s+s2}{\PYZdq{}}\PY{o}{.}\PY{n}{format}\PY{p}{(}\PY{l+s+s2}{\PYZdq{}}\PY{l+s+s2}{Duck}\PY{l+s+s2}{\PYZdq{}}\PY{p}{,}\PY{l+s+s2}{\PYZdq{}}\PY{l+s+s2}{Donald}\PY{l+s+s2}{\PYZdq{}}\PY{p}{)}\PY{p}{)}   \PY{c+c1}{\PYZsh{} Indentifikation durch Index}
\PY{n+nb}{print}\PY{p}{(}\PY{l+s+s2}{\PYZdq{}}\PY{l+s+s2}{Hallo mein Name ist }\PY{l+s+si}{\PYZob{}fname\PYZcb{}}\PY{l+s+s2}{ }\PY{l+s+si}{\PYZob{}lname\PYZcb{}}\PY{l+s+s2}{\PYZdq{}}\PY{o}{.}\PY{n}{format}\PY{p}{(}\PY{n}{lname} \PY{o}{=} \PY{l+s+s2}{\PYZdq{}}\PY{l+s+s2}{Duck}\PY{l+s+s2}{\PYZdq{}}\PY{p}{,} \PY{n}{fname} \PY{o}{=} \PY{l+s+s2}{\PYZdq{}}\PY{l+s+s2}{Donald}\PY{l+s+s2}{\PYZdq{}}\PY{p}{)}\PY{p}{)}  \PY{c+c1}{\PYZsh{} Indentifikation durch Bezeichner}

\PY{c+c1}{\PYZsh{} Platzhalter können auch dir Formatierung angeben}
\PY{n+nb}{print}\PY{p}{(}\PY{l+s+s2}{\PYZdq{}}\PY{l+s+s2}{Die Spannung beträgt }\PY{l+s+si}{\PYZob{}voltage:.2f\PYZcb{}}\PY{l+s+s2}{ V}\PY{l+s+s2}{\PYZdq{}}\PY{o}{.}\PY{n}{format}\PY{p}{(}\PY{n}{voltage}\PY{o}{=}\PY{l+m+mi}{5}\PY{p}{)}\PY{p}{)}     \PY{c+c1}{\PYZsh{} Auf zweich Nachkommastellen genau}
\PY{n+nb}{print}\PY{p}{(}\PY{l+s+s2}{\PYZdq{}}\PY{l+s+s2}{Die Binärdarstellung von }\PY{l+s+si}{\PYZob{}0\PYZcb{}}\PY{l+s+s2}{ ist }\PY{l+s+si}{\PYZob{}0:b\PYZcb{}}\PY{l+s+s2}{\PYZdq{}}\PY{o}{.}\PY{n}{format}\PY{p}{(}\PY{l+m+mi}{5}\PY{p}{)}\PY{p}{)}           \PY{c+c1}{\PYZsh{} Binärdarstellung}
\PY{n+nb}{print}\PY{p}{(}\PY{l+s+s2}{\PYZdq{}}\PY{l+s+s2}{Die Binärdarstellung von }\PY{l+s+si}{\PYZob{}0:3\PYZcb{}}\PY{l+s+s2}{ ist }\PY{l+s+si}{\PYZob{}0:0\PYZgt{}8b\PYZcb{}}\PY{l+s+s2}{\PYZdq{}}\PY{o}{.}\PY{n}{format}\PY{p}{(}\PY{l+m+mi}{5}\PY{p}{)}\PY{p}{)}        \PY{c+c1}{\PYZsh{} Binär und Deziaml mit führenden Nullen}
\PY{n+nb}{print}\PY{p}{(}\PY{l+s+s2}{\PYZdq{}}\PY{l+s+s2}{Die Binärdarstellung von }\PY{l+s+si}{\PYZob{}0:3\PYZcb{}}\PY{l+s+s2}{ ist }\PY{l+s+si}{\PYZob{}0:8b\PYZcb{}}\PY{l+s+s2}{\PYZdq{}}\PY{o}{.}\PY{n}{format}\PY{p}{(}\PY{l+m+mi}{255}\PY{p}{)}\PY{p}{)}      \PY{c+c1}{\PYZsh{} Binär und Deziaml mit führenden Leerzeichen}
\PY{n+nb}{print}\PY{p}{(}\PY{l+s+s2}{\PYZdq{}}\PY{l+s+s2}{Die Hexadeziamldarstellung von }\PY{l+s+si}{\PYZob{}0\PYZcb{}}\PY{l+s+s2}{ ist }\PY{l+s+si}{\PYZob{}0:x\PYZcb{}}\PY{l+s+s2}{ oder }\PY{l+s+si}{\PYZob{}0:X\PYZcb{}}\PY{l+s+s2}{\PYZdq{}}\PY{o}{.}\PY{n}{format}\PY{p}{(}\PY{l+m+mi}{254}\PY{p}{)}\PY{p}{)}  \PY{c+c1}{\PYZsh{} Hexadezimaldarstellung}
\PY{n+nb}{print}\PY{p}{(}\PY{l+s+s2}{\PYZdq{}}\PY{l+s+s2}{Große Kommazahl: }\PY{l+s+si}{\PYZob{}0:n\PYZcb{}}\PY{l+s+s2}{ }\PY{l+s+se}{\PYZbs{}t}\PY{l+s+s2}{ }\PY{l+s+si}{\PYZob{}0:g\PYZcb{}}\PY{l+s+s2}{ }\PY{l+s+se}{\PYZbs{}t}\PY{l+s+s2}{ }\PY{l+s+si}{\PYZob{}0:5.2F\PYZcb{}}\PY{l+s+s2}{ }\PY{l+s+se}{\PYZbs{}t}\PY{l+s+s2}{ }\PY{l+s+si}{\PYZob{}0:+8.3\PYZcb{}}\PY{l+s+s2}{\PYZdq{}}\PY{o}{.}\PY{n}{format}\PY{p}{(}\PY{l+m+mf}{15.7324234}\PY{p}{)}\PY{p}{)}
\PY{n+nb}{print}\PY{p}{(}\PY{l+s+s2}{\PYZdq{}}\PY{l+s+s2}{Große Zahl: }\PY{l+s+si}{\PYZob{}0:n\PYZcb{}}\PY{l+s+s2}{ }\PY{l+s+se}{\PYZbs{}t}\PY{l+s+s2}{ }\PY{l+s+si}{\PYZob{}0:,n\PYZcb{}}\PY{l+s+s2}{ }\PY{l+s+se}{\PYZbs{}t}\PY{l+s+s2}{ }\PY{l+s+si}{\PYZob{}0:e\PYZcb{}}\PY{l+s+s2}{ }\PY{l+s+se}{\PYZbs{}t}\PY{l+s+s2}{ }\PY{l+s+si}{\PYZob{}0:E\PYZcb{}}\PY{l+s+s2}{\PYZdq{}}\PY{o}{.}\PY{n}{format}\PY{p}{(}\PY{l+m+mi}{100000000}\PY{p}{)}\PY{p}{)}
\end{Verbatim}
\end{tcolorbox}

    \begin{Verbatim}[commandchars=\\\{\}]
Hallo mein Name ist Donald
Hallo mein Name ist Donald Duck
Hallo mein Name ist Donald Duck
Hallo mein Name ist Donald Duck
Die Spannung beträgt 5.00 V
Die Binärdarstellung von 5 ist 101
Die Binärdarstellung von   5 ist 00000101
Die Binärdarstellung von 255 ist 11111111
Die Hexadeziamldarstellung von 254 ist fe oder FE
Große Kommazahl: 15.7324         15.7324         15.73      +15.7
Große Zahl: 100000000    100,000,000     1.000000e+08    1.000000E+08
    \end{Verbatim}

    \hypertarget{string-formatierungen}{%
\subsubsection{String Formatierungen}\label{string-formatierungen}}

https://www.programiz.com/python-programming/methods/string/format

Allgemein:
\{{[}Platzhalterindentifizierung{]}:{[}Positionierung{]}{[}Numerische
Darstellung{]}\}

\begin{longtable}[]{@{}lllll@{}}
\toprule
Numerisch & & & Positionierung & (in Verbindung mit Platzangabe) \\
\midrule
\endhead
Format & Platzhalter & & Format & Platzhalter \\
Dezimal & d & & Linksbündig & \textless{} \\
Binär & b & & Rechtsbündig & \textgreater{} \\
Hexadezimal (klein) & x & & Zentral & \^{} \\
Hexadezimal (groß) & X & & Vorzeichen links & = \\
Octal & o & & Vorzeichen bei positiv & + \\
Nummersich & n & & Vorzeichen nur bei negativ & - \\
Kommazahl & f & & Leerzeichen bei Vorzeichen & \\
Kommazahl (groß) & F & & Unterstrich um 1000er zu trennen & \_ \\
Nummer & n & & Komma um 1000er zu trennen & , \\
Wissenschaftlich (e) & e & & & \\
Wissenschaflitch (E) & E & & Vorkomma(X) Nachkommastellen (Y) & X. \\
Generell & g & & & \\
Generell & G & & & \\
Prozent & \% & & & \\
\bottomrule
\end{longtable}

    \hypertarget{listen-und-co-todo}{%
\subsection{Listen und Co (ToDo)}\label{listen-und-co-todo}}

\begin{itemize}
\tightlist
\item
  Listen {[}{]}: Gerodnet (index), Änderbar, Duplikate erlaubt
\item
  Tuple (): Geordnet (index), Nicht Änderbar, Duplikate erlaubt
\item
  Set \{\}: Ungeordnet (kein index), Erweiterbar, Keine Duplikate
\item
  Dictionary \{key:value\}: Geordnet*, Änderbar, Keine Duplikate
\end{itemize}

    \hypertarget{listen}{%
\subsubsection{Listen}\label{listen}}

\begin{itemize}
\tightlist
\item
  Eckige Klammern
\item
  Dupplikate erlaubt
\item
  Verschiedene Datentypen mischbar
\item
  Geordnet (Index)
\item
  Änderbar
\end{itemize}

    \begin{tcolorbox}[breakable, size=fbox, boxrule=1pt, pad at break*=1mm,colback=cellbackground, colframe=cellborder]
\prompt{In}{incolor}{66}{\boxspacing}
\begin{Verbatim}[commandchars=\\\{\}]
\PY{c+c1}{\PYZsh{} Listen erstellen}

\PY{n}{meineDaten1} \PY{o}{=} \PY{p}{[}\PY{l+m+mi}{1}\PY{p}{,}\PY{l+m+mi}{2}\PY{p}{,}\PY{l+m+mi}{3}\PY{p}{]}                 \PY{c+c1}{\PYZsh{} Definiert über eckige Klammern}
\PY{n}{meineDaten2} \PY{o}{=} \PY{n+nb}{list}\PY{p}{(}\PY{p}{(}\PY{l+s+s2}{\PYZdq{}}\PY{l+s+s2}{1}\PY{l+s+s2}{\PYZdq{}}\PY{p}{,}\PY{l+s+s2}{\PYZdq{}}\PY{l+s+s2}{2}\PY{l+s+s2}{\PYZdq{}}\PY{p}{,}\PY{l+s+s2}{\PYZdq{}}\PY{l+s+s2}{3}\PY{l+s+s2}{\PYZdq{}}\PY{p}{)}\PY{p}{)}     \PY{c+c1}{\PYZsh{} über den Listen\PYZhy{}Construktor ((\PYZhy{}beachten}
\PY{n}{meineDaten3} \PY{o}{=} \PY{p}{[}\PY{l+s+s2}{\PYZdq{}}\PY{l+s+s2}{eins}\PY{l+s+s2}{\PYZdq{}}\PY{p}{,}\PY{l+s+s2}{\PYZdq{}}\PY{l+s+s2}{zwei}\PY{l+s+s2}{\PYZdq{}}\PY{p}{,}\PY{l+s+s2}{\PYZdq{}}\PY{l+s+s2}{drei}\PY{l+s+s2}{\PYZdq{}}\PY{p}{,}\PY{l+s+s2}{\PYZdq{}}\PY{l+s+s2}{vier}\PY{l+s+s2}{\PYZdq{}}\PY{p}{]}
\PY{n}{meineDaten4} \PY{o}{=} \PY{p}{[}\PY{l+s+s2}{\PYZdq{}}\PY{l+s+s2}{eins}\PY{l+s+s2}{\PYZdq{}}\PY{p}{,}\PY{l+m+mi}{2}\PY{p}{,}\PY{l+s+s2}{\PYZdq{}}\PY{l+s+s2}{drei}\PY{l+s+s2}{\PYZdq{}}\PY{p}{,}\PY{k+kc}{True}\PY{p}{]} \PY{c+c1}{\PYZsh{} Es müssen nicht mal die gleichen Typen verwendet werden}

\PY{n+nb}{print}\PY{p}{(}\PY{n}{meineDaten1}\PY{p}{)}
\PY{n+nb}{print}\PY{p}{(}\PY{n}{meineDaten2}\PY{p}{)}
\PY{n+nb}{print}\PY{p}{(}\PY{n}{meineDaten3}\PY{p}{)}
\PY{n+nb}{print}\PY{p}{(}\PY{n}{meineDaten4}\PY{p}{)}

\PY{c+c1}{\PYZsh{} Länge einer Liste (vgl. Strings)}
\PY{n+nb}{print}\PY{p}{(}\PY{n+nb}{len}\PY{p}{(}\PY{n}{meineDaten1}\PY{p}{)}\PY{p}{)}

\PY{c+c1}{\PYZsh{} Zugriff auf Elemente}
\PY{n+nb}{print}\PY{p}{(}\PY{n}{meineDaten4}\PY{p}{[}\PY{l+m+mi}{0}\PY{p}{]}\PY{p}{)}
\PY{n+nb}{print}\PY{p}{(}\PY{n}{meineDaten2}\PY{p}{[}\PY{o}{\PYZhy{}}\PY{l+m+mi}{1}\PY{p}{]}\PY{p}{)}
\PY{n+nb}{print}\PY{p}{(}\PY{n}{meineDaten3}\PY{p}{[}\PY{l+m+mi}{1}\PY{p}{:}\PY{o}{\PYZhy{}}\PY{l+m+mi}{1}\PY{p}{]}\PY{p}{)}

\PY{c+c1}{\PYZsh{} Loop List}
\PY{k}{for} \PY{n}{x} \PY{o+ow}{in} \PY{n}{meineDaten1}\PY{p}{:}
    \PY{n+nb}{print}\PY{p}{(}\PY{n}{x}\PY{p}{)}
\PY{c+c1}{\PYZsh{} Abkürzung für Durchlauf}
\PY{p}{[}\PY{n+nb}{print}\PY{p}{(}\PY{n}{x}\PY{p}{)} \PY{k}{for} \PY{n}{x} \PY{o+ow}{in} \PY{n}{meineDaten3}\PY{p}{]}

\PY{n+nb}{print}\PY{p}{(}\PY{n+nb}{sum}\PY{p}{(}\PY{n}{meineDaten1}\PY{p}{)}\PY{p}{)} \PY{c+c1}{\PYZsh{} Summer einer gesamten numerischen Liste}
\end{Verbatim}
\end{tcolorbox}

    \begin{Verbatim}[commandchars=\\\{\}]
[1, 2, 3]
['1', '2', '3']
['eins', 'zwei', 'drei', 'vier']
['eins', 2, 'drei', True]
3
eins
3
['zwei', 'drei']
1
2
3
eins
zwei
drei
vier
6
    \end{Verbatim}

    \begin{tcolorbox}[breakable, size=fbox, boxrule=1pt, pad at break*=1mm,colback=cellbackground, colframe=cellborder]
\prompt{In}{incolor}{67}{\boxspacing}
\begin{Verbatim}[commandchars=\\\{\}]
\PY{c+c1}{\PYZsh{} Elemente finden/filtern}
\PY{n+nb}{print}\PY{p}{(}\PY{n}{meineDaten3}\PY{p}{)}

\PY{n+nb}{print}\PY{p}{(}\PY{l+s+s2}{\PYZdq{}}\PY{l+s+s2}{vier}\PY{l+s+s2}{\PYZdq{}} \PY{o+ow}{in} \PY{n}{meineDaten3}\PY{p}{)}
\PY{n+nb}{print}\PY{p}{(}\PY{l+s+s2}{\PYZdq{}}\PY{l+s+s2}{fünf}\PY{l+s+s2}{\PYZdq{}} \PY{o+ow}{not} \PY{o+ow}{in} \PY{n}{meineDaten3}\PY{p}{)}
\PY{n+nb}{print}\PY{p}{(}\PY{n}{meineDaten3}\PY{o}{.}\PY{n}{index}\PY{p}{(}\PY{l+s+s2}{\PYZdq{}}\PY{l+s+s2}{vier}\PY{l+s+s2}{\PYZdq{}}\PY{p}{)}\PY{p}{)} \PY{c+c1}{\PYZsh{} Gibt den index des erste fundes}

\PY{n}{neueListe} \PY{o}{=} \PY{p}{[}\PY{p}{]}        \PY{c+c1}{\PYZsh{} ! Liste muss als solche initialisiert sein}
\PY{k}{for} \PY{n}{x} \PY{o+ow}{in} \PY{n}{meineDaten3}\PY{p}{:}
  \PY{k}{if} \PY{l+s+s2}{\PYZdq{}}\PY{l+s+s2}{r}\PY{l+s+s2}{\PYZdq{}} \PY{o+ow}{in} \PY{n}{x}\PY{p}{:}          \PY{c+c1}{\PYZsh{} Alle Elemente die ein r enthalten kommen in die neue Liste}
    \PY{n}{neueListe}\PY{o}{.}\PY{n}{append}\PY{p}{(}\PY{n}{x}\PY{p}{)}
\PY{n+nb}{print}\PY{p}{(}\PY{n}{neueListe}\PY{p}{)}

\PY{c+c1}{\PYZsh{}newlist = [expression for item in iterable if condition == True] }
\PY{n}{neueListe2} \PY{o}{=} \PY{p}{[}\PY{n}{x} \PY{k}{for} \PY{n}{x} \PY{o+ow}{in} \PY{n}{meineDaten3} \PY{k}{if} \PY{l+s+s2}{\PYZdq{}}\PY{l+s+s2}{r}\PY{l+s+s2}{\PYZdq{}} \PY{o+ow}{in} \PY{n}{x}\PY{p}{]}  \PY{c+c1}{\PYZsh{} Selbe Funktion in Kurzschreibweise}
\PY{n+nb}{print}\PY{p}{(}\PY{n}{neueListe2}\PY{p}{)}

\PY{n}{neueListe2} \PY{o}{=} \PY{p}{[}\PY{n}{x} \PY{k}{if} \PY{l+s+s2}{\PYZdq{}}\PY{l+s+s2}{r}\PY{l+s+s2}{\PYZdq{}} \PY{o+ow}{in} \PY{n}{x} \PY{k}{else} \PY{l+s+s2}{\PYZdq{}}\PY{l+s+s2}{\PYZhy{}}\PY{l+s+s2}{\PYZdq{}} \PY{k}{for} \PY{n}{x} \PY{o+ow}{in} \PY{n}{meineDaten3}\PY{p}{]}  \PY{c+c1}{\PYZsh{} Einfügen wenn \PYZsq{}r\PYZsq{} in Element sonst \PYZsq{}\PYZhy{}\PYZsq{} einfügen}
\PY{n+nb}{print}\PY{p}{(}\PY{n}{neueListe2}\PY{p}{)}

\PY{n}{neueListe3} \PY{o}{=}  \PY{p}{[}\PY{n}{x} \PY{k}{for} \PY{n}{x} \PY{o+ow}{in} \PY{n+nb}{range}\PY{p}{(}\PY{l+m+mi}{10}\PY{p}{)}\PY{p}{]}  \PY{c+c1}{\PYZsh{} Neue Liste von Elementen mit Range }
\PY{n+nb}{print}\PY{p}{(}\PY{n}{neueListe3}\PY{p}{)}
\end{Verbatim}
\end{tcolorbox}

    \begin{Verbatim}[commandchars=\\\{\}]
['eins', 'zwei', 'drei', 'vier']
True
True
3
['drei', 'vier']
['drei', 'vier']
['-', '-', 'drei', 'vier']
[0, 1, 2, 3, 4, 5, 6, 7, 8, 9]
    \end{Verbatim}

    \begin{tcolorbox}[breakable, size=fbox, boxrule=1pt, pad at break*=1mm,colback=cellbackground, colframe=cellborder]
\prompt{In}{incolor}{68}{\boxspacing}
\begin{Verbatim}[commandchars=\\\{\}]
\PY{c+c1}{\PYZsh{} Listen verändern \PYZsh{}1}

\PY{c+c1}{\PYZsh{} Einzelnes Element verändern}
\PY{n}{meineDaten3}\PY{p}{[}\PY{l+m+mi}{3}\PY{p}{]} \PY{o}{=} \PY{l+s+s2}{\PYZdq{}}\PY{l+s+s2}{drei}\PY{l+s+s2}{\PYZdq{}}
\PY{n+nb}{print}\PY{p}{(}\PY{n}{meineDaten3}\PY{p}{)}

\PY{c+c1}{\PYZsh{} Teilliste Verändern}
\PY{n}{meineDaten4}\PY{p}{[}\PY{l+m+mi}{1}\PY{p}{:}\PY{l+m+mi}{3}\PY{p}{]} \PY{o}{=} \PY{p}{[}\PY{l+s+s2}{\PYZdq{}}\PY{l+s+s2}{zwei}\PY{l+s+s2}{\PYZdq{}}\PY{p}{,}\PY{l+s+s2}{\PYZdq{}}\PY{l+s+s2}{drei}\PY{l+s+s2}{\PYZdq{}}\PY{p}{]}
\PY{n+nb}{print}\PY{p}{(}\PY{n}{meineDaten4}\PY{p}{)}

\PY{c+c1}{\PYZsh{} Teilliste Verändern}
\PY{n}{meineDaten4}\PY{p}{[}\PY{l+m+mi}{1}\PY{p}{:}\PY{l+m+mi}{2}\PY{p}{]} \PY{o}{=} \PY{p}{[}\PY{l+s+s2}{\PYZdq{}}\PY{l+s+s2}{zwei}\PY{l+s+s2}{\PYZdq{}}\PY{p}{,}\PY{l+s+s2}{\PYZdq{}}\PY{l+s+s2}{drei}\PY{l+s+s2}{\PYZdq{}}\PY{p}{]}    \PY{c+c1}{\PYZsh{} Hier wird die Liste an der Stelle erweitert}
\PY{n+nb}{print}\PY{p}{(}\PY{n}{meineDaten4}\PY{p}{)}

\PY{n}{meineDaten4}\PY{o}{.}\PY{n}{insert}\PY{p}{(}\PY{o}{\PYZhy{}}\PY{l+m+mi}{1}\PY{p}{,}\PY{l+s+s2}{\PYZdq{}}\PY{l+s+s2}{vier}\PY{l+s+s2}{\PYZdq{}}\PY{p}{)}         \PY{c+c1}{\PYZsh{} Einfügen an Stelle X}
\PY{n+nb}{print}\PY{p}{(}\PY{n}{meineDaten4}\PY{p}{)}

\PY{n}{meineDaten3}\PY{o}{.}\PY{n}{append}\PY{p}{(}\PY{l+s+s2}{\PYZdq{}}\PY{l+s+s2}{vier}\PY{l+s+s2}{\PYZdq{}}\PY{p}{)}            \PY{c+c1}{\PYZsh{} Hinten anhängen}
\PY{n+nb}{print}\PY{p}{(}\PY{n}{meineDaten3}\PY{p}{)}

\PY{n}{meineDaten4}\PY{o}{.}\PY{n}{extend}\PY{p}{(}\PY{n}{meineDaten3}\PY{p}{)}       \PY{c+c1}{\PYZsh{} Liste um Liste erweitern}
\PY{n+nb}{print}\PY{p}{(}\PY{n}{meineDaten4}\PY{p}{)}
\end{Verbatim}
\end{tcolorbox}

    \begin{Verbatim}[commandchars=\\\{\}]
['eins', 'zwei', 'drei', 'drei']
['eins', 'zwei', 'drei', True]
['eins', 'zwei', 'drei', 'drei', True]
['eins', 'zwei', 'drei', 'drei', 'vier', True]
['eins', 'zwei', 'drei', 'drei', 'vier']
['eins', 'zwei', 'drei', 'drei', 'vier', True, 'eins', 'zwei', 'drei', 'drei',
'vier']
    \end{Verbatim}

    \begin{tcolorbox}[breakable, size=fbox, boxrule=1pt, pad at break*=1mm,colback=cellbackground, colframe=cellborder]
\prompt{In}{incolor}{69}{\boxspacing}
\begin{Verbatim}[commandchars=\\\{\}]
\PY{c+c1}{\PYZsh{} Weitere Listen Funktionen}

\PY{n}{meineDaten1}\PY{p}{[}\PY{l+m+mi}{1}\PY{p}{:}\PY{l+m+mi}{2}\PY{p}{]} \PY{o}{=} \PY{p}{[}\PY{l+m+mi}{1}\PY{p}{,}\PY{l+m+mi}{2}\PY{p}{,}\PY{l+m+mi}{3}\PY{p}{,}\PY{l+m+mi}{4}\PY{p}{]}    \PY{c+c1}{\PYZsh{} Hier wird die Liste an der Stelle erweitert}
\PY{n+nb}{print}\PY{p}{(}\PY{n}{meineDaten1}\PY{p}{)}

\PY{n+nb}{print}\PY{p}{(}\PY{n}{meineDaten1}\PY{o}{.}\PY{n}{count}\PY{p}{(}\PY{l+m+mi}{1}\PY{p}{)}\PY{p}{)}     \PY{c+c1}{\PYZsh{} Wieviele von diesem Element existieren}

\PY{n}{meineDaten1}\PY{o}{.}\PY{n}{sort}\PY{p}{(}\PY{p}{)}              \PY{c+c1}{\PYZsh{} Sortiet die Elemente einer List }
\PY{n}{meineDaten3}\PY{o}{.}\PY{n}{sort}\PY{p}{(}\PY{n}{reverse} \PY{o}{=} \PY{k+kc}{True}\PY{p}{)}
\PY{n+nb}{print}\PY{p}{(}\PY{n}{meineDaten1}\PY{p}{)}
\PY{n+nb}{print}\PY{p}{(}\PY{n}{meineDaten3}\PY{p}{)}

\PY{n}{meineDaten1}\PY{o}{.}\PY{n}{reverse}\PY{p}{(}\PY{p}{)}           \PY{c+c1}{\PYZsh{} Dreht die Reihenfolge einer Liste um}
\PY{n+nb}{print}\PY{p}{(}\PY{n}{meineDaten1}\PY{p}{)}
\end{Verbatim}
\end{tcolorbox}

    \begin{Verbatim}[commandchars=\\\{\}]
[1, 1, 2, 3, 4, 3]
2
[1, 1, 2, 3, 3, 4]
['zwei', 'vier', 'eins', 'drei', 'drei']
[4, 3, 3, 2, 1, 1]
    \end{Verbatim}

    \begin{tcolorbox}[breakable, size=fbox, boxrule=1pt, pad at break*=1mm,colback=cellbackground, colframe=cellborder]
\prompt{In}{incolor}{27}{\boxspacing}
\begin{Verbatim}[commandchars=\\\{\}]
\PY{c+c1}{\PYZsh{} Liste Verändern \PYZsh{}2}

\PY{n+nb}{print}\PY{p}{(}\PY{n}{meineDaten4}\PY{p}{)}

\PY{n+nb}{print}\PY{p}{(}\PY{n}{meineDaten4}\PY{o}{.}\PY{n}{remove}\PY{p}{(}\PY{l+s+s2}{\PYZdq{}}\PY{l+s+s2}{drei}\PY{l+s+s2}{\PYZdq{}}\PY{p}{)}\PY{p}{)} \PY{c+c1}{\PYZsh{} entfernt nur ein Element bei mehreren!}
\PY{n+nb}{print}\PY{p}{(}\PY{n}{meineDaten4}\PY{p}{)}

\PY{n+nb}{print}\PY{p}{(}\PY{n}{meineDaten4}\PY{o}{.}\PY{n}{pop}\PY{p}{(}\PY{l+m+mi}{2}\PY{p}{)}\PY{p}{)} \PY{c+c1}{\PYZsh{} entfernt am angegeben Index (und gibt Element zurück)}
\PY{n+nb}{print}\PY{p}{(}\PY{n}{meineDaten4}\PY{p}{)}

\PY{n+nb}{print}\PY{p}{(}\PY{n}{meineDaten4}\PY{o}{.}\PY{n}{pop}\PY{p}{(}\PY{p}{)}\PY{p}{)} \PY{c+c1}{\PYZsh{} entfernt ohne Index das letzte Element(und gibt Element zurück)}
\PY{n+nb}{print}\PY{p}{(}\PY{n}{meineDaten4}\PY{p}{)}

\PY{k}{del} \PY{n}{meineDaten4}\PY{p}{[}\PY{l+m+mi}{0}\PY{p}{]}       \PY{c+c1}{\PYZsh{} entfernt erste Element über del}
\PY{n+nb}{print}\PY{p}{(}\PY{n}{meineDaten4}\PY{p}{)}

\PY{n}{meineDaten4}\PY{o}{.}\PY{n}{clear}\PY{p}{(}\PY{p}{)}      \PY{c+c1}{\PYZsh{} Löscht komplette Liste ! nicht die Liste selbst}
\PY{n+nb}{print}\PY{p}{(}\PY{n}{meineDaten4}\PY{p}{)}

\PY{n+nb}{print}\PY{p}{(}\PY{n+nb}{any}\PY{p}{(}\PY{n}{meineDaten4}\PY{p}{)}\PY{p}{)}  \PY{c+c1}{\PYZsh{} Befindet sich mindestens ein Element in der Liste?}
\end{Verbatim}
\end{tcolorbox}

    \begin{Verbatim}[commandchars=\\\{\}]
['eins', 'zwei', 'drei', 'drei', 'vier', True, 'eins', 'zwei', 'drei', 'drei',
'vier']
None
['eins', 'zwei', 'drei', 'vier', True, 'eins', 'zwei', 'drei', 'drei', 'vier']
drei
['eins', 'zwei', 'vier', True, 'eins', 'zwei', 'drei', 'drei', 'vier']
vier
['eins', 'zwei', 'vier', True, 'eins', 'zwei', 'drei', 'drei']
['zwei', 'vier', True, 'eins', 'zwei', 'drei', 'drei']
[]
False
    \end{Verbatim}

    \begin{tcolorbox}[breakable, size=fbox, boxrule=1pt, pad at break*=1mm,colback=cellbackground, colframe=cellborder]
\prompt{In}{incolor}{354}{\boxspacing}
\begin{Verbatim}[commandchars=\\\{\}]
\PY{c+c1}{\PYZsh{} Listen Kopieren}

\PY{n}{meineDaten5} \PY{o}{=} \PY{n}{meineDaten3}     
\PY{n}{meineDaten6} \PY{o}{=} \PY{n}{meineDaten3}\PY{o}{.}\PY{n}{copy}\PY{p}{(}\PY{p}{)}
\PY{n+nb}{print}\PY{p}{(}\PY{n}{meineDaten5}\PY{p}{)}
\PY{n+nb}{print}\PY{p}{(}\PY{n}{meineDaten6}\PY{p}{)}

\PY{n}{meineDaten3}\PY{o}{.}\PY{n}{clear}\PY{p}{(}\PY{p}{)}      \PY{c+c1}{\PYZsh{} Löscht komplette Liste ! nicht die Liste selbst}
\PY{n+nb}{print}\PY{p}{(}\PY{n}{meineDaten3}\PY{p}{)}
\PY{n+nb}{print}\PY{p}{(}\PY{n}{meineDaten5}\PY{p}{)}       \PY{c+c1}{\PYZsh{} Liste5 war nur ein \PYZdq{}Verweis\PYZdq{} auf Liste3}
\PY{n+nb}{print}\PY{p}{(}\PY{n}{meineDaten6}\PY{p}{)}       \PY{c+c1}{\PYZsh{} Liste6 war eine echte Kopie von Liste3 }
\end{Verbatim}
\end{tcolorbox}

    \begin{Verbatim}[commandchars=\\\{\}]
['zwei', 'vier', 'vier', 'eins', 'drei', 'drei']
['zwei', 'vier', 'vier', 'eins', 'drei', 'drei']
[]
[]
['zwei', 'vier', 'vier', 'eins', 'drei', 'drei']
    \end{Verbatim}

    \begin{tcolorbox}[breakable, size=fbox, boxrule=1pt, pad at break*=1mm,colback=cellbackground, colframe=cellborder]
\prompt{In}{incolor}{62}{\boxspacing}
\begin{Verbatim}[commandchars=\\\{\}]
\PY{n}{help}\PY{p}{(}\PY{n+nb}{list}\PY{p}{)}
\end{Verbatim}
\end{tcolorbox}

    \begin{Verbatim}[commandchars=\\\{\}]
object <class 'list'> is of type type
  append -- <function>
  clear -- <function>
  copy -- <function>
  count -- <function>
  extend -- <function>
  index -- <function>
  insert -- <function>
  pop -- <function>
  remove -- <function>
  reverse -- <function>
  sort -- <function>
    \end{Verbatim}

    \hypertarget{tuple}{%
\subsubsection{Tuple}\label{tuple}}

\begin{itemize}
\tightlist
\item
  Runde Klammern
\item
  Dupplikate erlaubt
\item
  Verschiedene Datentypen mischbar
\item
  Geordnet (Index)
\item
  Nicht Änderbar
\end{itemize}

    \begin{tcolorbox}[breakable, size=fbox, boxrule=1pt, pad at break*=1mm,colback=cellbackground, colframe=cellborder]
\prompt{In}{incolor}{108}{\boxspacing}
\begin{Verbatim}[commandchars=\\\{\}]
\PY{c+c1}{\PYZsh{} Tuple erstellen}

\PY{n}{meineDaten1} \PY{o}{=} \PY{p}{(}\PY{l+m+mi}{1}\PY{p}{,}\PY{l+m+mi}{2}\PY{p}{,}\PY{l+m+mi}{3}\PY{p}{)}                  \PY{c+c1}{\PYZsh{} Hier Runde Klammern!}
\PY{n}{meineDaten2} \PY{o}{=} \PY{n+nb}{tuple}\PY{p}{(}\PY{p}{(}\PY{l+s+s2}{\PYZdq{}}\PY{l+s+s2}{1}\PY{l+s+s2}{\PYZdq{}}\PY{p}{,}\PY{l+s+s2}{\PYZdq{}}\PY{l+s+s2}{2}\PY{l+s+s2}{\PYZdq{}}\PY{p}{,}\PY{l+s+s2}{\PYZdq{}}\PY{l+s+s2}{3}\PY{l+s+s2}{\PYZdq{}}\PY{p}{)}\PY{p}{)}     \PY{c+c1}{\PYZsh{} über den Tuple\PYZhy{}Construktor ((\PYZhy{}beachten}
\PY{n}{meineDaten3} \PY{o}{=} \PY{p}{(}\PY{l+s+s2}{\PYZdq{}}\PY{l+s+s2}{eins}\PY{l+s+s2}{\PYZdq{}}\PY{p}{,}\PY{l+s+s2}{\PYZdq{}}\PY{l+s+s2}{zwei}\PY{l+s+s2}{\PYZdq{}}\PY{p}{,}\PY{l+s+s2}{\PYZdq{}}\PY{l+s+s2}{drei}\PY{l+s+s2}{\PYZdq{}}\PY{p}{,}\PY{l+s+s2}{\PYZdq{}}\PY{l+s+s2}{vier}\PY{l+s+s2}{\PYZdq{}}\PY{p}{)}
\PY{n}{meineDaten4} \PY{o}{=} \PY{p}{(}\PY{l+s+s2}{\PYZdq{}}\PY{l+s+s2}{eins}\PY{l+s+s2}{\PYZdq{}}\PY{p}{,}\PY{l+m+mi}{2}\PY{p}{,}\PY{l+s+s2}{\PYZdq{}}\PY{l+s+s2}{drei}\PY{l+s+s2}{\PYZdq{}}\PY{p}{,}\PY{k+kc}{True}\PY{p}{)} \PY{c+c1}{\PYZsh{} Es müssen nicht mal die gleichen Typen verwendet werden}
\PY{n}{meineDaten5} \PY{o}{=} \PY{p}{(}\PY{l+m+mi}{1}\PY{p}{,}\PY{p}{)}                   \PY{c+c1}{\PYZsh{} Tuple mit nur einem Element müssen ein Komma enthalten, sonst kein Tuple!}

\PY{n+nb}{print}\PY{p}{(}\PY{n}{meineDaten1}\PY{p}{)}
\PY{n+nb}{print}\PY{p}{(}\PY{n}{meineDaten2}\PY{p}{)}
\PY{n+nb}{print}\PY{p}{(}\PY{n}{meineDaten3}\PY{p}{)}
\PY{n+nb}{print}\PY{p}{(}\PY{n}{meineDaten4}\PY{p}{)}

\PY{c+c1}{\PYZsh{} Länge einer Liste (vgl. Strings)}
\PY{n+nb}{print}\PY{p}{(}\PY{n+nb}{len}\PY{p}{(}\PY{n}{meineDaten1}\PY{p}{)}\PY{p}{)}

\PY{c+c1}{\PYZsh{} Zugriff auf Elemente}
\PY{n+nb}{print}\PY{p}{(}\PY{n}{meineDaten4}\PY{p}{[}\PY{l+m+mi}{0}\PY{p}{]}\PY{p}{)}
\PY{n+nb}{print}\PY{p}{(}\PY{n}{meineDaten2}\PY{p}{[}\PY{o}{\PYZhy{}}\PY{l+m+mi}{1}\PY{p}{]}\PY{p}{)}
\PY{n+nb}{print}\PY{p}{(}\PY{n}{meineDaten3}\PY{p}{[}\PY{l+m+mi}{1}\PY{p}{:}\PY{o}{\PYZhy{}}\PY{l+m+mi}{1}\PY{p}{]}\PY{p}{)}

\PY{c+c1}{\PYZsh{} Loop List}
\PY{k}{for} \PY{n}{x} \PY{o+ow}{in} \PY{n}{meineDaten1}\PY{p}{:}
    \PY{n+nb}{print}\PY{p}{(}\PY{n}{x}\PY{p}{)}
\PY{c+c1}{\PYZsh{} Abkürzung für Durchlauf}
\PY{p}{[}\PY{n+nb}{print}\PY{p}{(}\PY{n}{x}\PY{p}{)} \PY{k}{for} \PY{n}{x} \PY{o+ow}{in} \PY{n}{meineDaten3}\PY{p}{]}

\PY{n+nb}{print}\PY{p}{(}\PY{n+nb}{sum}\PY{p}{(}\PY{n}{meineDaten1}\PY{p}{)}\PY{p}{)} \PY{c+c1}{\PYZsh{} Summer einer gesamten numerischen Liste}
\end{Verbatim}
\end{tcolorbox}

    \begin{Verbatim}[commandchars=\\\{\}]
(1, 2, 3)
('1', '2', '3')
('eins', 'zwei', 'drei', 'vier')
('eins', 2, 'drei', True)
3
eins
3
('zwei', 'drei')
1
2
3
eins
zwei
drei
vier
6
    \end{Verbatim}

    \begin{tcolorbox}[breakable, size=fbox, boxrule=1pt, pad at break*=1mm,colback=cellbackground, colframe=cellborder]
\prompt{In}{incolor}{109}{\boxspacing}
\begin{Verbatim}[commandchars=\\\{\}]
\PY{c+c1}{\PYZsh{} Elemente finden/filtern}
\PY{n+nb}{print}\PY{p}{(}\PY{n}{meineDaten3}\PY{p}{)}

\PY{n+nb}{print}\PY{p}{(}\PY{l+s+s2}{\PYZdq{}}\PY{l+s+s2}{vier}\PY{l+s+s2}{\PYZdq{}} \PY{o+ow}{in} \PY{n}{meineDaten3}\PY{p}{)}
\PY{n+nb}{print}\PY{p}{(}\PY{l+s+s2}{\PYZdq{}}\PY{l+s+s2}{fünf}\PY{l+s+s2}{\PYZdq{}} \PY{o+ow}{not} \PY{o+ow}{in} \PY{n}{meineDaten3}\PY{p}{)}
\PY{n+nb}{print}\PY{p}{(}\PY{n}{meineDaten3}\PY{o}{.}\PY{n}{index}\PY{p}{(}\PY{l+s+s2}{\PYZdq{}}\PY{l+s+s2}{vier}\PY{l+s+s2}{\PYZdq{}}\PY{p}{)}\PY{p}{)} \PY{c+c1}{\PYZsh{} Gibt den index des erste fundes}

\PY{n}{neueListe} \PY{o}{=} \PY{p}{[}\PY{p}{]}        \PY{c+c1}{\PYZsh{} ! Liste muss als solche initialisiert sein}
\PY{k}{for} \PY{n}{x} \PY{o+ow}{in} \PY{n}{meineDaten3}\PY{p}{:}
    \PY{k}{if} \PY{l+s+s2}{\PYZdq{}}\PY{l+s+s2}{r}\PY{l+s+s2}{\PYZdq{}} \PY{o+ow}{in} \PY{n}{x}\PY{p}{:}          \PY{c+c1}{\PYZsh{} Alle Elemente die ein r enthalten kommen in die neue Liste}
        \PY{n}{neueListe}\PY{o}{.}\PY{n}{append}\PY{p}{(}\PY{n}{x}\PY{p}{)}
\PY{n+nb}{print}\PY{p}{(}\PY{n}{neueListe}\PY{p}{)}

\PY{c+c1}{\PYZsh{}newlist = [expression for item in iterable if condition == True] }
\PY{n}{neueListe2} \PY{o}{=} \PY{p}{[}\PY{n}{x} \PY{k}{for} \PY{n}{x} \PY{o+ow}{in} \PY{n}{meineDaten3} \PY{k}{if} \PY{l+s+s2}{\PYZdq{}}\PY{l+s+s2}{r}\PY{l+s+s2}{\PYZdq{}} \PY{o+ow}{in} \PY{n}{x}\PY{p}{]}  \PY{c+c1}{\PYZsh{} Selbe Funktion in Kurzschreibweise}
\PY{n+nb}{print}\PY{p}{(}\PY{n}{neueListe2}\PY{p}{)}

\PY{n}{neueListe2} \PY{o}{=} \PY{p}{[}\PY{n}{x} \PY{k}{if} \PY{l+s+s2}{\PYZdq{}}\PY{l+s+s2}{r}\PY{l+s+s2}{\PYZdq{}} \PY{o+ow}{in} \PY{n}{x} \PY{k}{else} \PY{l+s+s2}{\PYZdq{}}\PY{l+s+s2}{\PYZhy{}}\PY{l+s+s2}{\PYZdq{}} \PY{k}{for} \PY{n}{x} \PY{o+ow}{in} \PY{n}{meineDaten3}\PY{p}{]}  \PY{c+c1}{\PYZsh{} Einfügen wenn \PYZsq{}r\PYZsq{} in Element sonst \PYZsq{}\PYZhy{}\PYZsq{} einfügen}
\PY{n+nb}{print}\PY{p}{(}\PY{n}{neueListe2}\PY{p}{)}
\end{Verbatim}
\end{tcolorbox}

    \begin{Verbatim}[commandchars=\\\{\}]
('eins', 'zwei', 'drei', 'vier')
True
True
3
['drei', 'vier']
['drei', 'vier']
['-', '-', 'drei', 'vier']
<generator object '<genexpr>' at 20014210>
    \end{Verbatim}

    \begin{tcolorbox}[breakable, size=fbox, boxrule=1pt, pad at break*=1mm,colback=cellbackground, colframe=cellborder]
\prompt{In}{incolor}{91}{\boxspacing}
\begin{Verbatim}[commandchars=\\\{\}]
\PY{c+c1}{\PYZsh{} Tuple verändern \PYZsh{}1}

\PY{n+nb}{print}\PY{p}{(}\PY{n}{meineDaten3}\PY{p}{)}
\PY{c+c1}{\PYZsh{} meineDaten3[3] = \PYZdq{}drei\PYZdq{} \PYZsh{}! Tuple lassen sich nicht ändern!}

\PY{c+c1}{\PYZsh{} Workaround über Listen}
\PY{n}{neueListe} \PY{o}{=} \PY{n+nb}{list}\PY{p}{(}\PY{n}{meineDaten3}\PY{p}{)}
\PY{n+nb}{print}\PY{p}{(}\PY{n}{neueListe}\PY{p}{)}
\PY{n}{neueListe}\PY{p}{[}\PY{l+m+mi}{3}\PY{p}{]} \PY{o}{=} \PY{l+s+s2}{\PYZdq{}}\PY{l+s+s2}{drei}\PY{l+s+s2}{\PYZdq{}}
\PY{n}{meineDaten3}  \PY{o}{=} \PY{n+nb}{tuple}\PY{p}{(}\PY{n}{neueListe}\PY{p}{)}
\PY{n+nb}{print}\PY{p}{(}\PY{n}{meineDaten3}\PY{p}{)}
\end{Verbatim}
\end{tcolorbox}

    \begin{Verbatim}[commandchars=\\\{\}]
('eins', 'zwei', 'drei', 'vier')
['eins', 'zwei', 'drei', 'vier']
['eins', 'zwei', 'drei', 'vier']
('eins', 'zwei', 'drei', 'drei')
    \end{Verbatim}

    \begin{tcolorbox}[breakable, size=fbox, boxrule=1pt, pad at break*=1mm,colback=cellbackground, colframe=cellborder]
\prompt{In}{incolor}{100}{\boxspacing}
\begin{Verbatim}[commandchars=\\\{\}]
\PY{c+c1}{\PYZsh{} Weitere Tuple Funktionen}

\PY{c+c1}{\PYZsh{} Unpacking}
\PY{n}{varEins}\PY{p}{,} \PY{n}{varZwei}\PY{p}{,} \PY{n}{varDrei} \PY{o}{=} \PY{n}{meineDaten1}    \PY{c+c1}{\PYZsh{} Tuple wird direkt Variablen zugeordnet}
\PY{n+nb}{print}\PY{p}{(}\PY{n}{varZwei}\PY{p}{)}

\PY{n}{varEins}\PY{p}{,} \PY{n}{varZwei}\PY{p}{,} \PY{o}{*}\PY{n}{varDrei} \PY{o}{=} \PY{n}{meineDaten3}   \PY{c+c1}{\PYZsh{} Asterisk ermöglicht zuweisung bei weniger Variablen}
\PY{n+nb}{print}\PY{p}{(}\PY{n}{varDrei}\PY{p}{)}

\PY{n}{varEins}\PY{p}{,} \PY{o}{*}\PY{n}{varZwei}\PY{p}{,} \PY{n}{varDrei} \PY{o}{=} \PY{n}{meineDaten4}   \PY{c+c1}{\PYZsh{} Asterisk muss nicht an letzer Stelle stehen, Variable mit * wird Liste}
\PY{n+nb}{print}\PY{p}{(}\PY{n}{varZwei}\PY{p}{)}

\PY{n}{neuesTuple1} \PY{o}{=} \PY{n}{meineDaten1} \PY{o}{+} \PY{n}{meineDaten2}    \PY{c+c1}{\PYZsh{} Zusammenführen von Tuplen}
\PY{n+nb}{print}\PY{p}{(}\PY{n}{neuesTuple1}\PY{p}{)}

\PY{n}{neuesTuple2} \PY{o}{=} \PY{n}{meineDaten1} \PY{o}{*} \PY{l+m+mi}{2}              \PY{c+c1}{\PYZsh{} Multiplizieren eines Tuples }
\PY{n+nb}{print}\PY{p}{(}\PY{n}{neuesTuple2}\PY{p}{)}

\PY{n+nb}{print}\PY{p}{(}\PY{n}{neuesTuple2}\PY{o}{.}\PY{n}{count}\PY{p}{(}\PY{l+m+mi}{2}\PY{p}{)}\PY{p}{)}                \PY{c+c1}{\PYZsh{} Wieoft kommmt ein Element vor ?}
\PY{n+nb}{print}\PY{p}{(}\PY{n}{neuesTuple2}\PY{o}{.}\PY{n}{index}\PY{p}{(}\PY{l+m+mi}{2}\PY{p}{)}\PY{p}{)}                \PY{c+c1}{\PYZsh{} An welcher (Index\PYZhy{})Stelle erscheint es zuerst?}
\end{Verbatim}
\end{tcolorbox}

    \begin{Verbatim}[commandchars=\\\{\}]
2
['drei', 'drei']
[2, 'drei']
(1, 2, 3, '1', '2', '3')
(1, 2, 3, 1, 2, 3)
2
1
    \end{Verbatim}

    \begin{tcolorbox}[breakable, size=fbox, boxrule=1pt, pad at break*=1mm,colback=cellbackground, colframe=cellborder]
\prompt{In}{incolor}{101}{\boxspacing}
\begin{Verbatim}[commandchars=\\\{\}]
\PY{n}{help}\PY{p}{(}\PY{n+nb}{tuple}\PY{p}{)}
\end{Verbatim}
\end{tcolorbox}

    \begin{Verbatim}[commandchars=\\\{\}]
object <class 'tuple'> is of type type
  count -- <function>
  index -- <function>
    \end{Verbatim}

    \hypertarget{sets}{%
\subsubsection{Sets}\label{sets}}

\begin{itemize}
\tightlist
\item
  Geschweifte Klammern
\item
  Keine Dupplikate
\item
  Verschiedene Datentypen mischbar
\item
  Ungeordnet (Kein Index)
\item
  Nur Erweitern/Löschen, keine Verändern von Elementen
\end{itemize}

    \begin{tcolorbox}[breakable, size=fbox, boxrule=1pt, pad at break*=1mm,colback=cellbackground, colframe=cellborder]
\prompt{In}{incolor}{146}{\boxspacing}
\begin{Verbatim}[commandchars=\\\{\}]
\PY{c+c1}{\PYZsh{} Sets erstellen}

\PY{n}{meineDaten1} \PY{o}{=} \PY{p}{\PYZob{}}\PY{l+m+mi}{1}\PY{p}{,}\PY{l+m+mi}{2}\PY{p}{,}\PY{l+m+mi}{3}\PY{p}{\PYZcb{}}                  \PY{c+c1}{\PYZsh{} Hier geschweifte Klammern}
\PY{n}{meineDaten2} \PY{o}{=} \PY{n+nb}{set}\PY{p}{(}\PY{p}{(}\PY{l+s+s2}{\PYZdq{}}\PY{l+s+s2}{1}\PY{l+s+s2}{\PYZdq{}}\PY{p}{,}\PY{l+s+s2}{\PYZdq{}}\PY{l+s+s2}{2}\PY{l+s+s2}{\PYZdq{}}\PY{p}{,}\PY{l+s+s2}{\PYZdq{}}\PY{l+s+s2}{3}\PY{l+s+s2}{\PYZdq{}}\PY{p}{)}\PY{p}{)}       \PY{c+c1}{\PYZsh{} über den Set\PYZhy{}Construktor ((\PYZhy{}beachten}
\PY{n}{meineDaten3} \PY{o}{=} \PY{p}{\PYZob{}}\PY{l+s+s2}{\PYZdq{}}\PY{l+s+s2}{eins}\PY{l+s+s2}{\PYZdq{}}\PY{p}{,}\PY{l+s+s2}{\PYZdq{}}\PY{l+s+s2}{zwei}\PY{l+s+s2}{\PYZdq{}}\PY{p}{,}\PY{l+s+s2}{\PYZdq{}}\PY{l+s+s2}{drei}\PY{l+s+s2}{\PYZdq{}}\PY{p}{,}\PY{l+s+s2}{\PYZdq{}}\PY{l+s+s2}{vier}\PY{l+s+s2}{\PYZdq{}}\PY{p}{\PYZcb{}}
\PY{n}{meineDaten4} \PY{o}{=} \PY{p}{\PYZob{}}\PY{l+s+s2}{\PYZdq{}}\PY{l+s+s2}{eins}\PY{l+s+s2}{\PYZdq{}}\PY{p}{,}\PY{l+m+mi}{2}\PY{p}{,}\PY{l+s+s2}{\PYZdq{}}\PY{l+s+s2}{drei}\PY{l+s+s2}{\PYZdq{}}\PY{p}{,}\PY{k+kc}{True}\PY{p}{\PYZcb{}} \PY{c+c1}{\PYZsh{} Es müssen nicht mal die gleichen Typen verwendet werden}
\PY{n}{meineDaten5} \PY{o}{=} \PY{p}{\PYZob{}}\PY{l+m+mi}{1}\PY{p}{,}\PY{p}{\PYZcb{}}                   \PY{c+c1}{\PYZsh{} Tuple mit nur einem Element müssen ein Komma enthalten, sonst kein Tuple!}

\PY{n+nb}{print}\PY{p}{(}\PY{n}{meineDaten1}\PY{p}{)}
\PY{n+nb}{print}\PY{p}{(}\PY{n}{meineDaten2}\PY{p}{)}
\PY{n+nb}{print}\PY{p}{(}\PY{n}{meineDaten3}\PY{p}{)}
\PY{n+nb}{print}\PY{p}{(}\PY{n}{meineDaten4}\PY{p}{)}

\PY{c+c1}{\PYZsh{} Länge einer Liste (vgl. Strings)}
\PY{n+nb}{print}\PY{p}{(}\PY{n+nb}{len}\PY{p}{(}\PY{n}{meineDaten1}\PY{p}{)}\PY{p}{)}

\PY{c+c1}{\PYZsh{} Zugriff auf Elemente}
\PY{c+c1}{\PYZsh{} print(meineDaten4[0])    \PYZsh{} ! Kein Zugriff über Index (es gibt kein Index)}

\PY{c+c1}{\PYZsh{} Loop List}
\PY{k}{for} \PY{n}{x} \PY{o+ow}{in} \PY{n}{meineDaten1}\PY{p}{:}
    \PY{n+nb}{print}\PY{p}{(}\PY{n}{x}\PY{p}{)}
\PY{c+c1}{\PYZsh{} Abkürzung für Durchlauf}
\PY{p}{[}\PY{n+nb}{print}\PY{p}{(}\PY{n}{x}\PY{p}{)} \PY{k}{for} \PY{n}{x} \PY{o+ow}{in} \PY{n}{meineDaten3}\PY{p}{]}

\PY{n+nb}{print}\PY{p}{(}\PY{n+nb}{sum}\PY{p}{(}\PY{n}{meineDaten1}\PY{p}{)}\PY{p}{)} \PY{c+c1}{\PYZsh{} Summer einer gesamten numerischen Liste}
\end{Verbatim}
\end{tcolorbox}

    \begin{Verbatim}[commandchars=\\\{\}]
\{3, 1, 2\}
\{'1', '3', '2'\}
\{'eins', 'zwei', 'vier', 'drei'\}
\{'eins', True, 2, 'drei'\}
3
3
1
2
eins
zwei
vier
drei
6
    \end{Verbatim}

    \begin{tcolorbox}[breakable, size=fbox, boxrule=1pt, pad at break*=1mm,colback=cellbackground, colframe=cellborder]
\prompt{In}{incolor}{129}{\boxspacing}
\begin{Verbatim}[commandchars=\\\{\}]
\PY{c+c1}{\PYZsh{} Elemente finden/filtern}
\PY{n+nb}{print}\PY{p}{(}\PY{n}{meineDaten3}\PY{p}{)}

\PY{n+nb}{print}\PY{p}{(}\PY{l+s+s2}{\PYZdq{}}\PY{l+s+s2}{vier}\PY{l+s+s2}{\PYZdq{}} \PY{o+ow}{in} \PY{n}{meineDaten3}\PY{p}{)}
\PY{n+nb}{print}\PY{p}{(}\PY{l+s+s2}{\PYZdq{}}\PY{l+s+s2}{fünf}\PY{l+s+s2}{\PYZdq{}} \PY{o+ow}{not} \PY{o+ow}{in} \PY{n}{meineDaten3}\PY{p}{)}

\PY{n}{neueListe} \PY{o}{=} \PY{p}{[}\PY{p}{]}        \PY{c+c1}{\PYZsh{} ! Liste muss als solche initialisiert sein}
\PY{k}{for} \PY{n}{x} \PY{o+ow}{in} \PY{n}{meineDaten3}\PY{p}{:}
  \PY{k}{if} \PY{l+s+s2}{\PYZdq{}}\PY{l+s+s2}{r}\PY{l+s+s2}{\PYZdq{}} \PY{o+ow}{in} \PY{n}{x}\PY{p}{:}          \PY{c+c1}{\PYZsh{} Alle Elemente die ein r enthalten kommen in die neue Liste}
    \PY{n}{neueListe}\PY{o}{.}\PY{n}{append}\PY{p}{(}\PY{n}{x}\PY{p}{)}
\PY{n+nb}{print}\PY{p}{(}\PY{n}{neueListe}\PY{p}{)}

\PY{c+c1}{\PYZsh{}newlist = [expression for item in iterable if condition == True] }
\PY{n}{neueListe2} \PY{o}{=} \PY{p}{[}\PY{n}{x} \PY{k}{for} \PY{n}{x} \PY{o+ow}{in} \PY{n}{meineDaten3} \PY{k}{if} \PY{l+s+s2}{\PYZdq{}}\PY{l+s+s2}{r}\PY{l+s+s2}{\PYZdq{}} \PY{o+ow}{in} \PY{n}{x}\PY{p}{]}  \PY{c+c1}{\PYZsh{} Selbe Funktion in Kurzschreibweise}
\PY{n+nb}{print}\PY{p}{(}\PY{n}{neueListe2}\PY{p}{)}

\PY{n}{neueListe2} \PY{o}{=} \PY{p}{[}\PY{n}{x} \PY{k}{if} \PY{l+s+s2}{\PYZdq{}}\PY{l+s+s2}{r}\PY{l+s+s2}{\PYZdq{}} \PY{o+ow}{in} \PY{n}{x} \PY{k}{else} \PY{l+s+s2}{\PYZdq{}}\PY{l+s+s2}{\PYZhy{}}\PY{l+s+s2}{\PYZdq{}} \PY{k}{for} \PY{n}{x} \PY{o+ow}{in} \PY{n}{meineDaten3}\PY{p}{]}  \PY{c+c1}{\PYZsh{} Einfügen wenn \PYZsq{}r\PYZsq{} in Element sonst \PYZsq{}\PYZhy{}\PYZsq{} einfügen}
\PY{n+nb}{print}\PY{p}{(}\PY{n}{neueListe2}\PY{p}{)}

\PY{n}{neueListe3} \PY{o}{=}  \PY{p}{\PYZob{}}\PY{n}{x} \PY{k}{for} \PY{n}{x} \PY{o+ow}{in} \PY{n+nb}{range}\PY{p}{(}\PY{l+m+mi}{10}\PY{p}{)}\PY{p}{\PYZcb{}}  \PY{c+c1}{\PYZsh{} Neues Set von Elementen mit Range }
\PY{n+nb}{print}\PY{p}{(}\PY{n}{neueListe3}\PY{p}{)}
\end{Verbatim}
\end{tcolorbox}

    \begin{Verbatim}[commandchars=\\\{\}]
\{'eins', 'zwei', 'vier', 'drei'\}
True
True
['vier', 'drei']
['vier', 'drei']
['-', '-', 'vier', 'drei']
\{0, 1, 2, 3, 4, 5, 6, 7, 8, 9\}
    \end{Verbatim}

    \begin{tcolorbox}[breakable, size=fbox, boxrule=1pt, pad at break*=1mm,colback=cellbackground, colframe=cellborder]
\prompt{In}{incolor}{139}{\boxspacing}
\begin{Verbatim}[commandchars=\\\{\}]
\PY{c+c1}{\PYZsh{} Set verändern}

\PY{n+nb}{print}\PY{p}{(}\PY{n}{meineDaten2}\PY{p}{)}

\PY{n}{meineDaten2}\PY{o}{.}\PY{n}{add}\PY{p}{(}\PY{l+m+mi}{4}\PY{p}{)}                \PY{c+c1}{\PYZsh{} Element hinzufügen}
\PY{n}{meineDaten2}\PY{o}{.}\PY{n}{add}\PY{p}{(}\PY{l+m+mi}{2}\PY{p}{)}                \PY{c+c1}{\PYZsh{} Keine Dupplikate}
\PY{n+nb}{print}\PY{p}{(}\PY{n}{meineDaten2}\PY{p}{)}

\PY{c+c1}{\PYZsh{} Vereinigung}
\PY{n}{meineDaten2}\PY{o}{.}\PY{n}{update}\PY{p}{(}\PY{n}{meineDaten1}\PY{p}{)}   \PY{c+c1}{\PYZsh{} Ein komplettes Set hinzufügen, funktioniert auch mit Listen etc.}
\PY{n+nb}{print}\PY{p}{(}\PY{n}{meineDaten2}\PY{p}{)}

\PY{n}{neuesSet1} \PY{o}{=} \PY{n}{meineDaten2}\PY{o}{.}\PY{n}{union}\PY{p}{(}\PY{n}{meineDaten1}\PY{p}{)}  \PY{c+c1}{\PYZsh{} Vereint beide Sets (z.B. in einem neuen Set)}
\PY{n+nb}{print}\PY{p}{(}\PY{n}{neuesSet1}\PY{p}{)}

\PY{n}{neuesSet2} \PY{o}{=} \PY{n}{neuesSet1}\PY{o}{.}\PY{n}{union}\PY{p}{(}\PY{n}{meineDaten1}\PY{p}{)}  \PY{c+c1}{\PYZsh{} Keine Dupplikate}
\PY{n+nb}{print}\PY{p}{(}\PY{n}{neuesSet2}\PY{p}{)}
\end{Verbatim}
\end{tcolorbox}

    \begin{tcolorbox}[breakable, size=fbox, boxrule=1pt, pad at break*=1mm,colback=cellbackground, colframe=cellborder]
\prompt{In}{incolor}{140}{\boxspacing}
\begin{Verbatim}[commandchars=\\\{\}]
\PY{n+nb}{print}\PY{p}{(}\PY{n}{meineDaten3}\PY{p}{)}
\PY{n+nb}{print}\PY{p}{(}\PY{n}{meineDaten4}\PY{p}{)}

\PY{c+c1}{\PYZsh{} Schnittmente / Überschneidung }
\PY{n}{neuesSet2} \PY{o}{=} \PY{n}{meineDaten4}\PY{o}{.}\PY{n}{intersection}\PY{p}{(}\PY{n}{meineDaten3}\PY{p}{)}   \PY{c+c1}{\PYZsh{} Nur Elemente die in beiden Liste vorkommen}
\PY{n+nb}{print}\PY{p}{(}\PY{n}{neuesSet2}\PY{p}{)}

\PY{n}{neuesSet2}\PY{o}{.}\PY{n}{intersection\PYZus{}update}\PY{p}{(}\PY{n}{meineDaten3}\PY{p}{)}        \PY{c+c1}{\PYZsh{} Update: Nur Elemente die in beiden Liste vorkommen}
\PY{n+nb}{print}\PY{p}{(}\PY{n}{neuesSet2}\PY{p}{)}


\PY{c+c1}{\PYZsh{} Differenz }
\PY{n}{neuesSet2} \PY{o}{=} \PY{n}{meineDaten4}\PY{o}{.}\PY{n}{difference}\PY{p}{(}\PY{n}{meineDaten3}\PY{p}{)}   \PY{c+c1}{\PYZsh{} Nur Elemente die nicht in beiden Liste vorkommen}
\PY{n+nb}{print}\PY{p}{(}\PY{n}{neuesSet2}\PY{p}{)}

\PY{n}{neuesSet2}\PY{o}{.}\PY{n}{difference\PYZus{}update}\PY{p}{(}\PY{n}{meineDaten3}\PY{p}{)}        \PY{c+c1}{\PYZsh{} Update: Nur Elemente die nicht in beiden Liste vorkommen}
\PY{n+nb}{print}\PY{p}{(}\PY{n}{neuesSet2}\PY{p}{)}

\PY{c+c1}{\PYZsh{} Symetrische Differenz = Vereinigung bis auf dupplikate (XOR)}
\PY{n}{neuesSet2} \PY{o}{=} \PY{n}{meineDaten4}\PY{o}{.}\PY{n}{symmetric\PYZus{}difference}\PY{p}{(}\PY{n}{meineDaten3}\PY{p}{)}   \PY{c+c1}{\PYZsh{} Nur Elemente die in beiden Liste vorkommen}
\PY{n+nb}{print}\PY{p}{(}\PY{n}{neuesSet2}\PY{p}{)}

\PY{n}{neuesSet2}\PY{o}{.}\PY{n}{symmetric\PYZus{}difference\PYZus{}update}\PY{p}{(}\PY{n}{meineDaten4}\PY{p}{)}        \PY{c+c1}{\PYZsh{} Update: Nur Elemente die in beiden Liste vorkommen}
\PY{n+nb}{print}\PY{p}{(}\PY{n}{neuesSet2}\PY{p}{)}
\end{Verbatim}
\end{tcolorbox}

    \begin{tcolorbox}[breakable, size=fbox, boxrule=1pt, pad at break*=1mm,colback=cellbackground, colframe=cellborder]
\prompt{In}{incolor}{143}{\boxspacing}
\begin{Verbatim}[commandchars=\\\{\}]
\PY{c+c1}{\PYZsh{} Eigenschaften}
\PY{n}{schnittmenge} \PY{o}{=} \PY{n}{meineDaten4}\PY{o}{.}\PY{n}{intersection}\PY{p}{(}\PY{n}{meineDaten3}\PY{p}{)}   \PY{c+c1}{\PYZsh{} Nur Elemente die in beiden Liste vorkommen}

\PY{n+nb}{print}\PY{p}{(}\PY{n}{schnittmenge}\PY{o}{.}\PY{n}{issubset}\PY{p}{(}\PY{n}{meineDaten3}\PY{p}{)}\PY{p}{)}    \PY{c+c1}{\PYZsh{} Untermenge}
\PY{n+nb}{print}\PY{p}{(}\PY{n}{meineDaten3}\PY{o}{.}\PY{n}{issuperset}\PY{p}{(}\PY{n}{schnittmenge}\PY{p}{)}\PY{p}{)}  \PY{c+c1}{\PYZsh{} Obermenge}
\PY{n+nb}{print}\PY{p}{(}\PY{n}{schnittmenge}\PY{o}{.}\PY{n}{isdisjoint}\PY{p}{(}\PY{n}{meineDaten1}\PY{p}{)}\PY{p}{)}  \PY{c+c1}{\PYZsh{} Disjunkt}
\end{Verbatim}
\end{tcolorbox}

    \begin{Verbatim}[commandchars=\\\{\}]
True
True
True
    \end{Verbatim}

    \begin{tcolorbox}[breakable, size=fbox, boxrule=1pt, pad at break*=1mm,colback=cellbackground, colframe=cellborder]
\prompt{In}{incolor}{150}{\boxspacing}
\begin{Verbatim}[commandchars=\\\{\}]
\PY{n}{neuesSet} \PY{o}{=} \PY{n}{meineDaten2}\PY{o}{.}\PY{n}{copy}\PY{p}{(}\PY{p}{)}  \PY{c+c1}{\PYZsh{} Kopieren eines Sets}
\PY{n+nb}{print}\PY{p}{(}\PY{n}{neuesSet}\PY{p}{)}

\PY{n}{neuesSet}\PY{o}{.}\PY{n}{remove}\PY{p}{(}\PY{l+s+s1}{\PYZsq{}}\PY{l+s+s1}{1}\PY{l+s+s1}{\PYZsq{}}\PY{p}{)}    \PY{c+c1}{\PYZsh{} Entfernen von Element}
\PY{n+nb}{print}\PY{p}{(}\PY{n}{neuesSet}\PY{p}{)}

\PY{c+c1}{\PYZsh{} meineDaten2.remove(\PYZsq{}1\PYZsq{})  \PYZsh{} ! Wenn ELement nicht existiert \PYZhy{}\PYZgt{} ERROR}

\PY{n}{neuesSet}\PY{o}{.}\PY{n}{discard}\PY{p}{(}\PY{l+s+s1}{\PYZsq{}}\PY{l+s+s1}{1}\PY{l+s+s1}{\PYZsq{}}\PY{p}{)}     \PY{c+c1}{\PYZsh{} Discard entfernt auch, aber ohne Error falls Element nicht existiert}
\PY{n+nb}{print}\PY{p}{(}\PY{n}{neuesSet}\PY{p}{)}

\PY{n}{neuesSet}\PY{o}{.}\PY{n}{pop}\PY{p}{(}\PY{p}{)}            \PY{c+c1}{\PYZsh{} Entfernt letzes Element}
\PY{n+nb}{print}\PY{p}{(}\PY{n}{neuesSet}\PY{p}{)}

\PY{n}{neuesSet}\PY{o}{.}\PY{n}{clear}\PY{p}{(}\PY{p}{)}          \PY{c+c1}{\PYZsh{} Löscht Set\PYZhy{}Inhalt}
\PY{n+nb}{print}\PY{p}{(}\PY{n}{neuesSet}\PY{p}{)}
\end{Verbatim}
\end{tcolorbox}

    \begin{Verbatim}[commandchars=\\\{\}]
\{'1', '3', '2'\}
\{'3', '2'\}
\{'3', '2'\}
\{'2'\}
set()
    \end{Verbatim}

    \begin{tcolorbox}[breakable, size=fbox, boxrule=1pt, pad at break*=1mm,colback=cellbackground, colframe=cellborder]
\prompt{In}{incolor}{115}{\boxspacing}
\begin{Verbatim}[commandchars=\\\{\}]
\PY{n}{help}\PY{p}{(}\PY{n+nb}{set}\PY{p}{)}
\end{Verbatim}
\end{tcolorbox}

    \begin{Verbatim}[commandchars=\\\{\}]
object <class 'set'> is of type type
  add -- <function>
  clear -- <function>
  copy -- <function>
  discard -- <function>
  difference -- <function>
  difference\_update -- <function>
  intersection -- <function>
  intersection\_update -- <function>
  isdisjoint -- <function>
  issubset -- <function>
  issuperset -- <function>
  pop -- <function>
  remove -- <function>
  symmetric\_difference -- <function>
  symmetric\_difference\_update -- <function>
  union -- <function>
  update -- <function>
  \_\_contains\_\_ -- <function>
    \end{Verbatim}

    \hypertarget{dictionaries}{%
\subsubsection{Dictionaries}\label{dictionaries}}

\begin{itemize}
\tightlist
\item
  Geschweifte Klammern, Key:Value
\item
  Keine Dupplikate
\item
  Verschiedene Datentypen mischbar
\item
  Geordnet (Index)
\item
  Änderbar
\end{itemize}

    \begin{tcolorbox}[breakable, size=fbox, boxrule=1pt, pad at break*=1mm,colback=cellbackground, colframe=cellborder]
\prompt{In}{incolor}{192}{\boxspacing}
\begin{Verbatim}[commandchars=\\\{\}]
\PY{c+c1}{\PYZsh{} Dictionary erstellen}

\PY{c+c1}{\PYZsh{} Key:Value Paare}
\PY{n}{meineDaten1} \PY{o}{=} \PY{p}{\PYZob{}}\PY{l+m+mi}{1}\PY{p}{:}\PY{l+s+s2}{\PYZdq{}}\PY{l+s+s2}{eins}\PY{l+s+s2}{\PYZdq{}}\PY{p}{,}\PY{l+m+mi}{2}\PY{p}{:}\PY{l+s+s2}{\PYZdq{}}\PY{l+s+s2}{zwei}\PY{l+s+s2}{\PYZdq{}}\PY{p}{,}\PY{l+m+mi}{3}\PY{p}{:}\PY{l+s+s2}{\PYZdq{}}\PY{l+s+s2}{drei}\PY{l+s+s2}{\PYZdq{}}\PY{p}{,}\PY{l+m+mi}{4}\PY{p}{:}\PY{l+s+s2}{\PYZdq{}}\PY{l+s+s2}{vier}\PY{l+s+s2}{\PYZdq{}}\PY{p}{\PYZcb{}}
\PY{c+c1}{\PYZsh{} Doppelter Key \PYZhy{}\PYZgt{} Überschrieben. Auch z.B. lists sind values}
\PY{n}{meineDaten2} \PY{o}{=} \PY{p}{\PYZob{}}\PY{l+s+s2}{\PYZdq{}}\PY{l+s+s2}{vorname}\PY{l+s+s2}{\PYZdq{}}\PY{p}{:}\PY{l+s+s2}{\PYZdq{}}\PY{l+s+s2}{Donald}\PY{l+s+s2}{\PYZdq{}}\PY{p}{,}\PY{l+s+s2}{\PYZdq{}}\PY{l+s+s2}{nachname}\PY{l+s+s2}{\PYZdq{}}\PY{p}{:}\PY{l+s+s2}{\PYZdq{}}\PY{l+s+s2}{Duck}\PY{l+s+s2}{\PYZdq{}}\PY{p}{,}\PY{l+s+s2}{\PYZdq{}}\PY{l+s+s2}{alter}\PY{l+s+s2}{\PYZdq{}}\PY{p}{:}\PY{l+m+mi}{56}\PY{p}{,}\PY{l+s+s2}{\PYZdq{}}\PY{l+s+s2}{vorname}\PY{l+s+s2}{\PYZdq{}}\PY{p}{:}\PY{l+s+s2}{\PYZdq{}}\PY{l+s+s2}{Dagobert}\PY{l+s+s2}{\PYZdq{}}\PY{p}{,} \PY{l+s+s2}{\PYZdq{}}\PY{l+s+s2}{enkel}\PY{l+s+s2}{\PYZdq{}}\PY{p}{:}\PY{p}{[}\PY{l+s+s2}{\PYZdq{}}\PY{l+s+s2}{Tick}\PY{l+s+s2}{\PYZdq{}}\PY{p}{,}\PY{l+s+s2}{\PYZdq{}}\PY{l+s+s2}{Trick}\PY{l+s+s2}{\PYZdq{}}\PY{p}{,}\PY{l+s+s2}{\PYZdq{}}\PY{l+s+s2}{Track}\PY{l+s+s2}{\PYZdq{}}\PY{p}{]}\PY{p}{\PYZcb{}}  

\PY{n}{x} \PY{o}{=} \PY{p}{(}\PY{l+s+s2}{\PYZdq{}}\PY{l+s+s2}{1}\PY{l+s+s2}{\PYZdq{}}\PY{p}{,}\PY{l+s+s2}{\PYZdq{}}\PY{l+s+s2}{2}\PY{l+s+s2}{\PYZdq{}}\PY{p}{,}\PY{l+s+s2}{\PYZdq{}}\PY{l+s+s2}{3}\PY{l+s+s2}{\PYZdq{}}\PY{p}{)} 
\PY{n}{y} \PY{o}{=} \PY{p}{[}\PY{l+s+s2}{\PYZdq{}}\PY{l+s+s2}{1}\PY{l+s+s2}{\PYZdq{}}\PY{p}{,}\PY{l+s+s2}{\PYZdq{}}\PY{l+s+s2}{2}\PY{l+s+s2}{\PYZdq{}}\PY{p}{,}\PY{l+s+s2}{\PYZdq{}}\PY{l+s+s2}{3}\PY{l+s+s2}{\PYZdq{}}\PY{p}{]}
\PY{n}{meineDaten3} \PY{o}{=} \PY{n+nb}{dict}\PY{o}{.}\PY{n}{fromkeys}\PY{p}{(}\PY{n}{x}\PY{p}{,}\PY{l+m+mi}{0}\PY{p}{)}    \PY{c+c1}{\PYZsh{} Erstellt eine Dict aus einem vorgebenen Datensatz (hier set), Value für alle 0}
\PY{n}{meineDaten4} \PY{o}{=} \PY{n+nb}{dict}\PY{o}{.}\PY{n}{fromkeys}\PY{p}{(}\PY{n}{y}\PY{p}{)}      \PY{c+c1}{\PYZsh{} hier aus Liste erstellt. Value nicht angegeben \PYZhy{}\PYZgt{} None}

\PY{n+nb}{print}\PY{p}{(}\PY{n}{meineDaten1}\PY{p}{)}
\PY{n+nb}{print}\PY{p}{(}\PY{n}{meineDaten2}\PY{p}{)}
\PY{n+nb}{print}\PY{p}{(}\PY{n}{meineDaten3}\PY{p}{)}
\PY{n+nb}{print}\PY{p}{(}\PY{n}{meineDaten4}\PY{p}{)}

\PY{c+c1}{\PYZsh{} Länge des Dictionaries}
\PY{n+nb}{print}\PY{p}{(}\PY{n+nb}{len}\PY{p}{(}\PY{n}{meineDaten1}\PY{p}{)}\PY{p}{)}
\PY{n+nb}{print}\PY{p}{(}\PY{n+nb}{len}\PY{p}{(}\PY{n}{meineDaten4}\PY{p}{)}\PY{p}{)}
\end{Verbatim}
\end{tcolorbox}

    \begin{Verbatim}[commandchars=\\\{\}]
\{4: 'vier', 1: 'eins', 2: 'zwei', 3: 'drei'\}
\{'nachname': 'Duck', 'alter': 56, 'enkel': ['Tick', 'Trick', 'Track'],
'vorname': 'Dagobert'\}
\{'3': 0, '1': 0, '2': 0\}
\{'3': None, '1': None, '2': None\}
4
3
    \end{Verbatim}

    \begin{tcolorbox}[breakable, size=fbox, boxrule=1pt, pad at break*=1mm,colback=cellbackground, colframe=cellborder]
\prompt{In}{incolor}{196}{\boxspacing}
\begin{Verbatim}[commandchars=\\\{\}]
\PY{c+c1}{\PYZsh{} Zugriff auf Elemente}
\PY{n+nb}{print}\PY{p}{(}\PY{n}{meineDaten1}\PY{p}{[}\PY{l+m+mi}{1}\PY{p}{]}\PY{p}{)}                \PY{c+c1}{\PYZsh{} ! Kein Zugriff über Index, hier über key die eine Zahl ist}
\PY{n+nb}{print}\PY{p}{(}\PY{n}{meineDaten2}\PY{p}{[}\PY{l+s+s2}{\PYZdq{}}\PY{l+s+s2}{vorname}\PY{l+s+s2}{\PYZdq{}}\PY{p}{]}\PY{p}{)}        \PY{c+c1}{\PYZsh{} Zugriff über key}
\PY{n+nb}{print}\PY{p}{(}\PY{n}{meineDaten2}\PY{o}{.}\PY{n}{get}\PY{p}{(}\PY{l+s+s2}{\PYZdq{}}\PY{l+s+s2}{vorname}\PY{l+s+s2}{\PYZdq{}}\PY{p}{)}\PY{p}{)}    \PY{c+c1}{\PYZsh{} Zugriff über key und Funktion get()}

\PY{n+nb}{print}\PY{p}{(}\PY{n}{meineDaten2}\PY{o}{.}\PY{n}{get}\PY{p}{(}\PY{l+s+s2}{\PYZdq{}}\PY{l+s+s2}{zweitname}\PY{l+s+s2}{\PYZdq{}}\PY{p}{)}\PY{p}{)}   \PY{c+c1}{\PYZsh{} None wenn Key nicht exisitert}
\PY{n+nb}{print}\PY{p}{(}\PY{n}{meineDaten2}\PY{o}{.}\PY{n}{setdefault}\PY{p}{(}\PY{l+s+s2}{\PYZdq{}}\PY{l+s+s2}{ehepartner}\PY{l+s+s2}{\PYZdq{}}\PY{p}{)}\PY{p}{)} \PY{c+c1}{\PYZsh{} Setdefault alternative zu get}
\PY{n+nb}{print}\PY{p}{(}\PY{n}{meineDaten2}\PY{o}{.}\PY{n}{setdefault}\PY{p}{(}\PY{l+s+s2}{\PYZdq{}}\PY{l+s+s2}{familie}\PY{l+s+s2}{\PYZdq{}}\PY{p}{,}\PY{l+s+s2}{\PYZdq{}}\PY{l+s+s2}{Duck}\PY{l+s+s2}{\PYZdq{}}\PY{p}{)}\PY{p}{)} \PY{c+c1}{\PYZsh{} Bei setdefault kann Standardwert angegben werden}

\PY{n+nb}{print}\PY{p}{(}\PY{n}{meineDaten2}\PY{o}{.}\PY{n}{keys}\PY{p}{(}\PY{p}{)}\PY{p}{)}           \PY{c+c1}{\PYZsh{} Eine Liste von keys}

\PY{k}{for} \PY{n}{x} \PY{o+ow}{in} \PY{n}{meineDaten2}\PY{p}{:}   \PY{c+c1}{\PYZsh{} Durchläuft Keys}
    \PY{n+nb}{print}\PY{p}{(}\PY{n}{x}\PY{p}{)}
    
\PY{k}{for} \PY{n}{x} \PY{o+ow}{in} \PY{n}{meineDaten2}\PY{o}{.}\PY{n}{keys}\PY{p}{(}\PY{p}{)}\PY{p}{:}
    \PY{n+nb}{print}\PY{p}{(}\PY{n}{x}\PY{p}{)}
    
\PY{p}{[}\PY{n+nb}{print}\PY{p}{(}\PY{n}{x}\PY{p}{)} \PY{k}{for} \PY{n}{x} \PY{o+ow}{in} \PY{n}{meineDaten2}\PY{o}{.}\PY{n}{values}\PY{p}{(}\PY{p}{)}\PY{p}{]}   \PY{c+c1}{\PYZsh{} Eine Liste von Values}

\PY{n+nb}{print}\PY{p}{(}\PY{n}{meineDaten2}\PY{o}{.}\PY{n}{items}\PY{p}{(}\PY{p}{)}\PY{p}{)}
\end{Verbatim}
\end{tcolorbox}

    \begin{Verbatim}[commandchars=\\\{\}]
eins
Dagobert
Dagobert
None
None
Duck
dict\_keys(['nachname', 'vorname', 'familie', 'ehepartner', 'alter', 'enkel'])
nachname
vorname
familie
ehepartner
alter
enkel
nachname
vorname
familie
ehepartner
alter
enkel
Duck
Dagobert
Duck
None
56
['Tick', 'Trick', 'Track']
dict\_items([('nachname', 'Duck'), ('vorname', 'Dagobert'), ('familie', 'Duck'),
('ehepartner', None), ('alter', 56), ('enkel', ['Tick', 'Trick', 'Track'])])
    \end{Verbatim}

    \begin{tcolorbox}[breakable, size=fbox, boxrule=1pt, pad at break*=1mm,colback=cellbackground, colframe=cellborder]
\prompt{In}{incolor}{178}{\boxspacing}
\begin{Verbatim}[commandchars=\\\{\}]
\PY{c+c1}{\PYZsh{} Daten verändern/hinzufügen}

\PY{c+c1}{\PYZsh{} Werte Verändern}
\PY{n}{meineDaten1}\PY{p}{[}\PY{l+m+mi}{1}\PY{p}{]} \PY{o}{=} \PY{l+s+s2}{\PYZdq{}}\PY{l+s+s2}{null}\PY{l+s+s2}{\PYZdq{}}
\PY{n+nb}{print}\PY{p}{(}\PY{n}{meineDaten1}\PY{p}{)}
\PY{n}{meineDaten1}\PY{o}{.}\PY{n}{update}\PY{p}{(}\PY{p}{\PYZob{}}\PY{l+m+mi}{2}\PY{p}{:}\PY{l+s+s2}{\PYZdq{}}\PY{l+s+s2}{eins}\PY{l+s+s2}{\PYZdq{}}\PY{p}{\PYZcb{}}\PY{p}{)}
\PY{n+nb}{print}\PY{p}{(}\PY{n}{meineDaten1}\PY{p}{)}

\PY{c+c1}{\PYZsh{} Key\PYZhy{}Value\PYZhy{}Paar hinzufügen}
\PY{n}{meineDaten1}\PY{p}{[}\PY{l+m+mi}{5}\PY{p}{]} \PY{o}{=} \PY{l+s+s2}{\PYZdq{}}\PY{l+s+s2}{vier}\PY{l+s+s2}{\PYZdq{}}
\PY{n+nb}{print}\PY{p}{(}\PY{n}{meineDaten1}\PY{p}{)}
\PY{n}{meineDaten1}\PY{o}{.}\PY{n}{update}\PY{p}{(}\PY{p}{\PYZob{}}\PY{l+m+mi}{6}\PY{p}{:}\PY{l+s+s2}{\PYZdq{}}\PY{l+s+s2}{sechs}\PY{l+s+s2}{\PYZdq{}}\PY{p}{\PYZcb{}}\PY{p}{)}
\PY{n+nb}{print}\PY{p}{(}\PY{n}{meineDaten1}\PY{p}{)}

\PY{n}{neuesDict} \PY{o}{=} \PY{n}{meineDaten1}\PY{o}{.}\PY{n}{copy}\PY{p}{(}\PY{p}{)}
\PY{n+nb}{print}\PY{p}{(}\PY{n}{neuesDict}\PY{p}{)}
\end{Verbatim}
\end{tcolorbox}

    \begin{Verbatim}[commandchars=\\\{\}]
\{6: 'sechs', 1: 'null', 2: 'eins', 3: 'drei', 4: 'vier', 5: 'vier'\}
\{6: 'sechs', 1: 'null', 2: 'eins', 3: 'drei', 4: 'vier', 5: 'vier'\}
\{6: 'sechs', 1: 'null', 2: 'eins', 3: 'drei', 4: 'vier', 5: 'vier'\}
\{6: 'sechs', 1: 'null', 2: 'eins', 3: 'drei', 4: 'vier', 5: 'vier'\}
\{6: 'sechs', 1: 'null', 2: 'eins', 3: 'drei', 4: 'vier', 5: 'vier'\}
    \end{Verbatim}

    \begin{tcolorbox}[breakable, size=fbox, boxrule=1pt, pad at break*=1mm,colback=cellbackground, colframe=cellborder]
\prompt{In}{incolor}{175}{\boxspacing}
\begin{Verbatim}[commandchars=\\\{\}]
\PY{c+c1}{\PYZsh{} Daten löschen}
\PY{n+nb}{print}\PY{p}{(}\PY{n}{meineDaten1}\PY{p}{)}

\PY{n}{meineDaten1}\PY{o}{.}\PY{n}{popitem}\PY{p}{(}\PY{p}{)}   \PY{c+c1}{\PYZsh{} Entfernt zuletzt hinzugefügtes item}
\PY{n+nb}{print}\PY{p}{(}\PY{n}{meineDaten1}\PY{p}{)}

\PY{n}{meineDaten1}\PY{o}{.}\PY{n}{pop}\PY{p}{(}\PY{l+m+mi}{1}\PY{p}{)}      \PY{c+c1}{\PYZsh{} Entfernt Key\PYZhy{}Value Paar mit diesem Key}
\PY{n+nb}{print}\PY{p}{(}\PY{n}{meineDaten1}\PY{p}{)}

\PY{k}{del} \PY{n}{meineDaten1}\PY{p}{[}\PY{l+m+mi}{2}\PY{p}{]}      \PY{c+c1}{\PYZsh{} Entfernt Key\PYZhy{}Value Paar mit diesem Key (del Methode)}
\PY{n+nb}{print}\PY{p}{(}\PY{n}{meineDaten1}\PY{p}{)}

\PY{n}{meineDaten1}\PY{o}{.}\PY{n}{clear}\PY{p}{(}\PY{p}{)}     \PY{c+c1}{\PYZsh{} Löscht Dictionary Inhalt}
\PY{n+nb}{print}\PY{p}{(}\PY{n}{meineDaten1}\PY{p}{)}
\end{Verbatim}
\end{tcolorbox}

    \begin{Verbatim}[commandchars=\\\{\}]
\{6: 'sechs', 1: 'null', 2: 'eins', 5: 'vier'\}
\{1: 'null', 2: 'eins', 5: 'vier'\}
\{2: 'eins', 5: 'vier'\}
\{5: 'vier'\}
\{\}
    \end{Verbatim}

    \begin{tcolorbox}[breakable, size=fbox, boxrule=1pt, pad at break*=1mm,colback=cellbackground, colframe=cellborder]
\prompt{In}{incolor}{154}{\boxspacing}
\begin{Verbatim}[commandchars=\\\{\}]
\PY{n}{help}\PY{p}{(}\PY{n+nb}{dict}\PY{p}{)}
\end{Verbatim}
\end{tcolorbox}

    \begin{Verbatim}[commandchars=\\\{\}]
object <class 'dict'> is of type type
  clear -- <function>
  copy -- <function>
  fromkeys -- <classmethod>
  get -- <function>
  items -- <function>
  keys -- <function>
  pop -- <function>
  popitem -- <function>
  setdefault -- <function>
  update -- <function>
  values -- <function>
  \_\_getitem\_\_ -- <function>
  \_\_setitem\_\_ -- <function>
  \_\_delitem\_\_ -- <function>
    \end{Verbatim}

    \hypertarget{klassen-und-objekte}{%
\subsection{Klassen und Objekte}\label{klassen-und-objekte}}

    \begin{tcolorbox}[breakable, size=fbox, boxrule=1pt, pad at break*=1mm,colback=cellbackground, colframe=cellborder]
\prompt{In}{incolor}{199}{\boxspacing}
\begin{Verbatim}[commandchars=\\\{\}]

\end{Verbatim}
\end{tcolorbox}

    \begin{tcolorbox}[breakable, size=fbox, boxrule=1pt, pad at break*=1mm,colback=cellbackground, colframe=cellborder]
\prompt{In}{incolor}{199}{\boxspacing}
\begin{Verbatim}[commandchars=\\\{\}]

\end{Verbatim}
\end{tcolorbox}

    \begin{Verbatim}[commandchars=\\\{\}]
<class 'function'>
Meine Funktion
    \end{Verbatim}

    \hypertarget{kontrollstrukturen}{%
\section{Kontrollstrukturen}\label{kontrollstrukturen}}

    \hypertarget{if-then-else}{%
\subsection{If-Then-Else}\label{if-then-else}}

Es existiert (derzeit) kein switch-case in MicroPyhton!

(Erscheint in Python 3.10:
https://towardsdatascience.com/switch-case-statements-are-coming-to-python-d0caf7b2bfd3)

    \begin{tcolorbox}[breakable, size=fbox, boxrule=1pt, pad at break*=1mm,colback=cellbackground, colframe=cellborder]
\prompt{In}{incolor}{95}{\boxspacing}
\begin{Verbatim}[commandchars=\\\{\}]
\PY{c+c1}{\PYZsh{} klassiche IF\PYZhy{}Anweisung}
\PY{n}{x}\PY{o}{=}\PY{l+m+mi}{3}\PY{p}{;} \PY{n}{y}\PY{o}{=}\PY{l+m+mi}{5}

\PY{k}{if} \PY{n}{x} \PY{o}{\PYZlt{}} \PY{n}{y}\PY{p}{:}
    \PY{n+nb}{print}\PY{p}{(}\PY{n+nb}{str}\PY{p}{(}\PY{n}{x}\PY{p}{)}\PY{o}{+} \PY{l+s+s2}{\PYZdq{}}\PY{l+s+s2}{ ist kleiner als }\PY{l+s+s2}{\PYZdq{}} \PY{o}{+}\PY{n+nb}{str}\PY{p}{(}\PY{n}{y}\PY{p}{)}\PY{p}{)}
\PY{k}{elif} \PY{n}{x} \PY{o}{==} \PY{n}{y}\PY{p}{:}
    \PY{n+nb}{print}\PY{p}{(}\PY{n+nb}{str}\PY{p}{(}\PY{n}{x}\PY{p}{)}\PY{o}{+} \PY{l+s+s2}{\PYZdq{}}\PY{l+s+s2}{ ist gleich }\PY{l+s+s2}{\PYZdq{}} \PY{o}{+}\PY{n+nb}{str}\PY{p}{(}\PY{n}{y}\PY{p}{)}\PY{p}{)}
\PY{k}{else}\PY{p}{:}
    \PY{n+nb}{print}\PY{p}{(}\PY{n+nb}{str}\PY{p}{(}\PY{n}{x}\PY{p}{)}\PY{o}{+} \PY{l+s+s2}{\PYZdq{}}\PY{l+s+s2}{ ist größer als }\PY{l+s+s2}{\PYZdq{}} \PY{o}{+}\PY{n+nb}{str}\PY{p}{(}\PY{n}{y}\PY{p}{)}\PY{p}{)}
\end{Verbatim}
\end{tcolorbox}

    \begin{Verbatim}[commandchars=\\\{\}]
3 ist kleiner als 5
    \end{Verbatim}

    \begin{tcolorbox}[breakable, size=fbox, boxrule=1pt, pad at break*=1mm,colback=cellbackground, colframe=cellborder]
\prompt{In}{incolor}{99}{\boxspacing}
\begin{Verbatim}[commandchars=\\\{\}]
\PY{c+c1}{\PYZsh{} Kurzschreibweisen}
\PY{n}{a} \PY{o}{=} \PY{l+m+mi}{5}\PY{p}{;} \PY{n}{b} \PY{o}{=} \PY{l+m+mi}{3}

\PY{k}{if} \PY{n}{a} \PY{o}{\PYZgt{}} \PY{n}{b}\PY{p}{:} \PY{n+nb}{print}\PY{p}{(}\PY{l+s+s2}{\PYZdq{}}\PY{l+s+s2}{a ist größer als b}\PY{l+s+s2}{\PYZdq{}}\PY{p}{)}
\PY{n+nb}{print}\PY{p}{(}\PY{l+s+s2}{\PYZdq{}}\PY{l+s+s2}{A}\PY{l+s+s2}{\PYZdq{}}\PY{p}{)} \PY{k}{if} \PY{n}{a} \PY{o}{\PYZgt{}} \PY{n}{b} \PY{k}{else} \PY{n+nb}{print}\PY{p}{(}\PY{l+s+s2}{\PYZdq{}}\PY{l+s+s2}{B}\PY{l+s+s2}{\PYZdq{}}\PY{p}{)}
\end{Verbatim}
\end{tcolorbox}

    \begin{Verbatim}[commandchars=\\\{\}]
a ist größer als b
A
    \end{Verbatim}

    \begin{tcolorbox}[breakable, size=fbox, boxrule=1pt, pad at break*=1mm,colback=cellbackground, colframe=cellborder]
\prompt{In}{incolor}{103}{\boxspacing}
\begin{Verbatim}[commandchars=\\\{\}]
\PY{c+c1}{\PYZsh{} AND und OR}
\PY{n}{a} \PY{o}{=} \PY{l+m+mi}{3}\PY{p}{;} \PY{n}{b} \PY{o}{=} \PY{l+m+mi}{4}\PY{p}{;} \PY{n}{c} \PY{o}{=} \PY{l+m+mi}{7}

\PY{k}{if} \PY{n}{b} \PY{o}{\PYZgt{}} \PY{n}{a} \PY{o+ow}{and} \PY{n}{b} \PY{o}{\PYZlt{}} \PY{n}{c}\PY{p}{:} 
    \PY{n+nb}{print}\PY{p}{(}\PY{l+s+s2}{\PYZdq{}}\PY{l+s+s2}{b liegt zwischen a und c}\PY{l+s+s2}{\PYZdq{}}\PY{p}{)}

\PY{k}{if} \PY{n}{b} \PY{o}{\PYZlt{}} \PY{n}{a} \PY{o+ow}{or} \PY{n}{b} \PY{o}{\PYZgt{}} \PY{n}{c}\PY{p}{:} 
    \PY{n+nb}{print}\PY{p}{(}\PY{l+s+s2}{\PYZdq{}}\PY{l+s+s2}{b liegt nicht zwischen a und c}\PY{l+s+s2}{\PYZdq{}}\PY{p}{)}
\end{Verbatim}
\end{tcolorbox}

    \begin{Verbatim}[commandchars=\\\{\}]
b liegt zwischen a und c
    \end{Verbatim}

    \hypertarget{while-loop}{%
\subsection{While-Loop}\label{while-loop}}

    \begin{tcolorbox}[breakable, size=fbox, boxrule=1pt, pad at break*=1mm,colback=cellbackground, colframe=cellborder]
\prompt{In}{incolor}{128}{\boxspacing}
\begin{Verbatim}[commandchars=\\\{\}]
\PY{n}{x} \PY{o}{=} \PY{l+m+mi}{1} 

\PY{c+c1}{\PYZsh{} Einfache While\PYZhy{}Loop}
\PY{k}{while} \PY{n}{x} \PY{o}{\PYZlt{}}\PY{o}{=} \PY{l+m+mi}{3}\PY{p}{:}
    \PY{n+nb}{print}\PY{p}{(}\PY{n}{x}\PY{p}{)}
    \PY{n}{x} \PY{o}{+}\PY{o}{=} \PY{l+m+mi}{1}
\end{Verbatim}
\end{tcolorbox}

    \begin{Verbatim}[commandchars=\\\{\}]
1
2
3
    \end{Verbatim}

    \begin{tcolorbox}[breakable, size=fbox, boxrule=1pt, pad at break*=1mm,colback=cellbackground, colframe=cellborder]
\prompt{In}{incolor}{130}{\boxspacing}
\begin{Verbatim}[commandchars=\\\{\}]
\PY{c+c1}{\PYZsh{} While mit Abbruchbedingung:}
\PY{k}{while} \PY{n}{x} \PY{o}{\PYZgt{}} \PY{l+m+mi}{0}\PY{p}{:}
    \PY{n+nb}{print}\PY{p}{(}\PY{n}{x}\PY{p}{)}
    \PY{n}{x} \PY{o}{+}\PY{o}{=} \PY{l+m+mi}{1}
    \PY{k}{if} \PY{n}{x}\PY{o}{\PYZgt{}}\PY{l+m+mi}{5}\PY{p}{:} \PY{k}{break}
\end{Verbatim}
\end{tcolorbox}

    \begin{Verbatim}[commandchars=\\\{\}]
4
5
    \end{Verbatim}

    \begin{tcolorbox}[breakable, size=fbox, boxrule=1pt, pad at break*=1mm,colback=cellbackground, colframe=cellborder]
\prompt{In}{incolor}{131}{\boxspacing}
\begin{Verbatim}[commandchars=\\\{\}]
\PY{c+c1}{\PYZsh{} While mit continue}
\PY{k}{while} \PY{n}{x} \PY{o}{\PYZlt{}} \PY{l+m+mi}{10}\PY{p}{:}
    \PY{n}{x} \PY{o}{+}\PY{o}{=} \PY{l+m+mi}{1}
    \PY{k}{if} \PY{n}{x} \PY{o}{==} \PY{l+m+mi}{7}\PY{p}{:} \PY{k}{continue}
    \PY{n+nb}{print}\PY{p}{(}\PY{n}{x}\PY{p}{)}
\end{Verbatim}
\end{tcolorbox}

    \begin{Verbatim}[commandchars=\\\{\}]
8
9
10
    \end{Verbatim}

    \begin{tcolorbox}[breakable, size=fbox, boxrule=1pt, pad at break*=1mm,colback=cellbackground, colframe=cellborder]
\prompt{In}{incolor}{132}{\boxspacing}
\begin{Verbatim}[commandchars=\\\{\}]
\PY{c+c1}{\PYZsh{} While mit Else}
\PY{k}{while} \PY{n}{x} \PY{o}{\PYZlt{}} \PY{l+m+mi}{13}\PY{p}{:}
    \PY{n}{x} \PY{o}{+}\PY{o}{=} \PY{l+m+mi}{1}
    \PY{n+nb}{print}\PY{p}{(}\PY{n}{x}\PY{p}{)}
\PY{k}{else}\PY{p}{:}                    \PY{c+c1}{\PYZsh{} Die Else\PYZhy{}Anweisung wird ausfeführt, wenn die Schleifenbedingung nicht länger True ist}
    \PY{n+nb}{print}\PY{p}{(}\PY{l+s+s2}{\PYZdq{}}\PY{l+s+s2}{Die 13 wurde erreicht}\PY{l+s+s2}{\PYZdq{}}\PY{p}{)}
\end{Verbatim}
\end{tcolorbox}

    \begin{Verbatim}[commandchars=\\\{\}]
11
12
13
Die 13 wurde erreicht
    \end{Verbatim}

    \hypertarget{for-loop}{%
\subsection{For-Loop}\label{for-loop}}

    \begin{tcolorbox}[breakable, size=fbox, boxrule=1pt, pad at break*=1mm,colback=cellbackground, colframe=cellborder]
\prompt{In}{incolor}{134}{\boxspacing}
\begin{Verbatim}[commandchars=\\\{\}]
\PY{c+c1}{\PYZsh{} For\PYZhy{}Loop zum durchlauf eines Zahlenbereichs}
\PY{k}{for} \PY{n}{x} \PY{o+ow}{in} \PY{n+nb}{range}\PY{p}{(}\PY{l+m+mi}{5}\PY{p}{)}\PY{p}{:}
    \PY{n+nb}{print}\PY{p}{(}\PY{n}{x}\PY{p}{)}
\end{Verbatim}
\end{tcolorbox}

    \begin{Verbatim}[commandchars=\\\{\}]
0
1
2
3
4
    \end{Verbatim}

    \begin{tcolorbox}[breakable, size=fbox, boxrule=1pt, pad at break*=1mm,colback=cellbackground, colframe=cellborder]
\prompt{In}{incolor}{135}{\boxspacing}
\begin{Verbatim}[commandchars=\\\{\}]
\PY{c+c1}{\PYZsh{} range() erzeugt einen sequenz von nummern die durchlaufen wird. }
\PY{c+c1}{\PYZsh{} range(start, stop, step)}
\PY{k}{for} \PY{n}{x} \PY{o+ow}{in} \PY{n+nb}{range}\PY{p}{(}\PY{l+m+mi}{0}\PY{p}{,}\PY{l+m+mi}{6}\PY{p}{,}\PY{l+m+mi}{2}\PY{p}{)}\PY{p}{:}
    \PY{n+nb}{print}\PY{p}{(}\PY{n}{x}\PY{p}{)}
\end{Verbatim}
\end{tcolorbox}

    \begin{Verbatim}[commandchars=\\\{\}]
0
2
4
    \end{Verbatim}

    \begin{tcolorbox}[breakable, size=fbox, boxrule=1pt, pad at break*=1mm,colback=cellbackground, colframe=cellborder]
\prompt{In}{incolor}{137}{\boxspacing}
\begin{Verbatim}[commandchars=\\\{\}]
\PY{c+c1}{\PYZsh{} For\PYZhy{}Loop um Listen zu durchlaufen}
\PY{n}{languages} \PY{o}{=} \PY{p}{[}\PY{l+s+s2}{\PYZdq{}}\PY{l+s+s2}{c}\PY{l+s+s2}{\PYZdq{}}\PY{p}{,} \PY{l+s+s2}{\PYZdq{}}\PY{l+s+s2}{c++}\PY{l+s+s2}{\PYZdq{}}\PY{p}{,} \PY{l+s+s2}{\PYZdq{}}\PY{l+s+s2}{python}\PY{l+s+s2}{\PYZdq{}}\PY{p}{]}
\PY{k}{for} \PY{n}{x} \PY{o+ow}{in} \PY{n}{languages}\PY{p}{:}
    \PY{n+nb}{print}\PY{p}{(}\PY{n}{x}\PY{p}{)}
\end{Verbatim}
\end{tcolorbox}

    \begin{Verbatim}[commandchars=\\\{\}]
c
c++
python
    \end{Verbatim}

    \begin{tcolorbox}[breakable, size=fbox, boxrule=1pt, pad at break*=1mm,colback=cellbackground, colframe=cellborder]
\prompt{In}{incolor}{139}{\boxspacing}
\begin{Verbatim}[commandchars=\\\{\}]
\PY{c+c1}{\PYZsh{} For\PYZhy{}Loop um Zeichen eines Strings zu durchlaufen}
\PY{n}{language} \PY{o}{=} \PY{l+s+s2}{\PYZdq{}}\PY{l+s+s2}{python}\PY{l+s+s2}{\PYZdq{}}
\PY{k}{for} \PY{n}{x} \PY{o+ow}{in} \PY{n}{language}\PY{p}{:}
    \PY{n+nb}{print}\PY{p}{(}\PY{n}{x}\PY{p}{)}
\end{Verbatim}
\end{tcolorbox}

    \begin{Verbatim}[commandchars=\\\{\}]
p
y
t
h
o
n
    \end{Verbatim}

    \begin{tcolorbox}[breakable, size=fbox, boxrule=1pt, pad at break*=1mm,colback=cellbackground, colframe=cellborder]
\prompt{In}{incolor}{207}{\boxspacing}
\begin{Verbatim}[commandchars=\\\{\}]
\PY{c+c1}{\PYZsh{} Break, Continue, Else funktionieren auch hier}

\PY{n}{language} \PY{o}{=} \PY{l+s+s2}{\PYZdq{}}\PY{l+s+s2}{python}\PY{l+s+s2}{\PYZdq{}}
\PY{k}{for} \PY{n}{x} \PY{o+ow}{in} \PY{n}{language}\PY{p}{:}
    \PY{k}{if} \PY{n}{x} \PY{o}{==} \PY{l+s+s1}{\PYZsq{}}\PY{l+s+s1}{y}\PY{l+s+s1}{\PYZsq{}}\PY{p}{:} \PY{k}{continue}
    \PY{k}{if} \PY{n}{x} \PY{o}{==} \PY{l+s+s1}{\PYZsq{}}\PY{l+s+s1}{o}\PY{l+s+s1}{\PYZsq{}}\PY{p}{:} \PY{k}{break}    
    \PY{n+nb}{print}\PY{p}{(}\PY{n}{x}\PY{p}{)}
\PY{k}{else}\PY{p}{:}       \PY{c+c1}{\PYZsh{} Else wird nicht Ausgeführt, da zuvor mit Break die Schleife beendet wurde!}
    \PY{n+nb}{print}\PY{p}{(}\PY{l+s+s2}{\PYZdq{}}\PY{l+s+s2}{Was ist PTHON?}\PY{l+s+s2}{\PYZdq{}}\PY{p}{)}
\end{Verbatim}
\end{tcolorbox}

    \begin{Verbatim}[commandchars=\\\{\}]
p
t
h
    \end{Verbatim}

    \hypertarget{try-except}{%
\subsection{Try-Except}\label{try-except}}

    \begin{tcolorbox}[breakable, size=fbox, boxrule=1pt, pad at break*=1mm,colback=cellbackground, colframe=cellborder]
\prompt{In}{incolor}{256}{\boxspacing}
\begin{Verbatim}[commandchars=\\\{\}]
\PY{c+c1}{\PYZsh{} Bei Fehlern schmeißt Pyhton eine Exception}

\PY{c+c1}{\PYZsh{}del x      \PYZsh{} KeyError: x}
\PY{c+c1}{\PYZsh{}print(x)   \PYZsh{} NameError: name \PYZsq{}x\PYZsq{} isn\PYZsq{}t defined}
\end{Verbatim}
\end{tcolorbox}

    \begin{tcolorbox}[breakable, size=fbox, boxrule=1pt, pad at break*=1mm,colback=cellbackground, colframe=cellborder]
\prompt{In}{incolor}{258}{\boxspacing}
\begin{Verbatim}[commandchars=\\\{\}]
\PY{c+c1}{\PYZsh{} Fehler können behandelt werden in Try\PYZhy{}Blöcken (Kein Guter Stil hier)}
\PY{k}{try}\PY{p}{:}
    \PY{k}{del} \PY{n}{x}
\PY{k}{except}\PY{p}{:}
    \PY{k}{pass}  \PY{c+c1}{\PYZsh{} Tue nichts}
\end{Verbatim}
\end{tcolorbox}

    \begin{tcolorbox}[breakable, size=fbox, boxrule=1pt, pad at break*=1mm,colback=cellbackground, colframe=cellborder]
\prompt{In}{incolor}{259}{\boxspacing}
\begin{Verbatim}[commandchars=\\\{\}]
\PY{c+c1}{\PYZsh{} Fehler können behandelt werden in Try\PYZhy{}Blöcken (Auch kein Guter Stil hier)}
\PY{k}{try}\PY{p}{:}
    \PY{n+nb}{print}\PY{p}{(}\PY{n}{x}\PY{p}{)}
\PY{k}{except}\PY{p}{:}
    \PY{n+nb}{print}\PY{p}{(}\PY{l+s+s2}{\PYZdq{}}\PY{l+s+s2}{Ein Fehler}\PY{l+s+s2}{\PYZdq{}}\PY{p}{)}

\PY{c+c1}{\PYZsh{} Mit vordefinierter Ausgabe}
\PY{k+kn}{import} \PY{n+nn}{sys} \PY{c+c1}{\PYZsh{}oder usys}
\PY{k}{try}\PY{p}{:}
    \PY{n+nb}{print}\PY{p}{(}\PY{n}{x}\PY{p}{)}
\PY{k}{except} \PY{n+ne}{Exception} \PY{k}{as} \PY{n}{err}\PY{p}{:}   \PY{c+c1}{\PYZsh{}Exception ist die Oberklasse aller Exceptions}
    \PY{n+nb}{print}\PY{p}{(}\PY{n}{err}\PY{p}{)}
    \PY{n}{sys}\PY{o}{.}\PY{n}{print\PYZus{}exception}\PY{p}{(}\PY{n}{err}\PY{p}{)}
\end{Verbatim}
\end{tcolorbox}

    \begin{Verbatim}[commandchars=\\\{\}]
Ein Fehler
name 'x' isn't defined
Traceback (most recent call last):
  File "<stdin>", line 10, in <module>
NameError: name 'x' isn't defined
    \end{Verbatim}

    \begin{tcolorbox}[breakable, size=fbox, boxrule=1pt, pad at break*=1mm,colback=cellbackground, colframe=cellborder]
\prompt{In}{incolor}{260}{\boxspacing}
\begin{Verbatim}[commandchars=\\\{\}]
\PY{c+c1}{\PYZsh{} Besser: es wird genau definiert, welcher Fehler erwartet wird}
\PY{k}{try}\PY{p}{:} 
    \PY{k}{del} \PY{n}{x}
    \PY{n+nb}{print}\PY{p}{(}\PY{n}{x}\PY{p}{)}   \PY{c+c1}{\PYZsh{} Bei erstem Fehler wird unterbrochen !}
\PY{k}{except} \PY{n+ne}{NameError}\PY{p}{:}
    \PY{n+nb}{print}\PY{p}{(}\PY{l+s+s2}{\PYZdq{}}\PY{l+s+s2}{Variable existiert nicht (NameError)}\PY{l+s+s2}{\PYZdq{}}\PY{p}{)}
\PY{k}{except} \PY{n+ne}{KeyError}\PY{p}{:}
    \PY{n+nb}{print}\PY{p}{(}\PY{l+s+s2}{\PYZdq{}}\PY{l+s+s2}{Variable existiert nicht (KeyError)}\PY{l+s+s2}{\PYZdq{}}\PY{p}{)}   
\end{Verbatim}
\end{tcolorbox}

    \begin{Verbatim}[commandchars=\\\{\}]
Variable existiert nicht (KeyError)
    \end{Verbatim}

    \begin{tcolorbox}[breakable, size=fbox, boxrule=1pt, pad at break*=1mm,colback=cellbackground, colframe=cellborder]
\prompt{In}{incolor}{261}{\boxspacing}
\begin{Verbatim}[commandchars=\\\{\}]
\PY{c+c1}{\PYZsh{} Else und Finally}
\PY{n}{x} \PY{o}{=} \PY{l+s+s2}{\PYZdq{}}\PY{l+s+s2}{Hallo}\PY{l+s+s2}{\PYZdq{}}
\PY{c+c1}{\PYZsh{}del x}

\PY{k}{try}\PY{p}{:} 
    \PY{c+c1}{\PYZsh{}del x}
    \PY{n+nb}{print}\PY{p}{(}\PY{n}{x}\PY{p}{)}
\PY{k}{except} \PY{n+ne}{NameError}\PY{p}{:}
    \PY{n+nb}{print}\PY{p}{(}\PY{l+s+s2}{\PYZdq{}}\PY{l+s+s2}{Variable existiert nicht (NameError)}\PY{l+s+s2}{\PYZdq{}}\PY{p}{)}
\PY{k}{else}\PY{p}{:}                                            \PY{c+c1}{\PYZsh{} Wird ausgeführt wenn es kein Fehler gab}
    \PY{n+nb}{print}\PY{p}{(}\PY{l+s+s2}{\PYZdq{}}\PY{l+s+s2}{Hat alles geklappt}\PY{l+s+s2}{\PYZdq{}}\PY{p}{)}
\PY{k}{finally}\PY{p}{:}                                         \PY{c+c1}{\PYZsh{} Wird ausgeführt in jedem Fall,}
    \PY{n+nb}{print}\PY{p}{(}\PY{l+s+s2}{\PYZdq{}}\PY{l+s+s2}{Wie auch immer... weiter gehts}\PY{l+s+s2}{\PYZdq{}}\PY{p}{)}      \PY{c+c1}{\PYZsh{}gut um z.B. in jedem Fall eine Verbindung zu trennen oder Dateien zu schließen}
\end{Verbatim}
\end{tcolorbox}

    \begin{Verbatim}[commandchars=\\\{\}]
Hallo
Hat alles geklappt
Wie auch immer{\ldots} weiter gehts
    \end{Verbatim}

    \begin{tcolorbox}[breakable, size=fbox, boxrule=1pt, pad at break*=1mm,colback=cellbackground, colframe=cellborder]
\prompt{In}{incolor}{262}{\boxspacing}
\begin{Verbatim}[commandchars=\\\{\}]
\PY{c+c1}{\PYZsh{} Exception selbst werfen}
\PY{k+kn}{import} \PY{n+nn}{sys}

\PY{k}{def} \PY{n+nf}{myFunction1}\PY{p}{(}\PY{n}{x}\PY{p}{)}\PY{p}{:}
    \PY{k}{if} \PY{o+ow}{not} \PY{n+nb}{type}\PY{p}{(}\PY{n}{x}\PY{p}{)} \PY{o+ow}{is} \PY{n+nb}{int}\PY{p}{:}
        \PY{k}{raise} \PY{n+ne}{Exception}\PY{p}{(}\PY{l+s+s2}{\PYZdq{}}\PY{l+s+s2}{Bitte Nummer angeben}\PY{l+s+s2}{\PYZdq{}}\PY{p}{)}
        
\PY{k}{def} \PY{n+nf}{myFunction2}\PY{p}{(}\PY{n}{x}\PY{p}{)}\PY{p}{:}
    \PY{k}{if} \PY{o+ow}{not} \PY{n+nb}{type}\PY{p}{(}\PY{n}{x}\PY{p}{)} \PY{o+ow}{is} \PY{n+nb}{int}\PY{p}{:}
        \PY{k}{raise} \PY{n+ne}{TypeError}\PY{p}{(}\PY{l+s+s2}{\PYZdq{}}\PY{l+s+s2}{Bitte Nummer angeben}\PY{l+s+s2}{\PYZdq{}}\PY{p}{)}

\PY{k}{def} \PY{n+nf}{myFunction3}\PY{p}{(}\PY{n}{x}\PY{p}{)}\PY{p}{:}
    \PY{k}{if} \PY{o+ow}{not} \PY{n+nb}{type}\PY{p}{(}\PY{n}{x}\PY{p}{)} \PY{o+ow}{is} \PY{n+nb}{int}\PY{p}{:}
        \PY{k}{raise} \PY{n+ne}{TypeError}\PY{p}{(}\PY{l+s+s2}{\PYZdq{}}\PY{l+s+s2}{Bitte Nummer angeben}\PY{l+s+s2}{\PYZdq{}}\PY{p}{)}
    \PY{k}{if} \PY{n}{x} \PY{o}{\PYZlt{}} \PY{l+m+mi}{0}\PY{p}{:}
        \PY{k}{raise} \PY{n+ne}{ValueError}\PY{p}{(}\PY{l+s+s2}{\PYZdq{}}\PY{l+s+s2}{Bitte Zahl größer 0 angeben}\PY{l+s+s2}{\PYZdq{}}\PY{p}{)}

\PY{k}{try}\PY{p}{:}
    \PY{n}{myFunction1}\PY{p}{(}\PY{l+s+s2}{\PYZdq{}}\PY{l+s+s2}{hallo}\PY{l+s+s2}{\PYZdq{}}\PY{p}{)}
\PY{k}{except} \PY{n+ne}{Exception} \PY{k}{as} \PY{n}{ex}\PY{p}{:}
    \PY{n+nb}{print}\PY{p}{(}\PY{n}{ex}\PY{p}{)}
    \PY{n}{sys}\PY{o}{.}\PY{n}{print\PYZus{}exception}\PY{p}{(}\PY{n}{ex}\PY{p}{)}
    
\PY{k}{try}\PY{p}{:}
    \PY{n}{myFunction2}\PY{p}{(}\PY{l+s+s2}{\PYZdq{}}\PY{l+s+s2}{hallo}\PY{l+s+s2}{\PYZdq{}}\PY{p}{)}
\PY{k}{except} \PY{n+ne}{Exception} \PY{k}{as} \PY{n}{ex}\PY{p}{:}
    \PY{n+nb}{print}\PY{p}{(}\PY{n}{ex}\PY{p}{)}
    \PY{n}{sys}\PY{o}{.}\PY{n}{print\PYZus{}exception}\PY{p}{(}\PY{n}{ex}\PY{p}{)}
    
\PY{k}{try}\PY{p}{:}
    \PY{n}{myFunction3}\PY{p}{(}\PY{l+s+s2}{\PYZdq{}}\PY{l+s+s2}{hallo}\PY{l+s+s2}{\PYZdq{}}\PY{p}{)}
\PY{k}{except} \PY{n+ne}{Exception} \PY{k}{as} \PY{n}{ex}\PY{p}{:}
    \PY{n+nb}{print}\PY{p}{(}\PY{n}{ex}\PY{p}{)}
    \PY{n}{sys}\PY{o}{.}\PY{n}{print\PYZus{}exception}\PY{p}{(}\PY{n}{ex}\PY{p}{)}
    
\PY{k}{try}\PY{p}{:}
    \PY{n}{myFunction3}\PY{p}{(}\PY{o}{\PYZhy{}}\PY{l+m+mi}{1}\PY{p}{)}
\PY{k}{except} \PY{n+ne}{Exception} \PY{k}{as} \PY{n}{ex}\PY{p}{:}
    \PY{n+nb}{print}\PY{p}{(}\PY{n}{ex}\PY{p}{)}
    \PY{n}{sys}\PY{o}{.}\PY{n}{print\PYZus{}exception}\PY{p}{(}\PY{n}{ex}\PY{p}{)}
    
\end{Verbatim}
\end{tcolorbox}

    \begin{Verbatim}[commandchars=\\\{\}]
Bitte Nummer angeben
Traceback (most recent call last):
  File "<stdin>", line 19, in <module>
  File "<stdin>", line 6, in myFunction1
Exception: Bitte Nummer angeben
Bitte Nummer angeben
Traceback (most recent call last):
  File "<stdin>", line 25, in <module>
  File "<stdin>", line 10, in myFunction2
TypeError: Bitte Nummer angeben
Bitte Nummer angeben
Traceback (most recent call last):
  File "<stdin>", line 31, in <module>
  File "<stdin>", line 14, in myFunction3
TypeError: Bitte Nummer angeben
Bitte Zahl größer 0 angeben
Traceback (most recent call last):
  File "<stdin>", line 37, in <module>
  File "<stdin>", line 16, in myFunction3
ValueError: Bitte Zahl größer 0 angeben
    \end{Verbatim}

    \begin{tcolorbox}[breakable, size=fbox, boxrule=1pt, pad at break*=1mm,colback=cellbackground, colframe=cellborder]
\prompt{In}{incolor}{201}{\boxspacing}
\begin{Verbatim}[commandchars=\\\{\}]
\PY{c+c1}{\PYZsh{} Liste von Exceptions}

\PY{c+c1}{\PYZsh{} https://docs.micropython.org/en/latest/library/builtins.html?highlight=except\PYZsh{}exceptions}
\PY{k+kn}{import} \PY{n+nn}{builtins}

\PY{n}{list\PYZus{}no\PYZus{}exception} \PY{o}{=} \PY{p}{[}\PY{l+s+s2}{\PYZdq{}}\PY{l+s+s2}{Ellipsis}\PY{l+s+s2}{\PYZdq{}}\PY{p}{]}
\PY{k}{for} \PY{n}{x} \PY{o+ow}{in} \PY{n+nb}{dir}\PY{p}{(}\PY{n}{builtins}\PY{p}{)}\PY{p}{:}
    \PY{k}{if} \PY{n}{x}\PY{p}{[}\PY{l+m+mi}{0}\PY{p}{]}\PY{o}{.}\PY{n}{isupper}\PY{p}{(}\PY{p}{)} \PY{o+ow}{and} \PY{n}{x} \PY{o+ow}{not} \PY{o+ow}{in} \PY{n}{list\PYZus{}no\PYZus{}exception}\PY{p}{:}
        \PY{n+nb}{print}\PY{p}{(}\PY{n}{x}\PY{p}{)}
\end{Verbatim}
\end{tcolorbox}

    \begin{Verbatim}[commandchars=\\\{\}]
ArithmeticError
AssertionError
AttributeError
BaseException
EOFError
Exception
GeneratorExit
ImportError
IndentationError
IndexError
KeyError
KeyboardInterrupt
LookupError
MemoryError
NameError
NotImplementedError
OSError
OverflowError
RuntimeError
StopIteration
SyntaxError
SystemExit
TypeError
ValueError
ZeroDivisionError
StopAsyncIteration
UnicodeError
ViperTypeError
    \end{Verbatim}

    \hypertarget{funktionen}{%
\section{Funktionen}\label{funktionen}}

    \begin{tcolorbox}[breakable, size=fbox, boxrule=1pt, pad at break*=1mm,colback=cellbackground, colframe=cellborder]
\prompt{In}{incolor}{4}{\boxspacing}
\begin{Verbatim}[commandchars=\\\{\}]
\PY{c+c1}{\PYZsh{} Einfache Funktionen definieren}

\PY{k}{def} \PY{n+nf}{myFunction}\PY{p}{(}\PY{p}{)}\PY{p}{:}
    \PY{n+nb}{print}\PY{p}{(}\PY{l+s+s2}{\PYZdq{}}\PY{l+s+s2}{Dies ist meine Funktion}\PY{l+s+s2}{\PYZdq{}}\PY{p}{)}

\PY{n}{myFunction}\PY{p}{(}\PY{p}{)}

\PY{n+nb}{print}\PY{p}{(}\PY{n}{callable}\PY{p}{(}\PY{n}{myFunction}\PY{p}{)}\PY{p}{)} \PY{c+c1}{\PYZsh{} Überprüfung: ist das Objekt aufrufbar?}
\end{Verbatim}
\end{tcolorbox}

    \begin{Verbatim}[commandchars=\\\{\}]
Dies ist meine Funktion
True
    \end{Verbatim}

    \begin{tcolorbox}[breakable, size=fbox, boxrule=1pt, pad at break*=1mm,colback=cellbackground, colframe=cellborder]
\prompt{In}{incolor}{175}{\boxspacing}
\begin{Verbatim}[commandchars=\\\{\}]
\PY{c+c1}{\PYZsh{} Funktion mit einfachen Parameter}
\PY{k}{def} \PY{n+nf}{myFunction}\PY{p}{(}\PY{n}{name1}\PY{p}{,} \PY{n}{name2}\PY{p}{)}\PY{p}{:}
    \PY{n+nb}{print}\PY{p}{(}\PY{l+s+s2}{\PYZdq{}}\PY{l+s+s2}{Dies ist die Funktion von }\PY{l+s+s2}{\PYZdq{}}\PY{o}{+} \PY{n}{name1} \PY{o}{+} \PY{l+s+s2}{\PYZdq{}}\PY{l+s+s2}{ und }\PY{l+s+s2}{\PYZdq{}} \PY{o}{+} \PY{n}{name2}\PY{p}{)}

\PY{n}{myFunction}\PY{p}{(}\PY{l+s+s2}{\PYZdq{}}\PY{l+s+s2}{dir}\PY{l+s+s2}{\PYZdq{}}\PY{p}{,}\PY{l+s+s2}{\PYZdq{}}\PY{l+s+s2}{mir}\PY{l+s+s2}{\PYZdq{}}\PY{p}{)}
\PY{c+c1}{\PYZsh{} myFunction(3) \PYZsh{}! hier gibt es einen Fehler, Datentyp passt nicht!}
\PY{n}{myFunction}\PY{p}{(}\PY{n}{name2} \PY{o}{=} \PY{l+s+s2}{\PYZdq{}}\PY{l+s+s2}{mir}\PY{l+s+s2}{\PYZdq{}}\PY{p}{,} \PY{n}{name1}\PY{o}{=} \PY{l+s+s2}{\PYZdq{}}\PY{l+s+s2}{dir}\PY{l+s+s2}{\PYZdq{}}\PY{p}{)} \PY{c+c1}{\PYZsh{} Parameter werden explicit angegeben}
\end{Verbatim}
\end{tcolorbox}

    \begin{Verbatim}[commandchars=\\\{\}]
Dies ist die Funktion von dir und mir
Dies ist die Funktion von dir und mir
    \end{Verbatim}

    \begin{tcolorbox}[breakable, size=fbox, boxrule=1pt, pad at break*=1mm,colback=cellbackground, colframe=cellborder]
\prompt{In}{incolor}{173}{\boxspacing}
\begin{Verbatim}[commandchars=\\\{\}]
\PY{c+c1}{\PYZsh{} Funktion mit default\PYZhy{}Parametern}
\PY{k}{def} \PY{n+nf}{myFunction}\PY{p}{(}\PY{n}{name1} \PY{o}{=}\PY{l+s+s2}{\PYZdq{}}\PY{l+s+s2}{dir}\PY{l+s+s2}{\PYZdq{}}\PY{p}{,} \PY{n}{name2} \PY{o}{=} \PY{l+s+s2}{\PYZdq{}}\PY{l+s+s2}{mir}\PY{l+s+s2}{\PYZdq{}}\PY{p}{)}\PY{p}{:}
    \PY{n+nb}{print}\PY{p}{(}\PY{l+s+s2}{\PYZdq{}}\PY{l+s+s2}{Dies ist die Funktion von }\PY{l+s+s2}{\PYZdq{}}\PY{o}{+} \PY{n}{name1} \PY{o}{+} \PY{l+s+s2}{\PYZdq{}}\PY{l+s+s2}{ und }\PY{l+s+s2}{\PYZdq{}} \PY{o}{+} \PY{n}{name2}\PY{p}{)}

\PY{n}{myFunction}\PY{p}{(}\PY{p}{)}
\end{Verbatim}
\end{tcolorbox}

    \begin{Verbatim}[commandchars=\\\{\}]
Dies ist die Funktion von dir und mir
    \end{Verbatim}

    \begin{tcolorbox}[breakable, size=fbox, boxrule=1pt, pad at break*=1mm,colback=cellbackground, colframe=cellborder]
\prompt{In}{incolor}{179}{\boxspacing}
\begin{Verbatim}[commandchars=\\\{\}]
\PY{c+c1}{\PYZsh{} Funktion mit beliebiger Anzahl von Parametern}
\PY{k}{def} \PY{n+nf}{myFunction}\PY{p}{(}\PY{o}{*}\PY{n}{namen}\PY{p}{)}\PY{p}{:}
    \PY{n+nb}{print}\PY{p}{(}\PY{l+s+s2}{\PYZdq{}}\PY{l+s+s2}{Dies ist die Funktion von: }\PY{l+s+s2}{\PYZdq{}}\PY{p}{)}
    \PY{k}{for} \PY{n}{name} \PY{o+ow}{in} \PY{n}{namen}\PY{p}{:}
        \PY{n+nb}{print}\PY{p}{(}\PY{n}{name}\PY{p}{)}

\PY{n}{myFunction}\PY{p}{(}\PY{l+s+s2}{\PYZdq{}}\PY{l+s+s2}{Tick}\PY{l+s+s2}{\PYZdq{}}\PY{p}{,} \PY{l+s+s2}{\PYZdq{}}\PY{l+s+s2}{Trick}\PY{l+s+s2}{\PYZdq{}}\PY{p}{,} \PY{l+s+s2}{\PYZdq{}}\PY{l+s+s2}{Track}\PY{l+s+s2}{\PYZdq{}}\PY{p}{)}
\end{Verbatim}
\end{tcolorbox}

    \begin{Verbatim}[commandchars=\\\{\}]
Dies ist die Funktion von:
Tick
Trick
Track
    \end{Verbatim}

    \begin{tcolorbox}[breakable, size=fbox, boxrule=1pt, pad at break*=1mm,colback=cellbackground, colframe=cellborder]
\prompt{In}{incolor}{182}{\boxspacing}
\begin{Verbatim}[commandchars=\\\{\}]
\PY{k}{def} \PY{n+nf}{myFunction}\PY{p}{(}\PY{n}{vorname}\PY{p}{,} \PY{n}{nachname}\PY{p}{)}\PY{p}{:}
    \PY{k}{return}\PY{p}{(}\PY{l+s+s2}{\PYZdq{}}\PY{l+s+s2}{Sein Nachname lautet }\PY{l+s+s2}{\PYZdq{}} \PY{o}{+} \PY{n}{nachname}\PY{p}{)}

\PY{n}{lname} \PY{o}{=} \PY{n}{myFunction}\PY{p}{(}\PY{l+s+s2}{\PYZdq{}}\PY{l+s+s2}{Donald}\PY{l+s+s2}{\PYZdq{}}\PY{p}{,} \PY{l+s+s2}{\PYZdq{}}\PY{l+s+s2}{Duck}\PY{l+s+s2}{\PYZdq{}}\PY{p}{)}
\PY{n+nb}{print}\PY{p}{(}\PY{n}{lname}\PY{p}{)}
\end{Verbatim}
\end{tcolorbox}

    \begin{Verbatim}[commandchars=\\\{\}]
Sein Nachname lautet Duck
    \end{Verbatim}

    \begin{tcolorbox}[breakable, size=fbox, boxrule=1pt, pad at break*=1mm,colback=cellbackground, colframe=cellborder]
\prompt{In}{incolor}{183}{\boxspacing}
\begin{Verbatim}[commandchars=\\\{\}]
\PY{k}{def} \PY{n+nf}{myFunction}\PY{p}{(}\PY{n}{vorname}\PY{p}{,} \PY{n}{nachname}\PY{p}{)}\PY{p}{:}
    \PY{n}{vorname} \PY{o}{=} \PY{l+s+s2}{\PYZdq{}}\PY{l+s+s2}{Dagobert}\PY{l+s+s2}{\PYZdq{}}
    \PY{k}{return}\PY{p}{(}\PY{l+s+s2}{\PYZdq{}}\PY{l+s+s2}{Sein Nachname lautet }\PY{l+s+s2}{\PYZdq{}} \PY{o}{+} \PY{n}{nachname}\PY{p}{)}
    

\PY{n}{vorname} \PY{o}{=} \PY{l+s+s2}{\PYZdq{}}\PY{l+s+s2}{Donald}\PY{l+s+s2}{\PYZdq{}}\PY{p}{;} \PY{n}{nachname}\PY{o}{=}\PY{l+s+s2}{\PYZdq{}}\PY{l+s+s2}{Duck}\PY{l+s+s2}{\PYZdq{}}
\PY{n+nb}{print}\PY{p}{(}\PY{n}{myFunction}\PY{p}{(}\PY{l+s+s2}{\PYZdq{}}\PY{l+s+s2}{Donald}\PY{l+s+s2}{\PYZdq{}}\PY{p}{,} \PY{l+s+s2}{\PYZdq{}}\PY{l+s+s2}{Duck}\PY{l+s+s2}{\PYZdq{}}\PY{p}{)}\PY{p}{)}
\PY{n+nb}{print}\PY{p}{(}\PY{n}{vorname}\PY{p}{)}
\end{Verbatim}
\end{tcolorbox}

    \begin{Verbatim}[commandchars=\\\{\}]
Sein Nachname lautet Duck
Donald
    \end{Verbatim}

    \begin{tcolorbox}[breakable, size=fbox, boxrule=1pt, pad at break*=1mm,colback=cellbackground, colframe=cellborder]
\prompt{In}{incolor}{178}{\boxspacing}
\begin{Verbatim}[commandchars=\\\{\}]
\PY{c+c1}{\PYZsh{} Funktionsparameter als Dictionary}
\PY{k}{def} \PY{n+nf}{myFunction}\PY{p}{(}\PY{o}{*}\PY{o}{*}\PY{n}{names}\PY{p}{)}\PY{p}{:}
    \PY{n+nb}{print}\PY{p}{(}\PY{l+s+s2}{\PYZdq{}}\PY{l+s+s2}{Sein Nachname lautet }\PY{l+s+s2}{\PYZdq{}} \PY{o}{+} \PY{n}{names}\PY{p}{[}\PY{l+s+s2}{\PYZdq{}}\PY{l+s+s2}{lname}\PY{l+s+s2}{\PYZdq{}}\PY{p}{]}\PY{p}{)}

\PY{n}{myFunction}\PY{p}{(}\PY{n}{fname} \PY{o}{=} \PY{l+s+s2}{\PYZdq{}}\PY{l+s+s2}{Donald}\PY{l+s+s2}{\PYZdq{}}\PY{p}{,} \PY{n}{lname} \PY{o}{=} \PY{l+s+s2}{\PYZdq{}}\PY{l+s+s2}{Duck}\PY{l+s+s2}{\PYZdq{}}\PY{p}{)}
\end{Verbatim}
\end{tcolorbox}

    \hypertarget{lambda-funktionen}{%
\subsection{Lambda-Funktionen}\label{lambda-funktionen}}

    \begin{tcolorbox}[breakable, size=fbox, boxrule=1pt, pad at break*=1mm,colback=cellbackground, colframe=cellborder]
\prompt{In}{incolor}{192}{\boxspacing}
\begin{Verbatim}[commandchars=\\\{\}]
\PY{c+c1}{\PYZsh{} Syntax: lambda arguments : expression}
\PY{n}{x} \PY{o}{=} \PY{k}{lambda} \PY{n}{a} \PY{p}{:} \PY{n}{a} \PY{o}{+} \PY{l+m+mi}{10}
\PY{n+nb}{print}\PY{p}{(}\PY{n}{x}\PY{p}{(}\PY{l+m+mi}{5}\PY{p}{)}\PY{p}{)}
\end{Verbatim}
\end{tcolorbox}

    \begin{Verbatim}[commandchars=\\\{\}]
15
    \end{Verbatim}

    \begin{tcolorbox}[breakable, size=fbox, boxrule=1pt, pad at break*=1mm,colback=cellbackground, colframe=cellborder]
\prompt{In}{incolor}{196}{\boxspacing}
\begin{Verbatim}[commandchars=\\\{\}]
\PY{c+c1}{\PYZsh{} Syntax: lambda arguments : expression}
\PY{n}{x} \PY{o}{=} \PY{k}{lambda} \PY{n}{a}\PY{p}{,} \PY{n}{b} \PY{p}{:} \PY{n}{a} \PY{o}{*} \PY{n}{b}
\PY{n+nb}{print}\PY{p}{(}\PY{n}{x}\PY{p}{(}\PY{l+m+mi}{5}\PY{p}{,}\PY{l+m+mi}{3}\PY{p}{)}\PY{p}{)}
\end{Verbatim}
\end{tcolorbox}

    \begin{Verbatim}[commandchars=\\\{\}]
15
    \end{Verbatim}

    \begin{tcolorbox}[breakable, size=fbox, boxrule=1pt, pad at break*=1mm,colback=cellbackground, colframe=cellborder]
\prompt{In}{incolor}{198}{\boxspacing}
\begin{Verbatim}[commandchars=\\\{\}]
\PY{c+c1}{\PYZsh{} Die Funktion gibt eine Funktion zurück, der Parameter ersetzt n}
\PY{k}{def} \PY{n+nf}{myfunc}\PY{p}{(}\PY{n}{n}\PY{p}{)}\PY{p}{:}
    \PY{k}{return} \PY{k}{lambda} \PY{n}{a} \PY{p}{:} \PY{n}{a} \PY{o}{*} \PY{n}{n}

\PY{n}{mydoubler} \PY{o}{=} \PY{n}{myfunc}\PY{p}{(}\PY{l+m+mi}{2}\PY{p}{)}
\PY{n}{mytripler} \PY{o}{=} \PY{n}{myfunc}\PY{p}{(}\PY{l+m+mi}{3}\PY{p}{)}

\PY{n+nb}{print}\PY{p}{(}\PY{n}{mydoubler}\PY{p}{(}\PY{l+m+mi}{11}\PY{p}{)}\PY{p}{)}
\PY{n+nb}{print}\PY{p}{(}\PY{n}{mytripler}\PY{p}{(}\PY{l+m+mi}{11}\PY{p}{)}\PY{p}{)}
\end{Verbatim}
\end{tcolorbox}

    \begin{Verbatim}[commandchars=\\\{\}]
22
33
    \end{Verbatim}

    \hypertarget{spezialthema-byvalue-oder-byreference}{%
\subsection{Spezialthema: ByValue oder
ByReference?}\label{spezialthema-byvalue-oder-byreference}}

Quelle:
https://stackoverflow.com/questions/986006/how-do-i-pass-a-variable-by-reference

    \begin{tcolorbox}[breakable, size=fbox, boxrule=1pt, pad at break*=1mm,colback=cellbackground, colframe=cellborder]
\prompt{In}{incolor}{184}{\boxspacing}
\begin{Verbatim}[commandchars=\\\{\}]
\PY{c+c1}{\PYZsh{} Listen sind veränderbar (mutable type)}
\PY{k}{def} \PY{n+nf}{try\PYZus{}to\PYZus{}change\PYZus{}list\PYZus{}contents}\PY{p}{(}\PY{n}{the\PYZus{}list}\PY{p}{)}\PY{p}{:}
    \PY{n+nb}{print}\PY{p}{(}\PY{l+s+s1}{\PYZsq{}}\PY{l+s+s1}{got}\PY{l+s+s1}{\PYZsq{}}\PY{p}{,} \PY{n}{the\PYZus{}list}\PY{p}{)}
    \PY{n}{the\PYZus{}list}\PY{o}{.}\PY{n}{append}\PY{p}{(}\PY{l+s+s1}{\PYZsq{}}\PY{l+s+s1}{four}\PY{l+s+s1}{\PYZsq{}}\PY{p}{)}
    \PY{n+nb}{print}\PY{p}{(}\PY{l+s+s1}{\PYZsq{}}\PY{l+s+s1}{changed to}\PY{l+s+s1}{\PYZsq{}}\PY{p}{,} \PY{n}{the\PYZus{}list}\PY{p}{)}

\PY{n}{outer\PYZus{}list} \PY{o}{=} \PY{p}{[}\PY{l+s+s1}{\PYZsq{}}\PY{l+s+s1}{one}\PY{l+s+s1}{\PYZsq{}}\PY{p}{,} \PY{l+s+s1}{\PYZsq{}}\PY{l+s+s1}{two}\PY{l+s+s1}{\PYZsq{}}\PY{p}{,} \PY{l+s+s1}{\PYZsq{}}\PY{l+s+s1}{three}\PY{l+s+s1}{\PYZsq{}}\PY{p}{]}

\PY{n+nb}{print}\PY{p}{(}\PY{l+s+s1}{\PYZsq{}}\PY{l+s+s1}{before, outer\PYZus{}list =}\PY{l+s+s1}{\PYZsq{}}\PY{p}{,} \PY{n}{outer\PYZus{}list}\PY{p}{)}
\PY{n}{try\PYZus{}to\PYZus{}change\PYZus{}list\PYZus{}contents}\PY{p}{(}\PY{n}{outer\PYZus{}list}\PY{p}{)}
\PY{n+nb}{print}\PY{p}{(}\PY{l+s+s1}{\PYZsq{}}\PY{l+s+s1}{after, outer\PYZus{}list =}\PY{l+s+s1}{\PYZsq{}}\PY{p}{,} \PY{n}{outer\PYZus{}list}\PY{p}{)}
\end{Verbatim}
\end{tcolorbox}

    \begin{Verbatim}[commandchars=\\\{\}]
before, outer\_list = ['one', 'two', 'three']
got ['one', 'two', 'three']
changed to ['one', 'two', 'three', 'four']
after, outer\_list = ['one', 'two', 'three', 'four']
    \end{Verbatim}

    \begin{tcolorbox}[breakable, size=fbox, boxrule=1pt, pad at break*=1mm,colback=cellbackground, colframe=cellborder]
\prompt{In}{incolor}{185}{\boxspacing}
\begin{Verbatim}[commandchars=\\\{\}]
\PY{c+c1}{\PYZsh{} Wenn die Referenz sich ändert bleibt die original Liste unverändert}
\PY{k}{def} \PY{n+nf}{try\PYZus{}to\PYZus{}change\PYZus{}list\PYZus{}reference}\PY{p}{(}\PY{n}{the\PYZus{}list}\PY{p}{)}\PY{p}{:}
    \PY{n+nb}{print}\PY{p}{(}\PY{l+s+s1}{\PYZsq{}}\PY{l+s+s1}{got}\PY{l+s+s1}{\PYZsq{}}\PY{p}{,} \PY{n}{the\PYZus{}list}\PY{p}{)}
    \PY{n}{the\PYZus{}list} \PY{o}{=} \PY{p}{[}\PY{l+s+s1}{\PYZsq{}}\PY{l+s+s1}{and}\PY{l+s+s1}{\PYZsq{}}\PY{p}{,} \PY{l+s+s1}{\PYZsq{}}\PY{l+s+s1}{we}\PY{l+s+s1}{\PYZsq{}}\PY{p}{,} \PY{l+s+s1}{\PYZsq{}}\PY{l+s+s1}{can}\PY{l+s+s1}{\PYZsq{}}\PY{p}{,} \PY{l+s+s1}{\PYZsq{}}\PY{l+s+s1}{not}\PY{l+s+s1}{\PYZsq{}}\PY{p}{,} \PY{l+s+s1}{\PYZsq{}}\PY{l+s+s1}{lie}\PY{l+s+s1}{\PYZsq{}}\PY{p}{]}
    \PY{n+nb}{print}\PY{p}{(}\PY{l+s+s1}{\PYZsq{}}\PY{l+s+s1}{set to}\PY{l+s+s1}{\PYZsq{}}\PY{p}{,} \PY{n}{the\PYZus{}list}\PY{p}{)}

\PY{n}{outer\PYZus{}list} \PY{o}{=} \PY{p}{[}\PY{l+s+s1}{\PYZsq{}}\PY{l+s+s1}{we}\PY{l+s+s1}{\PYZsq{}}\PY{p}{,} \PY{l+s+s1}{\PYZsq{}}\PY{l+s+s1}{like}\PY{l+s+s1}{\PYZsq{}}\PY{p}{,} \PY{l+s+s1}{\PYZsq{}}\PY{l+s+s1}{proper}\PY{l+s+s1}{\PYZsq{}}\PY{p}{,} \PY{l+s+s1}{\PYZsq{}}\PY{l+s+s1}{English}\PY{l+s+s1}{\PYZsq{}}\PY{p}{]}

\PY{n+nb}{print}\PY{p}{(}\PY{l+s+s1}{\PYZsq{}}\PY{l+s+s1}{before, outer\PYZus{}list =}\PY{l+s+s1}{\PYZsq{}}\PY{p}{,} \PY{n}{outer\PYZus{}list}\PY{p}{)}
\PY{n}{try\PYZus{}to\PYZus{}change\PYZus{}list\PYZus{}reference}\PY{p}{(}\PY{n}{outer\PYZus{}list}\PY{p}{)}
\PY{n+nb}{print}\PY{p}{(}\PY{l+s+s1}{\PYZsq{}}\PY{l+s+s1}{after, outer\PYZus{}list =}\PY{l+s+s1}{\PYZsq{}}\PY{p}{,} \PY{n}{outer\PYZus{}list}\PY{p}{)}
\end{Verbatim}
\end{tcolorbox}

    \begin{Verbatim}[commandchars=\\\{\}]
before, outer\_list = ['we', 'like', 'proper', 'English']
got ['we', 'like', 'proper', 'English']
set to ['and', 'we', 'can', 'not', 'lie']
after, outer\_list = ['we', 'like', 'proper', 'English']
    \end{Verbatim}

    \begin{tcolorbox}[breakable, size=fbox, boxrule=1pt, pad at break*=1mm,colback=cellbackground, colframe=cellborder]
\prompt{In}{incolor}{186}{\boxspacing}
\begin{Verbatim}[commandchars=\\\{\}]
\PY{c+c1}{\PYZsh{} Strings sind nicht veränderbar (immutable type)}
\PY{k}{def} \PY{n+nf}{try\PYZus{}to\PYZus{}change\PYZus{}string\PYZus{}reference}\PY{p}{(}\PY{n}{the\PYZus{}string}\PY{p}{)}\PY{p}{:}
    \PY{n+nb}{print}\PY{p}{(}\PY{l+s+s1}{\PYZsq{}}\PY{l+s+s1}{got}\PY{l+s+s1}{\PYZsq{}}\PY{p}{,} \PY{n}{the\PYZus{}string}\PY{p}{)}
    \PY{n}{the\PYZus{}string} \PY{o}{=} \PY{l+s+s1}{\PYZsq{}}\PY{l+s+s1}{In a kingdom by the sea}\PY{l+s+s1}{\PYZsq{}}
    \PY{n+nb}{print}\PY{p}{(}\PY{l+s+s1}{\PYZsq{}}\PY{l+s+s1}{set to}\PY{l+s+s1}{\PYZsq{}}\PY{p}{,} \PY{n}{the\PYZus{}string}\PY{p}{)}

\PY{n}{outer\PYZus{}string} \PY{o}{=} \PY{l+s+s1}{\PYZsq{}}\PY{l+s+s1}{It was many and many a year ago}\PY{l+s+s1}{\PYZsq{}}

\PY{n+nb}{print}\PY{p}{(}\PY{l+s+s1}{\PYZsq{}}\PY{l+s+s1}{before, outer\PYZus{}string =}\PY{l+s+s1}{\PYZsq{}}\PY{p}{,} \PY{n}{outer\PYZus{}string}\PY{p}{)}
\PY{n}{try\PYZus{}to\PYZus{}change\PYZus{}string\PYZus{}reference}\PY{p}{(}\PY{n}{outer\PYZus{}string}\PY{p}{)}
\PY{n+nb}{print}\PY{p}{(}\PY{l+s+s1}{\PYZsq{}}\PY{l+s+s1}{after, outer\PYZus{}string =}\PY{l+s+s1}{\PYZsq{}}\PY{p}{,} \PY{n}{outer\PYZus{}string}\PY{p}{)}
\end{Verbatim}
\end{tcolorbox}

    \begin{tcolorbox}[breakable, size=fbox, boxrule=1pt, pad at break*=1mm,colback=cellbackground, colframe=cellborder]
\prompt{In}{incolor}{189}{\boxspacing}
\begin{Verbatim}[commandchars=\\\{\}]
\PY{c+c1}{\PYZsh{} Strings sind nicht veränderbar (immutable type)}
\PY{k}{def} \PY{n+nf}{try\PYZus{}to\PYZus{}change\PYZus{}string\PYZus{}reference}\PY{p}{(}\PY{n}{the\PYZus{}string}\PY{p}{)}\PY{p}{:}
    \PY{n+nb}{print}\PY{p}{(}\PY{l+s+s1}{\PYZsq{}}\PY{l+s+s1}{got}\PY{l+s+s1}{\PYZsq{}}\PY{p}{,} \PY{n}{the\PYZus{}string}\PY{p}{)}
    \PY{n}{the\PYZus{}string}\PY{o}{.}\PY{n}{replace}\PY{p}{(}\PY{l+s+s2}{\PYZdq{}}\PY{l+s+s2}{was}\PY{l+s+s2}{\PYZdq{}}\PY{p}{,}\PY{l+s+s2}{\PYZdq{}}\PY{l+s+s2}{is}\PY{l+s+s2}{\PYZdq{}}\PY{p}{)}
    \PY{n+nb}{print}\PY{p}{(}\PY{l+s+s1}{\PYZsq{}}\PY{l+s+s1}{set to}\PY{l+s+s1}{\PYZsq{}}\PY{p}{,} \PY{n}{the\PYZus{}string}\PY{p}{)}

\PY{n}{outer\PYZus{}string} \PY{o}{=} \PY{l+s+s1}{\PYZsq{}}\PY{l+s+s1}{It was many and many a year ago}\PY{l+s+s1}{\PYZsq{}}

\PY{n+nb}{print}\PY{p}{(}\PY{l+s+s1}{\PYZsq{}}\PY{l+s+s1}{before, outer\PYZus{}string =}\PY{l+s+s1}{\PYZsq{}}\PY{p}{,} \PY{n}{outer\PYZus{}string}\PY{p}{)}
\PY{n}{try\PYZus{}to\PYZus{}change\PYZus{}string\PYZus{}reference}\PY{p}{(}\PY{n}{outer\PYZus{}string}\PY{p}{)}
\PY{n+nb}{print}\PY{p}{(}\PY{l+s+s1}{\PYZsq{}}\PY{l+s+s1}{after, outer\PYZus{}string =}\PY{l+s+s1}{\PYZsq{}}\PY{p}{,} \PY{n}{outer\PYZus{}string}\PY{p}{)}
\end{Verbatim}
\end{tcolorbox}

    \begin{Verbatim}[commandchars=\\\{\}]
before, outer\_string = It was many and many a year ago
got It was many and many a year ago
set to It was many and many a year ago
after, outer\_string = It was many and many a year ago
    \end{Verbatim}

    Listen (reversed, filter, slice)

    \hypertarget{klassen-todo}{%
\section{Klassen (ToDo)}\label{klassen-todo}}

    \begin{tcolorbox}[breakable, size=fbox, boxrule=1pt, pad at break*=1mm,colback=cellbackground, colframe=cellborder]
\prompt{In}{incolor}{204}{\boxspacing}
\begin{Verbatim}[commandchars=\\\{\}]

\end{Verbatim}
\end{tcolorbox}

    \begin{tcolorbox}[breakable, size=fbox, boxrule=1pt, pad at break*=1mm,colback=cellbackground, colframe=cellborder]
\prompt{In}{incolor}{204}{\boxspacing}
\begin{Verbatim}[commandchars=\\\{\}]

\end{Verbatim}
\end{tcolorbox}

    \hypertarget{module-todo}{%
\section{Module (ToDo)}\label{module-todo}}

    \begin{tcolorbox}[breakable, size=fbox, boxrule=1pt, pad at break*=1mm,colback=cellbackground, colframe=cellborder]
\prompt{In}{incolor}{ }{\boxspacing}
\begin{Verbatim}[commandchars=\\\{\}]

\end{Verbatim}
\end{tcolorbox}

    \begin{tcolorbox}[breakable, size=fbox, boxrule=1pt, pad at break*=1mm,colback=cellbackground, colframe=cellborder]
\prompt{In}{incolor}{ }{\boxspacing}
\begin{Verbatim}[commandchars=\\\{\}]

\end{Verbatim}
\end{tcolorbox}

    \hypertarget{disconnect}{%
\section{Disconnect}\label{disconnect}}

    \begin{tcolorbox}[breakable, size=fbox, boxrule=1pt, pad at break*=1mm,colback=cellbackground, colframe=cellborder]
\prompt{In}{incolor}{149}{\boxspacing}
\begin{Verbatim}[commandchars=\\\{\}]
\PY{o}{\PYZpc{}}\PY{k}{disconnect}
\end{Verbatim}
\end{tcolorbox}

    \begin{Verbatim}[commandchars=\\\{\}]
\textcolor{ansi-blue}{attempt to exit paste mode
}\textcolor{ansi-blue}{[\textbackslash{}r\textbackslash{}x03\textbackslash{}x02] }b'\textbackslash{}r\textbackslash{}nMicroPython v1.14 on 2021-02-02; Raspberry Pi
Pico with RP2040\textbackslash{}r\textbackslash{}nType "help()" for more information.\textbackslash{}r\textbackslash{}n>>> '\textcolor{ansi-blue}{
Closing serial Serial<id=0x2093429aca0, open=True>(port='COM13',
baudrate=115200, bytesize=8, parity='N', stopbits=1, timeout=0.5, xonxoff=False,
rtscts=False, dsrdtr=False)
}
    \end{Verbatim}


    % Add a bibliography block to the postdoc
    
    
    
\end{document}
